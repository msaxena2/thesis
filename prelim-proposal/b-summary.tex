\noindent {\bf \large Overview:}

\noindent Preventable medical errors, characterized by adverse events due to human and
system factors (as opposed to diseases), are the third leading cause of mortality
in the United States \cite{MakaryBMJ16}. To mitigate these errors,
medical associations routinely publish evidence-based statements for various scenarios called
Best Practice Guidelines (\BPGs{}) \cite{field1990clinical}. For \BPGs{} to be
effective, they must integrated into care flow, and be readily available.
To this end, hospitals develop and employ systems that assist Healthcare Professionals
(\HCPs{}) adhere to \BPGs{}, called Clinical Guidance Systems (\CGSs{}).
Evidence from multi-center trials suggests \CGSs{} are effective at preventing medical
errors \cite{BenettJAMIA16,SahotaJIS11}.

Research on \CGSs{} generally falls into two categories. The first
is the development and evaluation of \CGSs{} for various \BPGs{}. This typically
involes translation of the \emph{informal}, \emph{non-executable} \BPG{} to executable code (\BPGLogic)
that is coupled with hospital-specific components such as a Graphical User
Interface (GUI), sensors, etc to build a holistic system.
Howerver, existing work hasn't been aimed at:
\begin{enumerate*}[label=(\alph*)]
  \item developing a modular architecture that allows sharing \BPGLogic{}
    \emph{safely} across different components, and,
  \item establishing semantic equivalence between the \BPGLogic{} and the
    \BPG{}.
\end{enumerate*}
The second category focuses on the development
and use of formal techniques to model and analyze \BPGs{}
to improve their quality. This has lead
the use of existing, and development of new languages
for expressing \BPGs{} that support efficient analysis. But, these
languages arent't intended to be used by physicians, and
rely on computer scientists to comprehend the \BPG{} and build a
model in the underlying formal language for analysis.
While such work has improved \BPG-quality, it's goal is not to address the gap between the \BPG{}
and the analyzed model.
This proposal attempts to complement existing research
by introducing:
\begin{enumerate*}[label=(\roman*)]
  \item a new Domain Specific Language (DSL),
    for expressing \BPGs{} that is \emph{\underline{comprehensible by physicians}}, and
    \emph{\underline{executable}}
    to enable medical associations to publish executable \BPGs{} that can serve as
    \BPGLogic{} in \CGSs{}, and,
 \item a \emph{\underline{modular}} architecture that allows said
   \BPGs{} to be combined with hospital-specific components while maintaining \emph{\underline{holistic system safety}}.
\end{enumerate*}
%As shown in \cite{SahotaJIS11}, often multiple \CGSs{} are
%developed for the same \BPG{}, with slight differences to cater to hospital's specific
%needs. The second category focuses on the development of use of formal analysis
%techniques to verify properties (such as completeness, correctness) of \BPGs{}.
%\begin{enumerate*}[label=(\roman*)]
%  \item development of \CGS{} for specific \BPGs{} and studying their
%    effectiveness , and,
%  \item development and application of formal analysis techniques to verify
%    properties (such as completeness, correctness) of \BPGs{}.
%\end{enumerate*}

%Such errors can be ensuring that treatment conforms to
%Best Practice Guidelines (\BPGs{}). These are systematically developed, evidence-based
%statements for handling various patient conditions published by medical associations
%\cite{field1990clinical}. Software systems that assist Healthcare Professionals
%(\HCPs{}) adhere to \BPGs{}, called Clinical Guidance Systems (\CGSs{}), have
%demonstrated effectiveness in this
%
%
%
%Systems (\CGSs{}) that help Healthcare Professionals (\HCPs{}) adhere to
%Best Practice Guidelines (\BPGs{}). \BPGs{} are systematically developed,
%evidence-based statements for handling various patient conditions routinely published by
%medical associatoins \cite{
%Correct actions can be obtained from
%Best Practice Guidelines (\BPG{}). These are systematically developed statements
%for handling various patient conditions routinely published by medical
%associations. Strategies to reduce deaths from medical errors include:
%\begin{enumerate*}[label=(\roman*)]
%  \item making deviations from \BPGs{} more visible,
%  \item having remedies at hand to mitigate adverse events from deviations, and,
%  \item engineering barriers to prevent deviations from occuring \cite{MakaryBMJ16}.
%\end{enumerate*}
%These strategies are implemented by Clinical Decision Support Systems
%(\CDSSs{}): computer programs that assist healthcare providers
%conform to \BPGs{}. \CDSSs{} suggest appropriate actions as specified in the \BPG{}, and
%detecting and generating warnings for deviations.
%\CDSSs{} can also tackle adverse events from deviations from \BPGs{} by ensuring
%conformance to appropriate remedial \BPGs{}.
%The effectiveness of \CDSSs{} is supported favorably by existing
%research \cite{BenettJAMIA16,SahotaJIS11}.
%
%Developing a \CDSSs{} for a \BPG{} is a cross-disciplinary endeavor.
%Physicians communicate the \BPG{} to software developers, who
%translate it into executable code (\BPGLogic{}). This \BPGLogic{}
%is combined with other components, such as Graphical User Interface (GUI),
%to obtain a complete system. This results in a gap between the system
%specification, i.e. the \BPG{}, and its implementation, i.e. the \BPGLogic{}.
%Manual inspection of the code is used to gain confidence in the conformance of \BPGLogic{}
%to the underlying \BPG{}. Despite their critical nature, there exists no
%\emph{standardized} way of developing \CDSSs{}.

%\noindent Best Practice Guidelines (\BPGs{}) are evidence-based statements
%published by domain experts at medical institutions that codify recommended treatment
%in various clinical circumstances \cite{ClinicalGuidelinesURL}.
%Compliance to \BPGs, while desirable, is a significant challenge to
%achieve in practice. This is reflected in the fact that medical errors arising
%from unintended deviations from \BPGs{} (as opposed to diseases) are the third leading cause of mortality in the United States \cite{MakaryBMJ16}.
%These errors can be prevented through the use of Decision Support Systems that assist Healthcare Providers
%(\HCPs{}) in achieving compliance to \BPGs{}, called Clinical Guidance Systems
%(\CGSs{}). Such systems, according to the Centres for Disease Control and Prevention (CDC),
%have a strong evidence base supporting their effectiveness \cite{ClinicalDecisionSupportURL}.
%

%In practice, strict compliance
%to guidelines is difficult to achieve, and deviations may occur, especially in acute
%care. Such compliance-related \emph{preventable} medical errors are the third-leading cause of
%mortality in the United States, accounting for more than 250,000 deaths every year
%\cite{makary2016medical}. For example, the American Heart Association (AHA)
%
%\noindent Errors attribute to human and system factors (as
%opposed to diseases), called medical errors, are the
%third leading cause of mortality in the United States \cite{MakaryBMJ16}.
%Some of these errors can be avoided by ensuring compliance to Clinical Best
%Practice Guidelines (\BPGs{}). Produced by domain experts,
%\BPGs{} are systematically developed statements that codify
%recommended treatment in various clinical circumstances
%\cite{ClinicalGuidelinesURL}. To maximize their effectiveness,
%software systems that assist in compliance to \BPGs{}, called Clinical Guidance
%Systems (\CGSs{}) are employed. \CGSs{} have demonstrated


\noindent {\bf \large Intellectual Merit:}

\noindent Our proposed DSL needs to be \emph{physician-friendly}, \emph{executable}
and well supported by a suite of formal analysis tools such as a
\emph{model-checker} and \emph{program-verifier}. Satisfying such diverse
requirements is challenging. To ensure \emph{physician-friendliness}, we
support common characteristics of existing \BPGs{}, such as use of flowchart-like
notation, concurrent workflows, and tables in the DSL. This makes our DSL-based \BPGs{}
similar to their paper-based counterparts familiar to \emph{physicians}.
By formalizing the semantics of our DSL in a rewrite-based language framework
called $\K{}$, we obtain \emph{correct-by-construction} tools such as an
interpreter, program-verifier and model-checker, which can be used to both
analyze the \BPG{} for correctness, and execute it as \emph{correct-by-construction} \BPGLogic{} in a \CGS{}.

Our proposed architecture allows the \BPGLogic{} to be combined with different
hospital-specific components, like \GUIs. While such components may not be specified
in our DSL, they still affect safety. To ensure the entire system
(not just the \BPGLogic{}) is safe, we intend to specify \emph{intended behavior}
as \emph{correctness specification} in our DSL that are used to generate
runtime monitors that enforce \emph{conformance}. Under the assumption
execution satisfies these \emph{conformance monitors}, we intend to
demonstrate holistic \emph{safety}. Generating
such \emph{semantically-correct} monitors is an open problem intend to address in this
proposal.


%\noindent Our proposed approach presents several challenges.
%Our proposed DSL for expressing \BPGs{} needs to \emph{physician-friendly} to
%find adoption. To this end, we intend to find common characteristics of \BPGs{},
%and support them in our language. These include multiple concurrent workflows,
%interactions with heterogenous agents (such as sensor, healthcare records, and
%\HCPs{}), and timing-related constructs. At the same time, since the DSL is intended to be used in
%safety-critical settings, it must have formally defined semantics,
%and a suite of formal analysis tool such as deductive verifiers, model checkers, etc.
%While there are languages that excel at some of these characteristics, developing a language
%that supports all of the diverse set of requirements above presents
%a significant challenge.

%Our proposed plug-and-play architecture also presents unique challenges.
%While the decoupling allows reusing components while developing a \CGS{}, the safety
%of the entire system may be compromised if any of the individual components
%malfunction. To address this, we intend to use our DSL to specify expected
%behavior for components, and generate runtime monitors to enforce conformance.
%The generation of semantically correct monitors is a problem we intend to address in this proposal.
%
\noindent {\bf \large Broader Impacts: }

\noindent \CGSs{} have demonstrated effectiveness at reducing preventable
medical errors. Our work is aimed at making developing \emph{safe} \CGSs{}
\emph{easier}, \emph{cheaper}, and increase \HCP{} \emph{confidence} in them.
To this end, we aim to make our DSL the \emph{de-facto}
that medical associations can use to publish \BPGs{} that hospitals can employ
with our architecture to obtain \emph{safe} \CGSs{}. By making the
\BPG{} \emph{executable}, and \emph{re-usable} across \CGSs{},
cost required for developing an executable \BPGLogic{} for a \BPG, and ensuring
it mirrors the \BPG{} is lowered, hastening \CGSs{} adoption,
even in places with fewer resources, such as rural areas and developing
countries.


%Ensuring
%our language is comprehensible to \HCPs{} should also increase confidence
%in such systems, and hasten their adoption.
%A combination of lower development costs and increased
%confidence should increase prevalence of such system in healthcare, including in
%places with fewer resources such as rural areas, and developing countries.


%\noindent The immediate outcome of this project would be a novel way of
%developing \emph{correct-by-construction} \CGS{} that eliminates the gap between
%the \BPG{}, and its implementation in a \CGS{}. Apart from improving safety, our approach
%reduces the cost of developing a \CGS by reducing time required in validating
%the \CGS{}'s \BPGLogic{} mirrors the \BPG{}. This can lead to greater adoption, and subsequently reduced mortality. Beyond the immediate
%outcome, our DSL can serve as the \emph{de-facto standard} for
%expressing \BPGs{}, leading to a \emph{standardized}
%library of executable \BPGs{} that is maintained by medical
%associations and institutions. Our work precludes the need for hospitals to develop
%\CGSs{} from scratch, allowing establishments with fewer resources (for
%instance, in rural areas or the developing world) to adopt \CGSs{} in patient care.

