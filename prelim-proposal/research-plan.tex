\section{Research Plan}

In this section, we discuss challenges we plan to address through this work.

\subsection{Semantics Based Control Flow Graphs Generation}\label{subsec:visual-representation-generation}

Extremely vital to the success of \MediK{}, the the ability
to generate visual representation of \MediK{} programs to aid
comprehensibility to medical domain experts, enabling them to verify
accuracy of encoded medical knowledge. Even though \MediK{} code
structurally resembles paper-based guidelines, said
representations aid comprehensibility for medical
domain experts as the representations resemble their paper-based counterparts,
imparting a sense of familiarity.

Control Flow Graphs (CFGs) form the basis of not only such visual
representations, but other important analyses. For example,
flow-sensitive analyzers can be used to verify that a medical procedure
is performed on both treatment branches.
\CGSs{} form use in applications from compilers to model checkers.
But, despite their importance, \CGSs{} are not well understood.
To illustrate the lack of a \emph{canonical} definition of a \CGS{},
in \cite{KoppelICFP22}, the authors
used three different \CFG{} generators
on the same program, and obtaining three different \CFGs{}, each
tailored to the objectives of the generator.

Thus, generation of \CFGs{} that present \emph{all relevant} information,
with \emph{minimal} intervention during generation, in a
\emph{correct-by-construction} fashion, is indeed challenging.
In \cite{KoppelICFP22} that authors argue that operational semantics
are not suitable to be used directly for generation of \CFGs{} by demonstrating
that many control flow edges correspond to different halves of the same rule,
while some correspond to none. To address this, the authors utilize an algorithm
to construct an Abstract Machines (AM) style semantics from the SOS.

Consider for example, the following SOS rules for assignment:
$$
\infer[AssnCong]
{x := e, \mu\; \leadsto\; x := e^{\prime}, \mu^{\prime}}{e, \mu\; \leadsto\; e^{\prime}\mu^{\prime}}
\quad \quad
\infer[AssnEval]
{x := v,\mu\; \leadsto\; \left(\text{skip}, \mu\left[x \rightarrow v\right]\right)}{}
$$

The authors then automatically generate an AM, written
$\left\langle \left(t, \mu\right) \;\mid\; K\right\rangle$ where $K$ is the
\emph{context} or \emph{continuation} composed of frames of the form $[x\ :=\
(\Box_{t}, \Box_{\mu})]$ indicating holes for an evaluated value and
environment. Thus, the AM rules are:
$$
\left\langle \left( x := e, \mu \right)\;\mid\; k \right\rangle \rightarrow
\left\langle \left( e, \mu \right)\;\mid\; k \circ \left[x := \left(\Box_t,\Box_{\mu}\right) \right]\right\rangle
$$
$$
\left\langle \left( v, \mu \right)\;\mid\; k \right\rangle \rightarrow
\left\langle \left(\text{skip}, \mu\left[x \rightarrow v \right] \right) \;\mid\; k \right\rangle
$$

Next, as it's possible for the AM to have an infinitely-many states, an
abstraction is required to collapse the states and make the transition system
finite. To this end, the authors utilize a \emph{value irrelevant} abstraction
that replaces all values with a single one $\star$ that representing any of
them. With this, the abstract states can be treated as nodes in the CFG.
Thus, program execution with this abstraction will produce a CFG.
For example, consider evaluation of statement $x := y$, under an
infinite number of environments, can lead to an infinite number of
states of the AM.
But, converting all values to $\star$ results in
one-possible execution: $\left\langle \left( x := y, \left[ y \mapsto \star
    \right] \right) \;\mid\; k \right\rangle$
$\rightarrow  \left\langle \left(y, \left[ y \mapsto \star \right] \right) \;\mid\; k \circ \left[ x := \left(\Box_t,\Box_u\right) \right] \right\rangle$
$\rightarrow  \left\langle \left(\star, \left[ y \mapsto \star \right] \right) \;\mid\; k \circ \left[ x := \left(\Box_t,\Box_u\right) \right] \right\rangle$
$\rightarrow  \left\langle \left(\text{skip}, \left[x \mapsto \star, y \mapsto \star \right] \right) \;\mid\; k \right\rangle$

Note that expressions built over $\star$-values can still pose a
challenge. For instance, a statement of the form \inlineimp{while e do s} when
evaluted under an environment $\left[\star \mapsto \star\right]$ requires
handling $\left\langle \left(e, \left[ \star \mapsto \star \right] \right)\;\mid\; k
\right\rangle$, which can be matched by any expression rule. To handle this,
$\left\langle \left(e, \left[ \star \mapsto \star \right] \right)\;\mid\; k \right\rangle$
is over-approximated to $\left \langle \left(\star, \left[ \star \mapsto \star \right] \right)\;\mid\; k \right\rangle$
as evaluating any such $e$ would eventually result in $\star$.


%However, this still requires manual specification of abstract
%semantics to collapse the inifinite transition system into a finite one.
%To this end, the
%authors utilize a \emph{value irrelevance} abstration, where all values
%are replaced by a single abstract value representing all of them: \startext.
%This makes the system finite without direct changes to the semantics.
%This approach enables them to canonically define any \CFG{} as a projection
%of the transition system an abstracted abstract state machine, where their
%treatment of Abstract Machine matches the one in \cite{VanhornArxiv10}.

In this work, we plan to utilize ideas from \cite{KoppelICFP22}
in conjunction with the $\K{}$-Summarizer toolchain. The $\K{}$-
Summarizer is an effort to improve performance of semantics-based tools
by modifying the $\K$-definition itself. One of the primary objectives
of the summarizer is to enable semantics-based compilation.
Semantics-based compilation can be considered a generalization of
established techniques for established techniques for program optimization
such as partial evaluation of programs \cite{Jones93Book}.

For a given language $L$, let $\llbracket\_\rrbracket$ be the
intepreter for L. Say, for given $p \in L$--$\text{Program}$, and inputs
$\text{in}_1, \text{in}_2$,
$\text{output} = \llbracket p \rrbracket_{L}\left[\text{in}_1,\text{in}_2\right]$
if $p$ terminates.
We can now express partial evaluation of program p, using another
program $\mix$, called so since performs a mix of both execution and code
generation. Now, we get a two stage execution process:
$$ p_{\text{in}_1} = \llbracket \mix \rrbracket [p, \text{in}_1]$$
$$ \text{output} = \llbracket p_{\text{in}_1} \rrbracket \text{in}_2$$
%  $$\llbracket \llbracket \mix \rrbracket \left[p, \text{in}_1\right] \rrbracket
%  \left[\text{in}_2\right]
%  = \llbracket p \rrbracket \left[\text{in}_1, \text{in}_2\right]$$
Now, for given program $p$, if $\text{in}_1$ is statically known (at stage 1),
and $\text{in}_2$ is dynamically known (at stage 2), then,
$\llbracket p_{\text{in}_1} \rrbracket$ can be computed \emph{once}, and
re-used for any new $\text{in}_2$.
Semantics-based compilation can be seen as a generalization of this idea.
Say $L$ is a $\K$ definition. We can generate a new definition
$L^{p_{\text{in}_1}}$ that directly implements $\llbracket p_{\text{in}_1}
\rrbracket$. At runtime, only the dynamically determined input $\text{in}_2$
to complete execution.

Conceptually, the summarization process results in a \CFG{}, where
each basic block is summarized as a $\K$ rule. But, in \MediK{}'s
case, the inherent concurrency presents additional challenges
that we intend to tackle.

While \SBC{} is one of the primary goals of the $\K{}$-Summarizer
project, it has also been used to optimize $\K$ definitions
by \emph{merging} multiple rules that take smaller steps into
one rule that \emph{summarizes} all of them. This is similar
to the translation from \SOS{} to abstract machine presented in
\cite{KoppelICFP22}. In our work, we plan to generate such
\emph{summaries} soundly, and utilize them to generate visualizations
of \MediK{} programs. Note, that both $\K$ Summarizer and the work
in \cite{KoppelICFP22} have been tried in \emph{deterministic} settings.
We plan to address additional challenges from \emph{non-determinism} in our
work.

\subsection{Formal Analysis of Best Practice Guidelines}\label{subsec:formal-analysis}

\paragraph{Responsiveness Verification:}

In \cite{SaxenaFMCAD23}, we implemented a real-world \CDSS{} for management of
sepsis in pediatric cases, and verified a desired safety property holds.
Recall that a \MediK{} program consists of a concurrently executing instances
of \FSMs{}. During excution, an instance may be considered \emph{stuck} if
the event at the head of its input buffer does not have an associated handler,
rendering it \emph{unresponsive}. For this reason, languages such as P
\cite{DesaiPLDI13} raise an exception in such cases. One way to ensure
\emph{responsiveness} is to define event handlers for all events
in all states. But, for \MediK{} we found that:
\begin{enumerate*}[label=(\alph*)]
  \item it is tedious and error prone to define handlers for all events in all
  states, and,
  \item comprehensibility of programs is reduced as many handlers may never fire
  during execution.
\end{enumerate*}

Thus, we employed a weaker notion of responsiveness. We verified that all
possible events that a state can potentially receive have associated handlers.
it. This presents a challenge for reactive systems, or systems involve interactions
with the external world, such as \MediK{}-based \CDSSs, as exploring
the system's state space requires modeling the external components.
In \MediK{}, we address this by specifying external components
as \emph{ghost} machines - a technique also used by other state machine
formalisms such as P \cite{DesaiPLDI13}.
For program analysis, \emph{ghost} machines substitute external agents,
permitting exploration of the state space.
During execution, \emph{ghosts} are discarded and replaced by actual external agents.
Due to this, ghosts machines may have statements to express non-determinism
in processes. Consider, for instance, on a positive sepsis diagnosis, a
\HCP{} may chose to either administer fluids first, followed by antibiotics, or
vice-versa. \MediK{} supports such non-determinism using \inlinemedik{either-or}
statements as follows:
\begin{lstlisting}[style=mediksty
,language=medik
,basicstyle=\ttfamily\scriptsize
,numberstyle=\tiny
,framexleftmargin=1.5em
,xleftmargin=2em
]
either {
  broadcast StartFluidTherapy;
  broadcast StartAntibioticTherapy;
} or {
  broadcast StartAntibioticTherapy;
  broadcast StartFluidTherapy;
}
\end{lstlisting}

To do so, we added a rule that takes a machine instance in an active state
with an unhandled event at the head of the input buffer to a terminal \emph{stuck}
state. This permits us to $\K$'s search capability to check if the \emph{stuck}
state is reachable along any path.
While in the ideal case, we would have symbolically executed the program while
search for the \emph{stuck} pattern, we ran into performance issues
with $\K$'s haskell backend that at the time of writing was the only backend
that supported symbolic execution. Thus, as a fallback, we
use an abstraction to
encompass all possible values that may be encountered during execution.
For instance, when modeling entering a parameter such as the Heart Rate,
we use an abstract value to represent all possible concrete values.
This also means we needed to add the following rules to the semantics to
handle the abstraction.
\begin{lstlisting}[
  style=ksty
  ,language=k
  ,basicstyle=\ttfamily\scriptsize
  ,numbers=left
  ,numberstyle=\tiny
  ,framexleftmargin=1.5em
  ,showspaces=false
  ,xleftmargin=2em
  ]
  rule #nondet + _:Val    => #nondet
  rule _:Val   <= #nondet => #nondet
  rule #nondet && _       => #nondet
  rule if (#nondet) Block => Block
  rule if (#nondet) _     => .
\end{lstlisting}

This abstraction mechanism enables us to use $\K$'s LLVM backend
that only supports concrete execution to search for the \emph{stuck}
state. However, we recognize that the inability to perform symbolic
reasoning is a limitation. In this proposal, we plan to
address performance issues to enable symbolic execution.
We also plan to prove that our abstraction-based methodology
that allows us to use concrete execution is indeed sound, i.e.,
to show that if the abstract program satisfies the responsiveness
property, then the corresponding concrete program will also satisfy it.

\paragraph{Generation of Formal Proofs of Execution (optional)}

\subsection{Applicability to other \CDSSs{}}\label{subsec:applicability}

While \MediK{} has been used to implement one real-world, complex
\CDSSs{} for management of sepsis in pediatric cases, it remains
to be seen if it is a convenient medium for building \CDSSs{} in general.
To answer this, we plan to implement at least one more real-world \CDSS{}
using \MediK{}. To this end, we plan to express the American Heart Institute's
(AHA) \BPG{} for management of cardiac arrest \cite{AHAUrl}. Our choice
is motivated by the following reasons:
\begin{enumerate*}[label=(\roman*)]
  \item an effective \CDSS{} for cardiac arrest can significantly improve
    outcomes,
  \item enables us to evaluate the \MediK{} code against directly encoding
    the guideline in $\K$, as we had used the same \BPG{} for a $\K$-based \CDSSs{}
    (see section \ref{subsec:BPG-in-K}).
\end{enumerate*}


