\section{Research Plan}

In this section, we discuss challenges we plan to address through this work.

\paragraph{Semantics-based Compilation of \MediK{} programs}

Semantics-based compilation (\SBC{}) addresses many problems
in the context of \MediK{}. Some, such as \emph{performance improvements}
for execution and analysis, are not specific to \MediK{}, but vital
nevertheless. Others, such as \emph{enhancing comprehensibility} of programs
to non-experts in Computer Science by enabling visual program summaries that only
present \emph{relevant semantic information} is very benefecial in
\MediK-centric settings.

Semantics-based compilation can be considered a generalization of
established techniques for established techniques for program optimization
such as partial evaluation of programs \cite{Jones93Book}.

For a given language $L$, let $\llbracket\_\rrbracket_{L}$ be the
intepreter for L. Say, for given $p \in L$--$\text{Program}$, and inputs
$\text{in}_1, \text{in}_2$,
$\text{output} = \llbracket p \rrbracket_{L}\left[\text{in}_1,\text{in}_2\right]$
if $p$ terminates.
We can now express partial evaluation of program p, using another
program $\mix$, called so since performs a mix of both execution and code
generation as follows:
  $$\llbracket \llbracket \mix \rrbracket \left[p, \text{in}_1\right] \rrbracket
  \left[\text{in}_2\right]
  = \llbracket p \rrbracket \left[\text{in}_1, \text{in}_2\right]$$
Now, for given program $p$, if $\text{in}_1$ is statically known (at stage 1),
and $\text{in}_2$ is dynamically known (at stage 2), then, executing the
entire program in one stage as:
$$ \llbracket p \rrbracket \left[\text{in}_1, \text{in}_2\right] $$ can be more
expensive than computing partially evaluating $p$ on $\text{in}_1$,
i.e. $p_{\text{in}_1} = \llbracket p \rrbracket \left [ \text{in}_1 \right]$ and
re-utilizing it for phase 2, i.e.
$\llbracket p \rrbracket \left [\text{in}_1, \text{in}_2 \right] = \llbracket
p_{\text{in}_1} \rrbracket \left [ \text{in}_2 \right]$


%Our experience of implementing pediatric sepsis management in \MediK{}
%revealed the importance of \emph{visual representations} to improve
%comprehensibility of \MediK{} programs to medical domain experts.
%\MediK{} programs structurally resemble flowhcharts by design, but
%if visualized, they resemble existing \BPGs{} found in medical
%literature that \HCPs{} are already familiar with.
%
%We plan to utilize a technique called semantics-based compilation (SBC) to generate
%visual representations.

\subsection{Formal Analysis of Best Practice Guidelines}

\paragraph{Responsiveness Verification}

\paragraph{Liveness Verification}


\paragraph{Generation of Formal Proofs of Execution (optional)}
