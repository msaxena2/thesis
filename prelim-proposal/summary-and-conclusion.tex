\section{Summary and Conclusion}\label{sec:summary-and-conclusion}

Preventable Medical Errors remain a challenge in medicine, but
can be mitigated through the use of Guidelines-based Clinical Decision Support
Systems (\CDSSs{}). Such systems codify Best Practice Guidelines (\BPGs{}) and
support Healthcare Providers (\HCPs) with situation-specific advice.
While such systems have shown to improve clinical outcomes, developing
them is challenging, which limits their adoption.

Our proposal intends to address these challenges using a \emph{semantics-first}
approach. At the core of our approach is a new language for expressing
\BPGs{} in an executable manner called \MediK{}. \MediK{} has a formal semantics
defined in the $\K$-framework, and thus has a \emph{correct-by-construction}
interpreter and analysis tools.

In section \ref{sec:introduction}, we identified research questions (RQs)
that we plan to address through this work. We now summarize how our
current progress and proposed work addresses them.

\paragraph{(RQ1) Language Design:} How do we design a language to accomodate
expressing complex \BPGs{} that is also comprehensible to Healthcare Providers
(\HCPs{})?

To answer this, we identified common characteristics of \BPGs{} that are
desirable of a language for expressing them in an executable manner.
In \cite{SaxenaFMCAD23}, we explain how our language is built around
these characteristics. Our work build on the state-of-art in modeling
large concurrent systems, but adapts it to specific needs of \BPGs{}.

\paragraph{(RQ2) Applications:} Can we build real-world \CDSSs{} using our
approach?

To answer this, we collaborated with the OSF Healthcare Children's Hospital of Illinois
in Peoria to build a \CDSS{} for management of sepsis in pediatric
cases based on their \BPG{}. Our tool has been approved by the hospital
Institutional Review Board, and is slated to under clinical simulations.
Thus, we argue our approach does indeed enable building real-world systems.
To further answer this question, in section \ref{subsec:applicability} we
identify another \BPG{} that we intend to encode in \MediK{}, and demonstrate
that our approach works for another \BPG{}.

\paragraph{(RQ3) Ecosystem:} Do the semantics-derived tools work well for
our purposes? Can we use these tools for execution and formal analysis?

The interpreter generated from the semantics is able to satisfy performance
needs of the sepsis management system. This \CDSS{} is based on a complex
\BPG{}, involving multiple workflows that are encoded as \MediK{} machines.
However, as discussed in section \ref{subsec:formal-analysis},
we ran into performance issues while utilizing $\K$'s symbolic
execution capabilities due to the size of the program. To get around this,
we utilized an abstraction to check that the sepsis management \CDSS{} satisfies
desired responsiveness properties. We propose to prove the soundness
of our abstraction-based technique and investigate using symbolic execution
to further make analyzing programs easier.

In section \ref{subsec:visual-representation-generation}, we propose to
implement a semantics-derived solution to generate visual representations
of \MediK{} programs. Such representations are vital to ensure that
\HCPs{} can comprehend \MediK{} code to validate accuracy of medical knowledge.

Thus, with these questions answered, we demonstrate an advancement over
the existing state-of-art in building \CDSSs{} discussed in section \ref{subsec:related-work}.

