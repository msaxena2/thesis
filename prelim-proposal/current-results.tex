\section{Current Results}
In this section, we describe progress made towards goals listed
in section \ref{sec:introduction}. In section \ref{subsec:evaluating-k},
we describe our  experience with the semantics-first approach in $\K$ that provided
confidence in using $\K$ as the basis of our solution. We then briefly
describe our experience of using a $\K$ definition directly to encode
medical guidelines in section \ref{subsec:BPG-in-K}, followed by
a discussion on the \MediK{} DSL in section \ref{subsec:medik}


\subsection{Evaluating $\K$ for our approach}\label{subsec:evaluating-k}

To derive maximum benefit from the $\K$ semantics, it is important that:
\begin{enumerate*}[label=(\roman*)]
  \item the $\K$ semantics is descriptive enough to implement language-specific
    tools such as interpreters, model checkers, etc, and,
  \item $\K$-derived tools compare favorably against their language-specific
    counterparts.
\end{enumerate*}
In \MediK{}'s case, we needed to ensure that the interpreter,
in particular, is appropriate for
use in a \CDSS{}. The concern for performance stems from the fact
that being language-agnostic, $\K{}$ derived tools
cannot accommodate language-specific optimizations.

To evaluate the feasibility of $\K$ semantics and derived tools, we
drew upon our work of defining the semantics of a real-world language
(unrelated to \MediK{}), where the $\K$ semantics and derived tools offer
are viable alternatives to their language-specific counterparts.

In \cite{HildenbrandtCSF18},
we completely defined the semantics of the Ethereum Virtual Machine (EVM)
in $\K{}$. The EVM is a low-level stack-based language
that forms backbone of the Ethereum Blockchain. Unlike \MediK{},
EVM had an existing informal paper-based semantics
called the Yellow Paper \cite{evmYellowpaperUrl},
and a reference implementation in C++ \cite{cppEthereumUrl}.
This allowed us to compare the $\K$ based semantics
and interpreter against mature language-specific ones.

We based our $\K$ semantics on the Yellow Paper, and
found several inconsistencies that were confirmed by its developers \cite{HildenbrandtCSF18}.
This stems from the fact that unlike the $\K$ semantics,
the Yellow Paper is an informal document in natural language.
Often, the Yellow Paper was found to be unclear, underspecified
or inconsistent with the behavior of actual implementations.
As the $\K$ semantics are executable, all details have to
considered for our interpreter to pass all tests in the reference
test-suite. We utilized the developer documentation for our semantics
to generate an alternative to the Yellow Paper itself, called the Jello Paper
\cite{evmJellopaperUrl}. The Jello Paper has been well-received by the
community, with discussions to designate it as the \emph{canonical document}
instead of the Yellow Paper. For more details, we refer the reader to
section VII of \cite{HildenbrandtCSF18}.

\begin{wrapfigure}{l}{0.6\textwidth}
      \begin{tabular}{ l l l l }
          \textbf{Test Set} (no. tests) & \textbf{Lem EVM} & \textbf{KEVM} & \textbf{cpp-ethereum} \\
          Lem (40665)                   & 288:39           & 34:23         & 3:06                  \\
          VMStress (18)                 & -                & 72:31         & 2:25                  \\
          VMNormal (40665)              & -                & 27:10         & 2:17                  \\
          VMAll (40683)                 & -                & 99:41         & 4:42                  \\
          GSNormal(22705)               & -                & 35:00         & 1:30                  \\
          GSQuad (250)                  & -                & 855:24        & 0:21                  \\
          GSAll (22955)                 & -                & 889:00        & 1:51                  \\
      \end{tabular}
  \caption{Lem EVM vs KEVM vs cpp-ethereum}\label{fig:evm-comparison}
\end{wrapfigure}
\figurename{} \ref{fig:evm-comparison} compares the $\K{}$-generated interpreter
against the reference implementation (\textbf{cpp-ethereum}) and
another formalization (\textbf{Lem EVM}) of the EVM semantics \cite{HiraiWSTC17} in Lem \cite{MulliganACM14}
on the official test that implementations are expected to pass.
All execution times are given as the full sequential CPU time (in MM:SS format) on an Intel i5-3330 processor (3GHz on 4 hardware threads) and 24 GB of RAM.
By comparing to the C++ reference implementation, we show the feasibility of using the KEVM formal semantics for prototyping, development, and test-suite evaluation.

The row Lem indicates a run of all the tests that the Lem semantics can run (a subset of the VMTests).
The row VMStress indicates a run of all 18 stress tests in the test-suite, to compare the performance of KEVM with the C++ implementation.
The row VMNormal is a run of all the non-stress tests in the test-suite (\textit{not} the same set of tests as Lem).
VMAll is the addition of the second and third rows and is included for completeness.
The last three rows indicate a runs of the GeneralStateTests; GSNormal are the non-stress tests, GSQuad are the stress tests, and GSAll is the addition of the two.
Under the GeneralStateTests, our tools performs well except in the case of QuadraticStateTests (250 out of 22955).

As shown in the comparison, not only did the $\K$ generated interpreter significantly
outperformed the Lem-based formal executable EVM semantics, it performs
favorably when compared to the language-specific reference implementation.
It was under 30 times slower on the stress tests, roughly 20 times
slower on all tests, and only 11 times slower on VMNormal tests.
Note that since the publication of the paper, newer versions
of $\K$ have been introduced with significant performance improvements.

In \cite{HildenbrandtCSF18,ParkFSE18}, we used the $\K$ semantics of the EVM to
build a tool for verification EVM smart contracts. As EVM bytecode is a
low-level stack based language lacking constructs such as functions,
verification of smart contracts proved tedious. To address this, we
introduced a DSL for simulating function calls modeled on the Ethereum ABI
\cite{ethereumAbiUrl} and several EVM-specific abstractions to make verification
scalable. As a result, we could verify several real world smart contracts
satisfied their correctness specifications. Thus, our work allowed to us
to conclude that
\begin{enumerate*}[label=(\roman*)]
  \item the performance of a $\K$ generated interpreter compares favorably to its
  language-specific counterpart,
  \item the $\K$ can be used to implement a language from scratch, i.e. without
  an informal paper based semantics, as in the case of our DSL for calling
  functions at the bytecode level, and,
  \item the verifier can be used on large programs, but may require additional
  work, such as simplification abstractions and lemmas.
\end{enumerate*}

\subsection{Executable Best-Practice Guidelines as $\K$ Definitions}\label{subsec:BPG-in-K}

In \cite{Saxena22TR}, we presented a \CDSS{} for management of
cardiac arrest based on a well known \BPG{} published by the American Heart
Association \cite{AHAUrl}. Recall from section \ref{subsec:cdss} that
\BPGs{} are often expressed using flowchart-like notation. The cardiac arrest
management \BPG{} also follows the same convention. In our work, we encoded
transitions in \BPG{} flowcharts as $\K$-rules. As $\K$'s
configuration abstraction mechanism enables only relevant parts of the
configuration to be mentioned, we argue that our methodology permits
\emph{succinct} representation of medical knowledge. This is supported by
the fact that our $\K$ definition encoding the \BPG{} was only a few hundred lines of
code, and could be used to implement an entire \CDSS{}. But, the $\K$ based
\BPG{} proved hard to comprehend, both to computer scientists not familiar with
$\K$, and healthcare professionals. Although this work \emph{failed} to address
the objectives from section \ref{sec:introduction}, it emphasized the importance
of comprehensibility to \HCPs{} for future endeavors.

\subsection{\MediK{}}\label{subsec:medik}

In \cite{SaxenaFMCAD23} we presented the \MediK{} DSL for expressing
\BPGs{} that are both executable and comprehensible to \HCPs{}. This
enables \HCPs{} to validate the correctness of medical knowledge encoded
in \MediK{} code. Recall from section \ref{subsec:cdss} the following
characteristics of \BPGs{}:
\begin{enumerate*}[label=(\alph*)]
  \item involve \emph{concurrent} workflows with inter-workflow dependencies,
  \item are often specified using a \emph{flowchart-like} notation, and,
  \item require communication between heterogeneous \emph{external} agents such
    as sensors, monitors and Electronic Health Records.
\end{enumerate*}

\begin{wrapfigure}{l}{0.45\textwidth}
    \caption{\MediK{} \FSM{} Skeleton}\label{fig:medik-skeleton}
\begin{lstlisting}[style=mediksty
,language=medik
,basicstyle=\ttfamily\notsotiny
,numbers=left
,numberstyle=\tiny
,framexleftmargin=1.5em
,xleftmargin=2em
]
[init] machine <IDENTIFIER>
  receives <IDENTIFIER_LIST> {
  vars <IDENTIFIER_LIST>;

  [init] state <IDENTIFIER> {
    entry [(IDENTIFIER_LIST)] {
      <STMT> // entry block
    }
    on <IDENTIFIER> [(IDENTIFIER_LIST)] do {
      <STMT> // event handler
    }
  }
}
\end{lstlisting}
\end{wrapfigure}

\MediK{} has been designed to account for aforementioned characteristics.
In \MediK{}, like in P, programs are expressed as concurrently
executing instances of state machines that communicate via passing messages.
Given a \BPG{} where each workflow is expressed as a flowchart,
we express said flowcharts as State Machines in \MediK{}. Each flowchart node
in the \BPG{} is represented as a state in a state machine, and
edges are represented as state transitions. During execution,
instances of these machines are created, which interact with each other by
passing events. Note the distinction between
machine and its instance. A machine
is analogous to an Object Oriented Programming (OOP) class, whereas
its instance is analogous to an OOP object.

As \FSMs{} are central to our approach,  we provide a skeleton
of a \MediK{} machine in \figurename{} \ref{fig:medik-skeleton} before explaining how \BPGs{} are
encoded in \MediK{} code.
Note that we use \inlinemedik{[...]} to denote
optional constructs, \inlinemedik{<...>} for mandatory constructs, lowercase for
terminals, and uppercase for non-terminals.

A \MediK{} program consists of a set of machine definitions.
A machine definition starts with the keyword \inlinemedik{machine},
followed by its name (line 1). On line 2, following the
\inlinemedik{receives} keyword, is a comma-separated list of identifiers
signifying the events that the machine can receive from other machines.
One machine in a program can be
prefixed with the \inlinemedik{init} keyword. This machine is referred to as the
initial machine.
On line 3, following the keyword
\inlinemedik{vars}, another comma-separated list of identifiers signifies
the instance-variables. During execution, each instance maintains a mapping from
these variables to values. Each machine defines a set of states, such as
the one in lines 5-11. A state
 has a name, an optional entry block (lines 6-8),
and a set of event handlers (lines 9-11). The entry block
begins with the keyword \inlinemedik{entry}, and may contain a list of variables
that are bound to values when the state is entered during execution.
One state in the machine may be prefixed with \inlinemedik{init}, specifying the
initial state. When execution begins, an implicit instance
of the initial machine is created, and the \inlinemedik{entry} block of
its initial state is executed. When an instance of a machine is dynamically
created during runtime, the \inlinemedik{entry} block of its initial state is executed.
Event handlers within a state begin with \inlinemedik{on} followed by the event name
and an optional list of variables. When the event handler is executed, data from
the received event's payload is bound to aforementioned variables which
can be used in the code block that follows the \inlinemedik{do} keyword.

To handle interaction with heterogeneous external sources, \MediK{}
models them as interfaces. An interface is a \FSM{} that has its
transition system defined externally. For example, certain measurements
such as the heart rate are often obtained from sensors.
The code in \figurename{} \ref{fig:interface} shows the process
of obtaining external measurements in \MediK{}.

\begin{figure}[b]
  \begin{subfigure}[b]{0.5\textwidth}
\begin{lstlisting}[
  style=mediksty
  ,language=medik
  ,basicstyle=\ttfamily\tiny
  ,numbers=left
  ,numberstyle=\tiny
  ,framexleftmargin=1.5em
  ,xleftmargin=2em
  ]
interface HeartRateSensor { }

machine TreatmentMachine { ...
  var hrSensor = createFromInterface(HeartRateSensor,
                              "heartRateSensor");
  var heartRate = obtainFrom(hrSensor, "heartRate");
}
\end{lstlisting}\caption{External Sources}\label{fig:interface}
  \end{subfigure}
  \begin{subfigure}[b]{0.47\textwidth}
\begin{lstlisting}[
  style=mediksty
  ,language=medik
  ,basicstyle=\ttfamily\tiny
  ,numbers=left
  ,numberstyle=\tiny
  ,framexleftmargin=1.5em
  ,xleftmargin=2em
  ]
fun isHeartRateNormal() {
  days(age) in {
    interval(days(0)  , months(1)): return hr > 205;
    interval(months(1), months(3)): return hr > 205;
    // omitting other cases
    default                       : return hr > 100;
  }
}
\end{lstlisting}
  \caption{Checking abnormality using tables}\label{fig:hr-check-fun}
  \end{subfigure}
  \caption{Executable \BPGs{} in \MediK{}}
\end{figure}

Since we don't have the transition system for
the heart rate sensor, we declare it as an interface (line 1).
Next, instead of using \inlinemedik{new} to create an instance, we use
a builtin \MediK{} construct \inlinemedik{createFromInterface}, which takes as
arguments
\begin{enumerate*}[label=(\alph*)]
  \item the inteface name (lines 4),
  \item a unique identifier string used to identify the
    instance outside the \MediK{} process.
\end{enumerate*}
All other \MediK{} machines can interact with external sensor using
variable \inlinemedik{hrSensor}. There is no need to make any distinction
between external, and \MediK{}-based machines.
To deal with external interactions, input and output
pipes are provided to the \MediK{} process at launch.
When the \inlinemedik{send} construct is used on an external machine,
\MediK{} will write
a JSON \cite{jsonUrl} message with the event data, the
identifier from line 5, and a unique transaction id
to the \emph{write-end} of the output pipe. At the
\emph{read-end}, we need to write external code (in any programming language)
to handle the JSON message. In the example above, this involves reading from
the external heart rate sensor. To send data to \MediK{}, a JSON message in
a pre-specified format needs to be written to the \emph{write-end} of the
input pipe.

%\begin{figure}[H]
%  \begin{subfigure}[b]{0.43\textwidth}
%    \begin{lstlisting}[style=mediksty
%    ,language=medik
%    ,basicstyle=\ttfamily\tiny
%    ,numbers=left
%    ,numberstyle=\tiny
%    ]
%    [init] machine <IDENTIFIER>
%      receives <IDENTIFIER_LIST> {
%      vars <IDENTIFIER_LIST>;
%
%      [init] state <IDENTIFIER> {
%        entry [(IDENTIFIER_LIST)] {
%          <STMT> // entry block
%        }
%        on <IDENTIFIER> [(IDENTIFIER_LIST)] do {
%          <STMT> // event handler
%        }
%      }
%    }
%    \end{lstlisting}
% \end{subfigure}
%  \begin{subfigure}[b]{0.55\textwidth}
%    \begin{lstlisting}[
%      style=ksty
%      ,language=k
%      ,basicstyle=\ttfamily\tiny
%      ,numbers=left
%      ,numberstyle=\tiny
%      ,showspaces=false
%      ]
%    syntax Exp ::= Id | Val | "this"
%      | Exp "." Exp
%      | "obtainFrom" "(" Exp "," Exp ")"
%      | "interval" "(" Exp "," Exp ")"
%      | Exp "in" Exp
%
%    syntax Stmt ::= Exp "=" Exp ";"
%      | "if" "(" Exp ")" Block "else" Block
%      | "new" Id "(" Exps ")" ";"
%      | "createFromInterface" "(" Id "," String ")" ";"
%      | "sleep" "(" Exp ")" ";"
%      | "send" Exp "," Id "," "(" Exps ")" ";"
%      | "broadcast" Id "," "(" Exps ")" ";"
%      | "goto" Id "(" Exps ")" ";"
%      | Exp "in" "{" CaseDecl "}"
%    \end{lstlisting}
%  \end{subfigure}
%\end{figure}


\begin{figure}[H]
  \begin{lstlisting}[style=mediksty, language=medik,  multicols=2,
  basicstyle=\ttfamily\tiny, numbers=left
  ,showspaces=false
  ,xleftmargin=2em]
machine SepsisScreening receives .. {
  init state Start {
    on StartScreening do {
      goto ObtainAge;
    }
  }
  state ObtainAge {
    entry {
      send tablet, Instruct, ("get age");
    } on ConfirmAgeEntered do {
      goto ObtainWeight;
    }
  }
  state ObtainWeight { ... }
  state ObtainHighRiskConditions { ... }
  state CalculateScore {
      var hrAbnormal = !isInNormalRange("HR", ...);
      var bucket1    = hrAbnormal ||  ...
      var bucket3    = mentalStatusAbnormal || ...

      var sepsisSuspected
        = bucket1 && bucket2 && bucket3;

      send tablet, SepsisDiagnosis
        , (sepsisSuspected);
  }
}
\end{lstlisting}
  \caption{Sepsis Screening in \MediK{}}\label{fig:medik-sepsis-screening}
\end{figure}

Recall from \ref{subsec:cdss} the guideline for sepsis screening shown in
\figurename{} \ref{fig:sepsis-screening}.
In \figurename{} \ref{fig:medik-sepsis-screening}, we show
\MediK{} code corresponding to the sepsis screening guideline.
When modeled in \MediK{}, a flowchart in the guideline is represented using
a \MediK{} machine. \emph{Nodes} in the flowchart are represented as
\emph{states} in a \MediK{} machine, while flowchart \emph{edges} as \emph{state-transitions}.
Note that we use \emph{node} to refer to constructs
in the flowchart, and \emph{state} to refer to counterparts in \MediK{}.
Also, while it's desirable to represent each flowchart \emph{node} as a
state machine \emph{state}, the task in the flowchart \emph{node} may warrant using
multiple state-machine \emph{states}. For example,
in \figurename{} \ref{fig:sepsis-screening}, the step \say{Obtain Patient Age,
Weight, and High Risk Conditions} is translated to states \inlinemedik{ObtainAge} (lines
7-13), \inlinemedik{ObtainWeight} (line 14), and
\inlinemedik{ObtainHighRiskConditions} (line 15) in \figurename{} \ref{fig:medik-sepsis-screening}.
Within each of these states, the code
permits communication with heterogeneous external agents for obtaining
required parameters. For instance, on line 9, an
\inlinemedik{Instruct} event is sent to an external \inlinemedik{tablet} machine
with the payload \inlinemedik{"get age"}. The recipient process
runs on a tablet held by the Healthcare Provider, and handles
the event by prompting the provider to enter the patient's age.
A \inlinemedik{ConfirmAgeEntered} event, emitted once the age
is obtained, enables the screening machine to proceed to the next
step (lines 11-13). Once all appropriate measurements have been obtained,
they are checked for abnormality (lines 18-26) using tables shown in \figurename{}
\ref{fig:hr-check-fun} to arrive upon a diagnosis.

We draw attention to the fact that the program in \MediK{} structurally
resembles the paper-based \BPG{} from \figurename{} \ref{fig:sepsis-screening}.
As demonstrated in \cite{SaxenaFMCAD23}, our \MediK{}-based approach allowed
us to conveniently model all screening and treatment workflows from section
\ref{subsec:cdss} to build a functioning \CDSS{} that is being considered for
clinical simulations.
