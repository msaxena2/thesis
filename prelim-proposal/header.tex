\documentclass{nsf_proposal}
%%%%%%%%%%%%%%%%%%%%%%%%%%%%%%%%%%%%%%%%%%%%%%%%%%%%%%%%%%%%%%%%%%%%%%%%
% LaTeX template to generate an NSF proposal
%
% First version by: Stefan Llewellyn Smith, Sarah Gille, others.
%
% Additions by: Ronni Grapenthin, New Mexico Tech.
%
% This version maintained/modified by: Jeremy A. Gibbs, Univ. of Utah
%
% This template is free source code. It comes without any warranty, to
% the extent permitted by applicable law. You can redistribute it and/or
% modify it under the terms of the Do What The Fuck You Want To Public
% License, Version 2, as published by Sam Hocevar. See
% http://www.wtfpl.net for more details.
%%%%%%%%%%%%%%%%%%%%%%%%%%%%%%%%%%%%%%%%%%%%%%%%%%%%%%%%%%%%%%%%%%%%%%%%
\usepackage[T1]{fontenc}
\usepackage[latin1,utf8]{inputenc}
\usepackage[amssymb]{SIunits}
\usepackage{latexsym}
\usepackage{proof}
\usepackage{stmaryrd}
\usepackage{amssymb}
\usepackage{multirow}
\usepackage{proof}
\usepackage{mathtools}
\usepackage{wrapfig}
\usepackage{amsmath, amsthm, amssymb}
\usepackage{amsfonts}
\usepackage[format=plain,indention=0cm, font=small, labelfont=bf]{caption}
\usepackage{subcaption}
\usepackage{longtable}
\usepackage{fancyhdr}
\usepackage[pdftex]{graphicx}
\usepackage[pdftex,
	    colorlinks,
	    pdfstartview=FitH,
	    linkcolor=black,
	    citecolor=black,
	    urlcolor=black,
	    filecolor=black
	    ]{hyperref}
\usepackage{lscape}
\usepackage{floatrow}
\usepackage{enumerate}
%\usepackage{enumitem}
\usepackage{tabularx}
\usepackage{ragged2e}
\usepackage{xcolor}
\usepackage[inline]{enumitem}
\usepackage{wrapfig}
\usepackage{pifont}
\usepackage{bbding}
\usepackage{cancel}
\usepackage{dirtytalk}
\usepackage{listings}
\usepackage{multicol}

\newcommand{\frontend}{\emph{frontend}}
\newcommand{\BPG}{BPG}
\newcommand{\BPGs}{BPGs}
\newcommand{\CGS}{CGS}
\newcommand{\CGSs}{CGSs}
\newcommand{\HCP}{HCP}
\newcommand{\HCPs}{HCPs}
\newcommand{\ED}{ED}
\newcommand{\EDs}{EDs}
\newcommand{\CDSS}{CDSS}
\newcommand{\CDSSs}{CDSSs}
\newcommand{\BPGLogic}{knowledge-base}
\newcommand{\K}{\mathbb{K}}
\newcommand{\MediK}{$\text{Medi}\K{}$}
\newcommand{\FSM}{\text{FSM}}
\newcommand{\FSMs}{\text{FSMs}}
\newcommand{\Var}{\text{Var}}
\newcommand{\LHS}{\emph{\text{LHS}}}
\newcommand{\RHS}{\emph{\text{RHS}}}
\renewcommand{\phi}{\varphi}
\newcommand{\GUI}{GUI}
\newcommand{\GUIs}{GUIs}
\newcommand{\PME}{PME}
\newcommand{\PMEs}{PMEs}
\newcommand{\CIG}{CIG}
\newcommand{\CIGs}{CIGs}
\newcommand{\EHRs}{EHRs}
\newcommand{\ACLS}{ACLS}
\newcommand{\CPR}{CPR}
\newcommand{\CISs}{CISs}
\newcommand{\RTSs}{RTSs}
\newcommand{\ASMs}{ASMs}
\newcommand{\DSL}{\text{DSL}}
\newcommand{\DSLs}{\text{DSLs}}
\newcommand{\SBC}{\text{SBC}}
\newcommand{\SBCs}{\text{SBCs}}

% Convenience Commands
\newcommand{\cmark}{\text{\ding{51}}}
\newcommand{\xmark}{\text{\ding{55}}}
\newcommand{\greencheck}{{\color{green}\cmark}}
\newcommand{\redcross}{{\color{red}\xmark}}
\newcommand{\cancelcheck}{\bcancel{\cmark}}
\newcommand{\stress}[1]{\underline{\emph{#1}}}

% Scheduling Commands
\newcommand{\Machine}{\mathcal{M}}
\newcommand{\Instance}{\mathcal{I}}
\newcommand{\scheduled}{\textit{scheduled}}
\newcommand{\enabled}{\textit{enabled}}
\newcommand{\epoch}{\textit{epoch}}

% SBC Commands
\newcommand{\mix}{\text{mix}}
\newcommand{\CFG}{\text{CFG}}
\newcommand{\CFGs}{\text{CFGs}}
\newcommand{\SOS}{\text{SOS}}


\definecolor{greybackground}{rgb}{0.95,0.95,0.92}
\lstdefinestyle{ksty}{
  keywordstyle=\color{magenta},
  basicstyle=\ttfamily\small,
  commentstyle=\color{codepurple},
  %backgroundcolor=\color{greybackground},
  framerule=0pt
}
\lstdefinelanguage{k}{
  morekeywords={rule,configuration,=>,syntax,multiplicity,type,module,endmodule,import,imports, left,strict,seqstrict,bracket,structural,requires},
  morecomment=[l]{//},
  morecomment=[s]{/*}{*/}
}

\definecolor{codegreen}{rgb}{0,0.6,0}
\definecolor{codegray}{rgb}{0.5,0.5,0.5}
\definecolor{codepurple}{rgb}{0.58,0,0.82}
\lstdefinestyle{mediksty}{
  keywordstyle=\color{magenta},
  commentstyle=\color{codepurple}
}
\lstdefinelanguage{medik}{
  morekeywords={either, or, machine, interface, vars, state, entry, on, do, goto, receives, ~>, =>},
  morecomment=[l]{//},
  morecomment=[s]{/*}{*/}
}

\lstdefinestyle{impsty}{
  keywordstyle=\color{magenta},
  basicstyle=\ttfamily\footnotesize
}
\lstdefinelanguage{imp}{
  morekeywords={if, while, var}
  morecomment=[l]{//},
  morecomment=[s]{/*}{*/}
}


\newcolumntype{Y}{ >{\RaggedRight\arraybackslash}X}
\newcommand\T{\rule{0pt}{2.6ex}}
\newcommand\B{\rule[-1.2ex]{0pt}{0pt}}

\renewcommand{\refname}{References Cited}

\newcommand{\degrees}{$\!\!$\char23$\!$}

\def\rrr#1\\{\par
\medskip\hbox{\vbox{\parindent=2em\hsize=6.12in
\hangindent=4em\hangafter=1#1}}}

\renewcommand\baselinestretch{1}
\pagestyle{empty}

\graphicspath{ {./figures/} }
\setlength{\parindent}{0pt}
\setlength{\parskip}{1ex}
\renewcommand{\figurename}{Fig. }
\let\origthelstnumber\thelstnumber
\makeatletter
\newcommand*\Suppressnumber{%
  \lst@AddToHook{OnNewLine}{%
    \let\thelstnumber\relax%
     \advance\c@lstnumber-\@ne\relax%
    }%
}

\newcommand*\Reactivatenumber{%
  \lst@AddToHook{OnNewLine}{%
   \let\thelstnumber\origthelstnumber%
   \advance\c@lstnumber\@ne\relax}%
}

\newcommand{\inlinemedik}[1]{\lstinline[style=mediksty,basicstyle=\ttfamily\footnotesize]{#1}}
\newcommand{\inlinek}[1]{\lstinline[style=ksty,basicstyle=\ttfamily\footnotesize]{#1}}
\newcommand{\inlineimp}[1]{\lstinline[style=impsty,basicstyle=\ttfamily\footnotesize]{#1}}
\newcommand\notsotiny{\@setfontsize\notsotiny\@viipt\@viiipt}
\newcommand{\bigstartext}{$\bigstar$}
\newcommand{\startext}{$\star$}
\lstset{ literate={~}{{\raisebox{0.5ex}{\texttildelow}}}{1} }

\makeatother
\let\oldBox\Box
\renewcommand{\Box}[1][0pt]{%
  \mathrel{\raisebox{#1}{$\oldBox$}}%
}
\newtheorem{definition}{Definition}
