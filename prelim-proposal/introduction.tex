\section{Introduction}

Preventable Medical Errors (\PMEs{}) characterized by
incorrect intended treatment, or incorrect executions of intended
treatment present a significant challenge in Healthcare
\cite{RodziewiczStatsPearls18}. According to a seminal report on the subject
\cite{DonaldsonBook00}, in 1997,
between 44,000 and 98,000 deaths were estimated to have been caused by \PMEs{} in
the United States alone. A more recent study analyzed data from the eight-year
period between 2000 and 2008, and estimated that in 2013, the number of deaths
caused by \PMEs{} was more than 250,000, making \PMEs{} the third-leading
cause of death in the United States \cite{MakaryBMJ16}.
The adverse effects of \PMEs{} extend beyond patient outcomes.
One study estimated the financial burden of \PMEs{} to the United States to be
19.5 billion dollars in 2008 \cite{AndelJHCF12}. According to the authors of
\cite{RodziewiczStatsPearls18}, \PMEs{} caused psychological effects such
as anger and guilt in healthcare providers (\HCPs{}), adversely impacting their mental
health.

One strategy to mitigate \PMEs{} is to utilize evidence-based statements
published by hospital and medical associations that codify recommended
interventions for various clinical scenarios called Best Practice Guidelines (\BPGs{})
\cite{field1990clinical}. High quality guidelines are routinely updated to account for
 results from clinical trials and advances in medicine, and make the latest
 diagnosis and treatment information accessible to providers \cite{SteinbergNAP11}.

While \BPGs{} have the potential to reduce medical errors, their effectiveness hinges
on the adherence of healthcare providers to them.
For example, consider Advanced Cardiac Life Support (\ACLS{}): a \BPG{} published
by the American Heart Association (AHA) for management
of a life threatening condition called in-hospital cardiac arrest (IHCA) \cite{AHAGuidelineAdult, AHAGuidelinePed}. Studies suggest that management
of IHCA in 30\% of adult, and 17\% of pediatric cases deviates from the
AHA-prescribed \BPG, resulting in worse patient outcomes \cite{Ornato2012DeviationAdult,Wolfe2020DeviationPediatric,
Crowley2020DeviationAdult,Honarmand2018Adherence,Mcevoy2014Adherence}.

While \BPG{}-adherence is difficult to achieve in
practice \cite{RandJAMA99,DavisCMAJ97},
integrating \BPGs{} with existing patient care-flow,
and making them readily-accessible when required can improve adherence \cite{WoolfBMJ99}.
To this end, hospitals commission computerized Decision Support Systems (\CDSSs{})
that codify \BPGs{} and support \HCPs{} with situation-specific advice.
Such systems have been shown to improve \BPG{}-adherence \cite{GargJAMA06,KawamotoBMJ05}, and evidence from multi-center clinical trials
suggests that they reduce \PMEs{} \cite{BenettJAMIA16,SahotaJIS11}.
Thus, guideline-based \CDSSs{} are now considered imperative to the
future of medical decision making in general \cite{JamesNEJM01}.
A guidelines-based \CDSS{} typically consists of:
\begin{enumerate*}[label=(\roman*)]
  \item a translation of the guideline to an executable medium, called the
  \BPGLogic{},
  \item an interface for user-interaction, and,
  \item additional infrastructure that integrates with external data sources
  such as sensors, health records \cite{SuttonNature20}.
\end{enumerate*}
Typically, to develop a \CDSS{}, domain experts in medicine
collaborate with computer scientists to develop requirements documentation
that presents the \BPGs{}'s
semantics in a manner amenable to software development \cite{PelegJBI13}.
This documentation is then utilized to develop the \BPGLogic{}, which is subsequently integrated with data sources (such as patient-parameter sensors and health records),
and a User Interface (UI) to obtain a complete system. Thus, the \BPG{}
serves as a functional specification for the \CDSS{}'s \BPGLogic{}.
But, the aforementioned process has several limitations.
First, the implementation, i.e., the \BPGLogic{} may not concur with its specification,
i.e., the text-based \BPG{}. \BPGs{} are specified as long, complex textual documents,
where the exact meaning of terms may not be explicitly stated, and recommendations may be ambiguous \cite{ClerqAIM03}.
Capturing and communicating these complexities via requirements documentation
is challenging, and incorrect or incomplete documentation has resulted in failed
implementations \cite{KubbenBook19}. Second, as \BPGs{} evolve to reflect
new evidence or local adaptions, corresponding updates must be made to the
\CDSS{} as well. However, due to the gap between the \BPG{} and the \BPGLogic{},
effort must be expended into bringing the \BPGLogic{} to reflect said updates.

To address above mentioned limitations, several Domain Specific Languages
(\DSLs{}) for directly expressing \BPGLogic{} as Computer Interpretable
Guidelines (\CIGs{}) have been introduced. By providing mechanisms to facilitate
representation of medical knowledge, such \DSLs{} allow the \CIG{} to serve
as both the system specification, i.e. the \BPG{}, and implementation, i.e. the
\BPGLogic{}. This ensures that there is no gap between the \BPG{} and
its executable counterpart.
Given the safety-critical nature of \CDSSs{}, the need for formally verified
execution engines and analysis tools has been recognized.
To this end, some existing \DSLs{} have partially-defined semantics and
support for verification via model-checking.
However, as identified by the authors of \cite{SuttonAMIA03, ShaharAMIA96},
existing languages lack a complete formal and executable semantics,
interpreters or compilers with correctness guarantees,
and a comprehensive suite of accompanying tools such as model-checkers, symbolic-execution
engines, and deductive verifiers. The difficulties of formal analysis are further compounded
by the fact that \CDSSs{} are concurrent systems involving interactions with
heterogeneous external agents such as sensors and
\HCPs{}, making their analysis challenging.

This proposal aims to utilize the \emph{semantics-first} approach
to building \emph{safe} \CDSSs{}. By \emph{semantics-first}, we
mean that:
\begin{enumerate*}[label=(\alph*)]
  \item the \BPGLogic{} of the \CDSS{} is semantically accurate, and,
  \item the execution semantics of the underlying programming language
    is formally defined, from which tools such as an interpreter, model checker,
    and deductive verifier are derived in a \emph{correct-by-construction}
    manner.
\end{enumerate*}
At the core of our approach is new language Domain Specific Language (DSL)
for \CIGs{} called \MediK{}. By emphasizing \HCP{}-\emph{comprehensibility},
\MediK{} enables \HCPs{} to verify the semantic correctness of a \CDSSs{}' \BPGLogic{}.
It has a complete formal executable semantics specified in the $\K{}$ framework,
from which a correct-by-construction interpreter and analysis tools
such as a model checker and deductive verifier for its programs are derived.

A comprehensive \MediK{}-based approach for building \CDSSs{}
needs to address the following research challenges (RCs):

\paragraph{RC 1:} Is \MediK{}'s design conducive to expressing diverse
\BPGs{}?

\BPGs{} can vary greatly by scope and purpose. For instance,
consider differences between the \BPGs{} for managing cardiac
arrest and sepsis. While the \BPG{} for cardiac arrest can be
succinctly depicted by a single workflow, the \BPG{} for
management of sepsis involves multiple workflows with complex
inter-workflow interactions. This proposal seek to answer whether \MediK{}'s
design can adequately accomodate diversity in \BPGs{}, without
compromising on readiability.

\paragraph{RC 2:} Can \MediK{}'s toolchain be used to establish
appropriate safety and liveness properties?

\MediK{} has a comprehesive suite of \emph{correct-by-construction}
analysis tools derived from its semantics. But, as real-world \BPGs{}
are complex, establishing appropriate safety and liveness properties using
said tools presents various challenges. The proposal seeks to build on
\MediK{}'s toolchain to support verification of desired safety and liveness
properties.


%We address these by introducing \MediK{} (pronounced Medi-kay),
%a \DSL{} for expressing a \CDSS{}'s \BPGLogic{} as concurrently-executing state machines. \MediK{}
%provides:
%\begin{enumerate}
%  \item A \stress{complete executable formal semantics} specified in the $\K{}$
%  semantics framework.
%  \item A \stress{correct-by-construction interpreter}, and \stress{analysis
%  tools} such as a model-checker and deductive verifier.
%  \item A uniform way of modeling \stress{heterogeneous agents} for both
%  \emph{execution} and \emph{analysis}.
%\end{enumerate}


