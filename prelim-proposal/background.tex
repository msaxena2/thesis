\section{Background}
In this section, we go over relevant background work.

\subsection{Semantics-First Approach}

The semantics first approach dictates that the semantics
of a language should be formally defined. Tools for the language
such as interpreters, compilers, model checkers and deductive
verifiers should be derived from the semantics in a
\emph{correct-by-construction} fashion, instead of being implemented
from scratch in an ad-hoc manner.

For a language like \MediK{}, the semantics first approach has many benefits. First,
since \MediK{} is meant to be used in safety-critical settings,
it's vital that its interpreter has correctness guarantees, which
the approach ensures. Second, \MediK{} has to evolve quickly
to incorporate feedback that it receives from domain experts.
Such changes in the semantics-first approach only require changes to
the semantics, and all tools are updated automatically.
