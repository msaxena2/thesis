\section{Background}
In this section, we go over relevant background work.

\subsection{Limitations of Existing Approaches}
\begin{center}
\renewcommand{\arraystretch}{0.5}
%\setlength\extrarowheight{-9pt}
  \begin{table}
  \begin{tabularx}{\textwidth}{
      >{\centering\arraybackslash}X
    || >{\centering\arraybackslash}X
    | >{\centering\arraybackslash}X
    | >{\centering\arraybackslash}X
    | >{\centering\arraybackslash}X
  }
                 & Implementation-Specification Gap & Complete Formal Semantics & Formal Analysis Tools & Holistic Safety  \\
    Arden Syntax & $\greencheck$                               & $\redcross$               & $\redcross$           & $\redcross$ \\
    GLIF         & $\greencheck$                               & $\redcross$               & $\redcross$           & $\redcross$ \\
    Asbru        & $\greencheck$                               & $\cancelcheck$            & $\greencheck$         & $\redcross$ \\
    PROForma     & $\greencheck$                               & $\greencheck$             & $\redcross$           & $\redcross$ \\
    GLARE        & $\greencheck$                               & $\cancelcheck$            & $\cancelcheck$        & $\cancelcheck$ \\
    Promela/SPIN & $\redcross$                                 & $\greencheck$             & $\greencheck$         & $\cancelcheck$ \\
    AMSs         & $\redcross$                                 & $\greencheck$             & $\greencheck$         & $\redcross$ \\
    SAGE         & $\greencheck$                               & $\redcross$               & $\redcross$           & $\redcross$ \\
  \end{tabularx}
  \caption{Comparison of Existing Approaches}\label{table:existing-approaches}
  \end{table}
\end{center}

While existing approaches have been imperative to increasing \CGS{} adoption, to the
best of our knowledge, none of them address all of the aforementioned
limitations. We briefly describe notable approaches, their successes, and their limitations.

The Arden Syntax \cite{HripcsakCBM94} a widely used medium for
expressing \CIGs{}.  Guidelines as described using Medical
Logic Modules that contains information related to guideline's purpose
, maintainance, and medical knowledge. The modules are modular to allow
re-use and sharing across hospitals. But, Arden Syntax
is focused on describing simple, modular, and independent
guidelines (such as reminders), and not on guidelines with complex logic (such
as treatment protocols) \cite{PelegJBI01}.
Arden Syntax's limitation in modeling complexity is addressed by
GLIF \cite{BoxwalaJBI04}: a language that uses flowcharts to expressed
guidelines. A multi-level approach is
employed to manage complexity: at the top is the conceptual level, where
only high-level details relevant for human-comprehension are present. In the
middle is a computable-level, where details of guideline execution flow
and patient data elements are specified. At the bottom is the implementable
level, where institution-specific details and mappings into patient data are
specified. Both Arden Syntax and GLIF  eliminate
the gap between the \BPG{}, i.e. the specification, and the \CIG{}, i.e. implementation as
they're meant to be either directly used by clinicians (or in collaboration with
computer scientists) to express \BPGs{} in an executable medium. \CIGs{}
expressed in them are meant to be shared across hospitals, and are thus modular.
However, neither formalism has complete formal semantics, or comprehensive support for
rigorous formal analysis.

The need for formal analysis is identified by Asbru: a formalism with formally
defined syntax and semantics \cite{ShaharAMIA96}. In Asbru, a guideline is modeled as a plan
that contains:
\begin{enumerate*}[label=(\roman*)]
  \item intentions that define aims,
  \item conditions that specify when the plan is applicable,
  \item effects that define expected behavior during execution, and,
  \item a body containing other subplans.
\end{enumerate*}
Apart from an execution engine, the Asbru ecosystem also contains
other tools, such as a model checker for verification \cite{BaumlerSPIN06}.
However, the formal semantics of Asbru have been only partially defined, and
is insufficient to implement tools for the language \cite{SuttonAMIA03}.
The importance of a complete formal-semantics is identified and addressed
by PROforma \cite{SuttonAMIA03}, another formalism that uses plans to
model guidelines. A PROforma plan is made of a sequence of tasks.
The plan defines constraints on their enactment, and circumstances
for termination (for example, exceptions) \cite{SuttonAMIA03}. But, despite
having complete formal semantics, it does not have a comprehensive suite of
formal analysis tools such as model checkers, deductive verifiers.

The SAGE guideline model \cite{TuSAGE04} uses the Prot\'eg\'e knowledge
representation framework \cite{NoyAMIA03} to model guidelines,
and improves on aforementioned approaches by
enabling seamless integration into hospitals' existing Clinical Information Systems
(\CISs). But, it lacks complete formal semantics, and analysis tools
such as deductive verifiers and model checkers.
The GLARE formalism \cite{TerenzianiBook04} uses an actions based approach
to represent guidelines, and addresses clnician-comprehensibility and
modularity. For formal analysis, GLARE guidelines can be translated to
Promela: the SPIN model checker's specification language \cite{GiordanoAMIA06}.
The approach partly addresses holistic safety as
external agents (such as clinicians) can be modelled and analyzed.
But, the scenario where the external agent's behavior
deviates from the model during system execution isn't addressed.
Non medical-domain specific languages can also be used to reason about
medical systems. For example, in \cite{ArcainiMEMCODE15}, Abstract State
Machines (\ASMs) are used to validate and verify a system for measuring
patients' stereoacuity in the diagnosis of amyblyopia. But such a
formalism, while suitable for formal verification, may
not be easily comprehensible to clinicians for validation.

In \tablename{} \ref{table:existing-approaches}, we provide an overview of
the strengths and limitations of existing approaches. Note that we use
\greencheck{}, \cancelcheck{}, and \redcross{} to depict that an approach
fully-addresses, partly-addresses, or doesn't address a limitation respectively.


\subsection{Semantics-First Approach}

The semantics first approach dictates that the semantics
of a language should be formally defined. Tools for the language
such as interpreters, compilers, model checkers and deductive
verifiers should be derived from the semantics in a
\emph{correct-by-construction} fashion, instead of being implemented
from scratch in an ad-hoc manner.

For a language like \MediK{}, the semantics first approach has many benefits. First,
since \MediK{} is meant to be used in safety-critical settings,
it's vital that its interpreter has correctness guarantees, which
the approach ensures. Second, \MediK{} has to evolve quickly
to incorporate feedback that it receives from domain experts.
Such changes in the semantics-first approach only require changes to
the semantics, and all tools are updated automatically.
