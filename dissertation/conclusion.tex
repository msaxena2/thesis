\chapter{Conclusion}\label{chapter:conclusion}

Chapter \ref{chapter:introduction} explains the persistent challenge
that preventable medical errors (\PMEs{}) present in medicine.
The seriousness of the problem can be gauged by the fact that
President's Council on Science and Technology (\PCAST{})
declared the prevalence of \PMEs{} to be an
\say{urgent national public health issue}.

Among the solutions
outlined in the PCAST{} report include:
\begin{enumerate*}[label=(\roman*)]
  \item Utilizing \emph{evidence-based} medicine.
  \item Investing in technology that assists healthcare
  practitioners (\HCPs{}) in decision making.
\end{enumerate*}
Clinical best practice guidelines (\BPGs{}) exemplify evidence-based medicine.
Produced by hospital and medical associations, these guidelines
are evidence-based statements that codify recommended treatment for
various clinical scenarios, and growing evidence supports their use
in medicine to reduce \PMEs{}. But, their effectiveness in practice is often
limited by challenges in their uptake and adoption. As outlined in
\autoref{chapter:hurdles-cdss-adoption}, integrating \BPGs{} into \HCP{} workflows
and following them in practice is often difficult, resulting in treatment
that deviates from prescribed best-practices, and as a result, leads to
worse outcomes. Guidelines-based clinical decision support systems
that assist \HCPs{} with situation-specific advice to enhance guideline
adherence have hence been shown to improve patient outcomes.

Despite many apparent benefits, the prevalence of \CDSSs{} in practice
remains low, due to factors outlined in \autoref{chapter:hurdles-cdss-adoption}.
In this work, we utilize a semantics-first approach to building safe clinical
decision support systems.

By semantics-first, we mean that:
\begin{itemize}
  \item The semantics of medical knowledge are accurately captured in the
  system.
  \item The language for expressing medical knowledge has a well defined, formal
  semantics from which tools for execution and formal analysis are derived.
\end{itemize}

\CDSSs{} are typically developed by software developers lacking expertise in
medicine. Thus, it is possible for medical knowledge in a \CDSS{} to be incorrectly
encoded during development. As \CDSSs{} operate in safety critical settings, such
bugs can lead to adverse outcomes for patients. To address this,
we developed a domain specific language for describing medical knowledge
called \MediK{}. Developed as an academic endeavor, \MediK{} has been designed to be
comprehensible to \HCPs{}. This allows \HCPs{} to examine and validate the accuracy
medical knowledge in a \MediK-based \CDSS{}, enabling greater transparency in
the source of the \CDSS{}'s recommendations.

As outlined in \autoref{chapter:related-work}, the need for \DSLs{} for
expressing \BPGs{} to have formal semantics and support for comprehensive
support for formal analysis has already been recognized. \MediK{} builds
on the state-of-art in modeling concurrent systems and medical knowledge
to enable complex medical guidelines with multiple concurrent workflows
and interdependencies to be conveniently expressed. To demonstrate this,
we collaborate with OSF St. Francis Hospital in Peoria, Illinois--a
major pediatric hospital to develop a \CDSS{} for their pediatric
sepsis management guidelines. During the development process, bugs in
medical logic were uncovered by collaborating physicians by examining
\MediK{} code, and learnings from our endeavor were incorporated into the
design of the language.

We believe the work outlined in this thesis presents the first steps towards
addressing challenges inhibiting broader \CDSS{} uptake. But, a comprehensive
solution involves addressing several more challenges, some of which are
outlined \autoref{chapter:future-work}. Some of these challenges demand
collaboration from multiple stake-holders, from both computer-science and
medicine. Thus, they remain unaddressed for now. But, we hope addressing
them, as part of building on top of work presented here, can lower \PMEs{},
leading to better quality healthcare at reduced costs.
