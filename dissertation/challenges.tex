\chapter{Hurdles to \CDSS{} Adoption}

There is now increasing evidence to suggest that
well implemented \CDSSs{} can significantly improve quality of care
\cite{GargJAMA05,WellsEJPC08}. However, despite several advantages,
several challenges continue to inhibit wider \CDSS{} adoption \cite{Nam17}.
Some challenges are non-technical, i.e., require changes to legislation,
incentive mechanisms and practitioner education and training, and are beyond the
scope of this work. But, several limitations in existing \CDSS{} technology have also
inhibited further adoption. This chapter discusses said challenges, and
the progress made by existing state of art towards addressing them.

Recall, from section \ref{sec:hurdles-cdss-adoption}, that in 2017,
the National Academy of Medicine published a report on \CDSSs{} that
laid out a roadmap to optmize \CDSS{} uptake in medicine. According to the
report, several challenges need to be tackled for wider adoption, such as:


\begin{enumerate}[label=C\arabic*.]
\itemsep0.0em
\item Absence of systematic ways of \emph{validating content}
in a \emph{reliable}, \emph{accessible} and \emph{updateable} manner.
\item Lack of \emph{reliable}, \emph{shareable} \CDSS{} content
that can be easily adopted across healthcare organizations and their (Information
Technology) \IT{} systems.
\item Technical difficulties of sharing due to \emph{need for
  adaptation} to diverse Electronic Health Records (\EHR) systems.
\item \emph{Suboptimal} User Interfaces (\UIs), implementation choices and
workflows.
\end{enumerate}

\section{Addressing Adoption Hurdles}

\CDSS{} first appeared in the 1960s, and have evolved over time
to address aforementioned challenges. The following sections
describe progress made towards addressing aforementioned challnges.

\subsection{Monolithic \CDSSs{}}\label{sec:monolithic-cdss}

Early \CDSSs{} were developed as monolithic standalone systems
that were self-contained, requiring direct user input for clinical data
\cite{RodriguezBook16}. Such systems co-existed with
primitve electronic health records (\EHR{}) systems,
and thus had to rely on manual data entry before administering support
\cite{ShortliffeBook12,BarnettJAMA87}. The reliance on manual entry
made using such systems time-consuming. As support for \EHR{} matured,
\CDSSs{} implementations became better integrated with \EHR{} systems
for automated clinical data retrieval. However,
the integration was usually \EHR{} specific \cite{RodriguezBook16}.
Migrating or sharing \CDSS{} content across medical establishments
presented significant challenges as a system desgined
for an establishment's \EHR{} system couldn't be used with a different
establishment's \EHR{} system \cite{KawamotoJBI10}.

\subsection{Modular \CDSS{} Architectures}

In section \ref{sec:cdss-components}, we presented the components that
every guidelines-based clinical decision suppport can conceptually be decomposed
into. Early \CDSSs{} from section \ref{sec:monolithic-cdss}
were not designed with modularty that enabled sharing components
between different implementations. As the need for scaling \CDSSs{}
across institutions grew, component-based architectures that
enabled \EHR{} agnostic systems to be developed became prevalent
\cite{KawamotoJBI10}.




