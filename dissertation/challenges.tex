\chapter{Hurdles to \CDSS{} Adoption}\label{chapter:hurdles-cdss-adoption}

There is now increasing evidence to suggest that
well implemented \CDSSs{} can significantly improve quality of care
\cite{GargJAMA05,WellsEJPC08}. However, despite several advantages,
several challenges continue to inhibit wider \CDSS{} adoption \cite{Nam17}.
Some challenges are non-technical, i.e., require changes to legislation,
incentive mechanisms and practitioner education and training, and are beyond the
scope of this work. But, several limitations in existing \CDSS{} technology have also
inhibited further adoption. This chapter discusses said challenges, and
the progress made by existing state of art towards addressing them.

Recall, from section \ref{sec:hurdles-cdss-adoption}, that in 2017,
the National Academy of Medicine published a report on \CDSSs{} that
laid out a roadmap to optmize \CDSS{} uptake in medicine. According to the
report, several challenges need to be tackled for wider adoption, such as:


\begin{enumerate}[label=C\arabic*.]
\itemsep0.0em
\item Absence of systematic ways of \emph{validating content}
in a \emph{reliable}, \emph{accessible} and \emph{updateable} manner.
\item Lack of \emph{reliable}, \emph{shareable} \CDSS{} content
that can be easily adopted across healthcare organizations and their (Information
Technology) \IT{} systems.
\item Technical difficulties of sharing due to \emph{need for
  adaptation} to diverse Electronic Health Records (\EHR) systems.
\item \emph{Suboptimal} User Interfaces (\UIs), implementation choices and
workflows.
\end{enumerate}

\section{Addressing Adoption Hurdles}

\CDSS{} first appeared in the 1960s, and have evolved over time
to address aforementioned challenges. The following sections
describe progress made towards addressing aforementioned challnges.

\subsection{Monolithic \CDSSs{}}\label{sec:monolithic-cdss}

Early \CDSSs{} were developed as monolithic standalone systems
that were self-contained, requiring direct user input for clinical data
\cite{RodriguezBook16}. Such systems co-existed with
primitive electronic health records (\EHR{}) systems,
and thus had to rely on manual data entry before administering support.

Several successful \CDSSs{} implementations utilized a standalone
architecture. Early \CDSS{} implementations
such as MYCIN \cite{ShortliffeBook12} required the \HCP{}
to answer a set of questions to provide advice regarding microbial therapy.
Other early \CDSSs{} such as DXplain \cite{BarnettJAMA87} utilized a wide
range of findings (history, data, etc.) to come up with a diagnosis, and
is still in active development.

The reliance on manual entry made using such systems time-consuming. As support for \EHR{} matured,
\CDSSs{} implementations became better integrated with \EHR{} systems
for automated clinical data retrieval.
However, with monolithic systems, the integration was usually \EHR{}-specific \cite{RodriguezBook16}.
Migrating or sharing \CDSS{} content across medical establishments
presented significant challenges as a system designed
for an establishment's \EHR{} system couldn't be used with a different
establishment's \EHR{} system \cite{KawamotoJBI10}.

\subsection{Modular \CDSS{} Architectures}\label{sec:modular-architectures}

In section \ref{sec:cdss-components}, we presented the components that
every guidelines-based clinical decision support can conceptually be decomposed
into. Early \CDSSs{} from section \ref{sec:monolithic-cdss}
were not designed with modularity that enabled sharing components
between different implementations. As the need for scaling \CDSSs{}
across institutions grew, component-based architectures that
enabled \EHR{} agnostic systems to be developed became prevalent \cite{KawamotoJBI10}.

\EHR{}-agnostic architectures represent a significant step towards
addressing several challenges. Such architectures
address C3 as the knowledge-base can be shared across institutions with
different \EHR{} systems. C2 is partially addressed as the
knowledge-base can be independently developed, maintained and distributed.
C4 is also partially addressed as decoupled components, such as the \UI{},
are easier to adapt to \HCP{} preferences.

Over the years, several \CDSS{} implementations have utilized a
components-based architecture. For instance, in \cite{KawamotoJBI10}, the
authors utilize the service-oriented architecture \cite{ErlBook05} to build
a \CDSS{} web service that can be utilized in a completely \EHR{}-agnostic way.
Recent efforts include \CDSSs{} platforms such as
EvidencePoint that enable \CDSS{} to be integrated
closely with the hospital's \EHR{} without being tightly coupled \cite{SolomonJMIR23}.
\EHR{}-agnostic architectures allow decision support to be administered using the
\EHR{}'s \UI{}. Given their prevalence in modern medical establishments,
\EHR{} systems have become integrated into workflows,
and HCPs are accustomed to using them. Dispensing clinical decision
support through the \EHR{}'s \UI{} is vital for adoption, as better workflow
integration can lead to higher adoption \cite{PressJMIR16,LiJMI16}.

Approaches that utilize a component-based architecture
enable medical knowledge to be shared more efficiently.
But, it's possible for medical knowledge itself to be incorrectly
encoded. \BPGs{} are generally expressed as long textual documents meant to
be understood by \HCPs{} \cite{SchiffmanYMI13}. To build a \CDSS{}, the \BPG{} has to be
systematically expressed in a computable medium. This translation process,
referred to as knowledge formalization, is generally ad-hoc, and can be
the source of inconsistencies in encoded medical knowledge
resulting in \CDSSs{} that render wrong advice \cite{ShaharIOS04}.

Typically, in order to translate textual guidelines to a computable
medium, experts in medicine collaborate with computer scientists and
software developers to come up with a requirements document.
This is subsequently utilized to develop the knowledge base, i.e.,
the computer interpretable encoding of the \BPG{} \cite{PelegJBI13} in a
traditional programming language. Thus, the \BPG{} as a
functional specification for the knowledge base.
For instance, in the case of the aforementioned EvidencePoint,
the knowledge base is expressed in  Javascript \cite{SolomonJMIR23}.

However, since computer scientists/software
developers don't understand medicine, and experts in medicine typically
don't understand programming languages,
it's possible for the functional specification, i.e., the \BPG{}, to
diverge from its implementation, i.e., the knowledge base. Thus,
C1 from section \ref{chapter:hurdles-cdss-adoption} remains unaddressed.

\subsection{Computer-Interpretable \BPGs{}}

\CDSSs{} are safety-critical systems, where bugs can have serious consequences.
Thus, ensuring correctness of \CDSSs{} is vital to widespread adoption.
To ensure bugs arising out of divergences between the textual \BPG{}
and its computable translation, i.e., the knowledge base, the gap between
the \BPG{} and the knowledge base must be eliminated. To this end, instead of
encoding \BPGs{} in a conventional programming language, domain-specific
languages (\DSLs{}) for directly expressing \BPGs{} in a computer-interpretable manner can
be utilized. By emphasizing comprehensibility to \HCPs{}, these \DSLs{} enable
\HCPs{} to validate the accuracy of encoded medical knowledge.
A computer-interpretable \BPGs{} can both as the specification, i.e. the textual \BPG{},
and the implementation, i.e., the knowledge base, thereby eliminating the
specification-implementation gap.

In 1989, the Arden Syntax was developed as a result of incompatibility
of medical knowledge between medical institutions \cite{HripcsakCBM94}.
While it was designed to represent simple guidelines,
such as those related to reminders, more complex treatment protocols couldn't
be represented. Arden syntax had a formal Backus-Naur Form (\BNF{})
syntax definition, but no formal semantics, or a standardized execution engine.
Instead, ad-hoc execution engines for the language have been developed over
time \cite{ClerqAIM03}.

Shortcomings regarding ability of the Arden Syntax to represent complex
treatment protocols were addressed by formalisms such as GLIF \cite{PelegAMIA00}
that permit encoding complex guidelines through the use of a multi-layer
approach to represent both high-level medical knowledge and low-level
implementation details. However, just like the Arden Syntax, GLIF lacks
a formal semantics or formal analysis tools.

The need for formal analysis is identified by Asbru: a formalism with formally
defined syntax and semantics \cite{ShaharAMIA96}. In Asbru, a guideline is modeled as a plan
that contains:
\begin{enumerate*}[label=(\roman*)]
  \item intentions that define aims,
  \item conditions that specify when the plan is applicable,
  \item effects that define expected behavior during execution, and,
  \item a body containing other subplans.
\end{enumerate*}
Apart from an execution engine, the Asbru ecosystem also contains
other tools, such as a model checker for verification \cite{BaumlerSPIN06}.
However, the formal semantics of Asbru have been only partially defined, and
is insufficient to implement tools for the language \cite{SuttonAMIA03}.
The importance of a complete formal-semantics is identified and addressed
by PROforma \cite{SuttonAMIA03}, another formalism that uses plans to
model guidelines. A PROforma plan is made of a sequence of tasks.
The plan defines constraints on their enactment, and circumstances
for termination (for example, exceptions) \cite{SuttonAMIA03}. But, despite
having complete formal semantics, PROforma's semantics is not executable.
Therefore, an interpreter and analysis tools have to be implemented in an
ad-hoc manner.

Over the years, significant progress has been made towards making \CDSSs{}
safer, easier, effective and cheaper. In section \ref{sec:monolithic-cdss},
the earliest \CDSSs{} attempted to increased adherence to evidence-based best practices
at medical establishments. To ensure \CDSSs{} could be scaled better,
components-based architectures, discussed in section
\ref{sec:modular-architectures} enabled medical knowledge to
be developed and maintained independently of other components (such as system
\UI{}). To ensure that medical knowledge is indeed encoded correctly,
several \DSLs{} that eliminate the gap between a textual \BPG{} and
the \CDSS{}' knowledge base have been developed. However,
the following issues still remain:
\begin{itemize}
  \item Interpreters and compilers for said \DSLs{} are developed in an
    ad-hoc way, and can be prone to bugs that manifest during execution.
  \item There is a lack of formal analysis tools such as model checkers and
    program verifiers that can be utilized to establish that the
    computer-interpretable \BPGs{} satisfy desired safety and liveness
    properties.
  \item \DSLs{} lack complete formal semantics that can be utilized
    as a language model for a suite of formal analysis and execution tools.
\end{itemize}

Addressing aforementioned issues is vital to improving \CDSS{} adoption.
In order to have trustworthy and useable computer interpretable \BPGs{},
it's important to buttress  \HCP{}-friendly \DSLs{}
with a suite of formal analysis tools that can establish desired safety
properties, thus comprehensively addressing C1.

