\chapter{Semantics-First Approach to Clinical Decision Support}

In chapter \ref{chapter:introduction}, we explained that, despite
advances in medicine, mortality and costs associated with preventable
medical errors (\PMEs{}) remain unacceptably high. In chapter
\ref{chapter:background}, we explained how systems that
assist healthcare practitioners (\HCPs{}) with situation-specific
advice based on evidence-based best practice guidelines (\BPGs{}),
called clinical decision (\CDSSs{}) can reduce both mortality
and costs associated with \PMEs{}. But, despite their potential,
the uptake of such systems in practice is hindered by challenges
that were introduced in section \ref{sec:hurdles-cdss-adoption}, and
discussed in depth in chapter \ref{chapter:hurdles-cdss-adoption}.
In brief, the following challenges (Cs) were outlined:
\begin{enumerate}[label=C\arabic*.]
\itemsep0.0em
\item Absence of systematic ways of \emph{validating content}
in a \emph{reliable}, \emph{accessible} and \emph{updateable} manner.
\item Lack of \emph{reliable}, \emph{shareable} \CDSS{} content
that can be easily adopted across healthcare organizations and their (Information
Technology) \IT{} systems.
\item Technical difficulties of sharing due to \emph{need for
  adaptation} to diverse Electronic Health Records (\EHR) systems.
\item \emph{Suboptimal} User Interfaces (\UIs), implementation choices and
workflows.
\end{enumerate}

Over the years, significant progress has been made towards
addressing these challenges. In chapter \ref{chapter:related-work},
we discussed how existing approaches have attempted to
address said challenges, and their limitations.
