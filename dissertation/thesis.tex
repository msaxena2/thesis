% For copyright and license information, see uiucthesis2021.dtx and derivatives.
\documentclass[edeposit,tocnosub,noragright,centerchapter,fullpagesingle,12pt]{uiuc_csthesis21}

\makeatletter

\usepackage{setspace}  % Useful for single, 1.5, and double spacing
\usepackage[numbers, sort]{natbib}  % Useful for formatting reference section
\usepackage{url}  % Useful for URLs
%\usepackage{hyperref}  % Another package useful for URLs

\usepackage{lscape}  % Useful for wide tables or figures.
% Following command definition is from Stack Exchange: https://tex.stackexchange.com/questions/278113/single-landscape-page-with-page-number-at-the-bottom
% It adds *rotated* page numbers to the bottom of landscaped pages to meet the Graduate College standards (see page 7 here: https://grad.illinois.edu/files/pdfs/thesis-sample-chapter-straight-numbering.pdf)
\def\fillandplacepagenumber{
	\par
	\pagestyle{empty}
	\vbox to 0pt{\vss}
	\vfill
	\vbox to 0pt{
		\baselineskip 0pt
		\hbox to \linewidth{\hss}
		\baselineskip\footskip
		\hbox to \linewidth{\hfil\thepage\hfil}\vss
	}
}

\usepackage[utf8]{inputenc}
\usepackage[inference]{semantic}
\usepackage[english]{babel}
\usepackage{csquotes}
\usepackage{pifont}
\usepackage{stmaryrd}
\usepackage{microtype}
\usepackage{amsmath,amsthm,amssymb}
\usepackage[bookmarksdepth=3,linktoc=all,colorlinks=true,urlcolor=blue,linkcolor=blue,citecolor=blue]{hyperref}
\usepackage[capitalize]{cleveref}
%\usepackage[style=ieee]{biblatex}
\usepackage[inline]{enumitem}
\usepackage{dirtytalk}
\usepackage{graphicx}
\usepackage{wrapfig}
\usepackage{subcaption}
\usepackage{listings}
\usepackage{xcolor}
\usepackage{longtable}
\usepackage{tabularx}
\usepackage{mathtools}
\usepackage{varwidth}
\usepackage{proof}
\usepackage[bottom]{footmisc}
\usepackage[bbgreekl]{mathbbol}


\DeclareSymbolFontAlphabet{\mathbbm}{bbold}
\DeclareSymbolFontAlphabet{\mathbb}{AMSb}%

% \usepackage{ruledchapters}  % example of compliant heading format, uncomment to use

% uncomment the below to show a grid on all pages
% \usepackage[grid, gridunit=in, gridcolor=blue!40, subgridcolor=blue!20]{eso-pic}

\theoremstyle{definition}
\newtheorem{definition}{Definition}[chapter]
\newtheorem{lemma}{Lemma}[chapter]
\newtheorem{theorem}{Theorem}[chapter]
\newtheorem{corollary}{Corollary}[chapter]
\newtheorem{conjecture}{Conjecture}[chapter]
\newtheorem{remark}{Remark}[chapter]

\renewcommand{\qedsymbol}{QED.}  % Change symbol at end of proofs to meet the Graduate College standard
%%%%%%%%%%%%%%%%%%%%%%%%%%%%%%%%%%%%%%%%%%%%%%%%%%%%%%%%%%%%%%%%%%%%%%%%%%%%%%%
% ALGORITHM AND CODE PACKAGES (Comment out this section if unnecessary for your dissertation)
%
\usepackage{listings}  % Useful for formatting code blocks, see here for further information about formatting code: https://en.wikibooks.org/wiki/LaTeX/Source_Code_Listings
\usepackage[ruled,linesnumbered]{algorithm2e}  % Useful for formatting algorithms (pseudocode)
\numberwithin{algocf}{chapter}     % Change numbering of algorithms to meet the Graduate College standards


\definecolor{greybackground}{rgb}{0.95,0.95,0.92}
\definecolor{codegreen}{rgb}{0,0.6,0}
\definecolor{codegray}{rgb}{0.5,0.5,0.5}
\definecolor{codepurple}{rgb}{0.58,0,0.82}

% K lst definition
\lstdefinestyle{ksty}{
  keywordstyle=\color{magenta},
  basicstyle=\ttfamily\small,
  commentstyle=\color{codepurple},
  backgroundcolor=\color{greybackground},
  framerule=0pt
}
\lstdefinestyle{inlineksty}{
  basicstyle=\ttfamily\small
}
\lstdefinelanguage{k}{
  morekeywords={rule,configuration,=>,syntax,multiplicity,type,module,endmodule,import,imports, left,strict,seqstrict,bracket,structural,requires},
  morecomment=[l]{//},
  morecomment=[s]{/*}{*/}
}

% MediK lst definition
\lstdefinestyle{mediksty}{
  keywordstyle=\color{magenta},
  basicstyle=\ttfamily\small,
  commentstyle=\color{codepurple},
  backgroundcolor=\color{greybackground},
  framerule=0pt
}
\lstdefinestyle{inlinemediksty}{
  basicstyle=\ttfamily\small,
}
\lstdefinelanguage{medik}{
  morekeywords={either, or, machine, interface, vars, state, entry, on, do, goto, receives, ~>, =>},
  morecomment=[l]{//},
  morecomment=[s]{/*}{*/}
}

%Imp lst definition
\lstdefinestyle{impsty}{
  keywordstyle=\color{magenta},
  basicstyle=\ttfamily\small,
  backgroundcolor=\color{greybackground},
  commentstyle=\color{codepurple},
  framerule=0pt
}
\lstdefinelanguage{imp}{
  morekeywords={if, while, var}
  morecomment=[l]{//},
  morecomment=[s]{/*}{*/}
}

\newcommand{\inlinemedik}[1]{\lstinline[style=inlinemediksty,language=medik]{#1}}
\newcommand{\inlinek}[1]{\lstinline[style=inlineksty,language=k]{#1}}
\newcommand{\inlineimp}[1]{\lstinline[style=impsty,language=imp]{#1}}
\newcommand{\inlinekmath}[1]{\lstinline[mathescape,style=inlineksty,language=k,keepspaces]!#1!}
\newcommand{\inlinemedikmath}[1]{\lstinline[mathescape,style=inlinemediksty,language=medik]!#1!}
\newcommand{\inlinemedikimp}[1]{\lstinline[mathescape,style=inlineimpsty,language=imp]!#1!}
\lstset{ literate={~}{{\raisebox{0.5ex}{\texttildelow}}}{1} }
\lstset{captionpos=b,escapeinside={@}{@}}
\providecommand*{\lstnumberautorefname}{Line}
\def\boxit#1{%
  \smash{\color{red}\fboxrule=1pt\relax\fboxsep=2pt\relax%
  \llap{\rlap{\fbox{\vphantom{0}\makebox[#1]{}}}~}}\ignorespaces
}

\graphicspath{{./figures}}

%\addbibresource{./references.bib}

%\newcounter{counterforappendices}

\newcommand{\frontend}{\emph{frontend}}
\newcommand{\BPG}{BPG}
\newcommand{\BPGs}{BPGs}
\newcommand{\CGS}{CGS}
\newcommand{\CGSs}{CGSs}
\newcommand{\HCP}{HCP}
\newcommand{\HCPs}{HCPs}
\newcommand{\ED}{ED}
\newcommand{\EDs}{EDs}
\newcommand{\CDSS}{CDSS}
\newcommand{\CDSSs}{CDSSs}
\newcommand{\BPGLogic}{knowledge-base}
\newcommand{\K}{\mathbb{K}}
\newcommand{\MediK}{\text{Medi}\K{}}
\newcommand{\FSM}{\emph{FSM}}
\newcommand{\FSMs}{\emph{FSMs}}
\newcommand{\Var}{\text{Var}}
\newcommand{\LHS}{\emph{\text{LHS}}}
\newcommand{\RHS}{\emph{\text{RHS}}}
\renewcommand{\phi}{\varphi}
\newcommand{\GUI}{GUI}
\newcommand{\UI}{UI}
\newcommand{\UIs}{UIs}
\newcommand{\GUIs}{GUIs}
\newcommand{\PME}{PME}
\newcommand{\PMEs}{PMEs}
\newcommand{\CIG}{CIG}
\newcommand{\CIGs}{CIGs}
\newcommand{\EHRs}{EHRs}
\newcommand{\ACLS}{ACLS}
\newcommand{\CPR}{CPR}
\newcommand{\CISs}{CISs}
\newcommand{\RTSs}{RTSs}
\newcommand{\ASMs}{ASMs}
\newcommand{\DSL}{\text{DSL}}
\newcommand{\DSLs}{\text{DSLs}}
\newcommand{\IT}{IT}
\newcommand{\EHR}{EHR}
\newcommand{\ONC}{ONC}
\newcommand{\NAM}{NAM}
\newcommand{\BNF}{BNF}
\newcommand{\MLM}{\text{MLM}}
\newcommand{\MLMs}{\text{MLMs}}
\newcommand{\GLIF}{\text{GLIF}}
\newcommand{\GEODECM}{\text{GEODE-CM}}

% Convenience Commands
\newcommand{\cmark}{\text{\ding{51}}}
\newcommand{\xmark}{\text{\ding{55}}}
\newcommand{\greencheck}{{\color{green}\cmark}}
\newcommand{\redcross}{{\color{red}\xmark}}
\newcommand{\cancelcheck}{\bcancel{\cmark}}
\newcommand{\stress}[1]{\underline{\emph{#1}}}

% Scheduling Commands
\newcommand{\Machine}{\mathcal{M}}
\newcommand{\Instance}{\mathcal{I}}
\newcommand{\scheduled}{\textit{scheduled}}
\newcommand{\enabled}{\textit{enabled}}
\newcommand{\epoch}{\textit{epoch}}



\phdthesis
%\otherdoctorate[abbrev]{Title of Degree}
%\othermasters[abbrev]{Title of Degree}

\title{A Semantics-First Approach to Safe Guidelines-based Clinical Decision Support}
\author{Manasvi Saxena}
\department{Computer Science}
\degreeyear{2024}

% Advisor name is required for
% - doctoral students for the ProQuest abstract
% - master's students who do not have a master's committee
\advisor{Professor Grigore Ro\c{s}u}

% Uncomment the \committee command for
% - all doctoral students
% - master's students who have a master's committee
\committee{Professor Grigore Ro\c{s}u, Chair\\
        Professor Jose Meseguer \\
        Professor Lui Sha \\
        Doctor Serdar Tasiran, Amazon Web Services (AWS)} % etc.

\begin{document}

%%%%%%%%%%%%%%%%%%%%%%%%%%%%%%%%%%%%%%%%%%%%%%%%%%%%%%%%%%%%%%%%%%%%%%%%%%%%%%%
% COPYRIGHT
%
%\copyrightpage
%\blankpage

%%%%%%%%%%%%%%%%%%%%%%%%%%%%%%%%%%%%%%%%%%%%%%%%%%%%%%%%%%%%%%%%%%%%%%%%%%%%%%%
% TITLE
%
\maketitle

%\raggedright
\parindent 1em%

\frontmatter

%%%%%%%%%%%%%%%%%%%%%%%%%%%%%%%%%%%%%%%%%%%%%%%%%%%%%%%%%%%%%%%%%%%%%%%%%%%%%%%

\begin{abstract}
  Preventable medical errors (PMEs), characterized by misdiagnosis or mistreatment,
  pose a significant challenge in healthcare.
  %According to a September 2023 report
  %by the President's Council on Science and Technology (PCAST) on patient safety,
  %a quarter of Medicare patients experience adverse outcomes during hospitalization,
  %of which 40\% were due to PMEs.
  In the United States, PMEs were estimated to have caused between
  44,000 and 98,000 deaths in 1997,
  rising to more than 250,000 in 2013. Additionally, the
  financial burden of PMEs to the U.S. economy in 2008 was estimated at \$19.5 billion.

  One strategy to reduce PMEs in medicine is through the use of
  clinical best practice guidelines (BPGs). BPGs are systematically developed,
  evidence-based statements published by medical institutions and associations
  that standardize diagnosis and treatment for various clinical scenarios.
  %Growing evidence indicates BPG-conformant treatment lowers the risk of preventable
  %medical errors.
 % However, following these guidelines in practice can be
 % challenging.
  %To assist with BPG-conformance, computerized
  %decision support systems that encode
  %medical knowledge in BPGs, and provide HCPs with relevant, situation-specific,
  %guideline-prescribed advice can be utilized.
  BPG-conformance has been linked with reduced rates of PMEs, but,
  following BPGs in practice can be challenging.
  Computerized Decision Support Systems (CDSSs) aim to improve conformance
  by encoding medical knowledge in BPGs and providing HCPs with
  situation-specific, guideline-conformant advice.
  Growing evidence suggests that
  effective CDSSs can reduce PMEs by boosting adherence to best practice guidelines.

  This work presents a semantics-first approach to implementing safe clinical
  decision support systems. By semantics-first,
  we mean that
  \begin{enumerate*}[label=\roman*]
    \item semantics of medical knowledge is
  accurately captured in the CDSS, and,
  \item the semantics of the programming language
    used to encode medical knowledge is formally defined.
  \end{enumerate*}
  At the core of our approach is \MediK{}: a novel domain specific language
  for expressing best practice guidelines that emphasizes comprehensibility
  to HCPs, enabling them to validate the accuracy of medical knowledge in its
  programs. \MediK{} has a complete, executable formal semantics in the \K{} Framework,
  from which all execution and analysis tools are derived in a
  correct-by-construction manner.

  To evaluate our approach, we collaborated with a major pediatric hospital
  to develop a complex real-world CDSS for the screening and management of
  Sepsis in pediatric cases, and validated that it satisfies desired safety properties.
  We outline how our approach improves upon the existing state-of-art,
  optimizations to address domain-specific needs of healthcare practitioners,
  and discuss challenges for future work.
  %Our CDSS is Institutional Review Board (IRB) approved and is slated to undergo clinical simulations.


  %Despite their benefits, widespread CDSS adoption is hindered by several
  %challenges.

  %Several Domain-Specific Languages (DSLs) have been proposed to facilitate expressing BPGs
  %in a computer-interpretable format that is easily comprehensible to HCPs.
  %Given the safety-critical nature of CDSSs, the need for such languages to have complete
  %formal semantics and an ecosystem of formal analysis tools has been recognized.
  %Moreover, since these languages evolve over time to accommodate complexities in modeling BPGs,
  %tools for them must also be adaptable to changes. But, existing languages lack complete formal semantics,
  %or analysis tools and execution tools derived from them.

  %This work presents a semantics-first approach to building safe CDSSs.
  %By semantics-first, we mean that (i) semantics of medical knowledge is
  %accurately captured in the CDSS, and, (ii)
  %This work introduces MediK: a new DSL for expressing BPGs with a complete executable formal semantics, and formal analysis tools, including a model checker, symbolic execution engine, and deductive verifier. As MediK's tools are derived from its semantics, any update to the language is automatically reflected across all tools. To evaluate our approach, we collaborated with a major pediatric hospital to develop a MediK-based CDSS for the screening and management of Pediatric Sepsis and validated that it satisfies desired safety properties. Our CDSS is Institutional Review Board (IRB) approved and is slated to undergo clinical simulations.

\end{abstract}

%%%%%%%%%%%%%%%%%%%%%%%%%%%%%%%%%%%%%%%%%%%%%%%%%%%%%%%%%%%%%%%%%%%%%%%%%%%%%%%
% DEDICATION (Uncomment this section if desired)
%
\begin{dedication}
To my parents and my sister,
for their love and unconditional support,
and for always believing in me, even when challenges ensued.
\end{dedication}

%%%%%%%%%%%%%%%%%%%%%%%%%%%%%%%%%%%%%%%%%%%%%%%%%%%%%%%%%%%%%%%%%%%%%%%%%%%%%%%
% ACKNOWLEDGMENTS
%
\begin{acknowledgments}

While my journey at the University of Illinois has been long and challenging,
  I have been fortunate to have had the support of several wonderful people.

  First, I would like to thank my advisor, Prof. Grigore Ro\c{s}u,
  for his unwavering support and encouragement ever since I joined his research group as an undergraduate in the summer of 2014.

I am also extremely grateful to Prof. Lui Sha,
  with whom I have worked closely during the latter half of my PhD.
  His guidance, insights, and encouragement have been invaluable.
  I’d also like to thank him for his financial support, which enabled me to further develop my ideas.

I would like to express my sincere thanks to
  Dr. Serdar Tasiran for his mentorship during two summer internships at AWS’s S3 Automated Reasoning Group (S3-ARG),
  where I also had the opportunity to collaborate with Dr. Ankush Desai and Dr. Dongyun Jin,
  from whom I learned a great deal.
  I am also thankful to Serdar for impressing upon me the importance of
  effectively presenting ideas--a skill I have worked hard to improve based on his advice.

I am grateful to Prof. Jose Meseguer for his valuable insights
into implementing industrial-scale systems for use by non-experts in computer science.

I would also like to thank Prof. Rosu, Prof. Meseguer, Prof. Sha,
  and Dr. Tasiran for serving on my doctoral committee and for their time, expertise, and feedback.

I’d like to thank my labmates,
  both current and former, with whom I’ve had the pleasure of working with and learning from.
  Thank you, Daejun, Yi, Owolabi, Lucas, Xiaohong, Mircea, Nishant, Shuang, and Simon.
  A special shout-out goes to Nishant and Shuang for their companionship outside the lab,
  and to old friends Amulya, Neelabh, Rajiv, Dhruv, Balaji, and Aditya for always keeping me in the loop,
  even when I was too busy to respond.

I would like to extend my deepest gratitude to my family,
  both in the U.S. and in India.
I'd like to thank my uncle, aunt, and cousins, who always made sure I had a home away from home.
And to my parents and sister--thank you for the unconditional love, support, and unwavering belief in me, even during my moments of doubt.



\end{acknowledgments}

%%%%%%%%%%%%%%%%%%%%%%%%%%%%%%%%%%%%%%%%%%%%%%%%%%%%%%%%%%%%%%%%%%%%%%%%%%%%%%%




%According to a September 2023 President's Council on Science and Technology Report,
%a quarter of medicare patients suffer adverse outcomes during hospitalization,
%of which 40\% were \emph{preventable}
%Clinical Best Practice Guidelines (BPGs) are systematically developed,
%evidence-based statements published by medical institutions and associations
%that standardize diagnosis and treatment for various clinical scenarios. When
%expressed in an executable medium, BPGs can be utilized to build systems that
%assist healthcare professionals (HCPs) with situation-specific advice. Such
%systems, known as Guideline-based Clinical Decision Support Systems (CDSSs),
%  have been shown to improve patient outcomes.
%
%Several Domain-Specific Languages (DSLs) have been proposed to facilitate expressing BPGs in a computer-interpretable format that is easily comprehensible to HCPs.
%Given the safety-critical nature of CDSSs, the need for such languages to have complete formal semantics and an ecosystem of formal analysis tools has been recognized.
%Moreover, since these languages evolve over time to accommodate complexities in modeling BPGs, tools for them must also be adaptable to changes.
%But, existing languages lack complete formal semantics, or analysis tools derived from them.
%
%This work introduces \MediK{}: a new DSL for expressing BPGs with a complete executable formal semantics,
%and formal analysis tools, including a model checker, symbolic execution engine, and deductive verifier.
%As MediK's tools are derived from its semantics, any update to the language is automatically reflected across all tools. To evaluate our approach, we collaborated with a major pediatric hospital to develop a MediK-based CDSS for the screening and management of Pediatric Sepsis and validated that it satisfies desired safety properties.
%Our CDSS is Institutional Review Board (IRB) approved and is slated to undergo clinical simulations.
%\end{abstract}
%
%{
%    \hypersetup{linkcolor=black}  % disable link coloring locally
%    \tableofcontents
%    % the Graduate College doesn't recommend including lot or lof
%    % \listoftables
%    % \listoffigures
%}

%\chapter{List of Abbreviations}
%
%\begin{abbrevlist}
%\item[AAP]      American Academic of Pediatrics
%\item[AAP]      American Academy of Pediatrics
%\item[ACLS]     Advanced Cardiovascular Life Support
%\item[AED]      Automatic Emergency Defibrillator
%\item[AHA]      American Heart Association
%\item[ALS]      Advanced Life Support
%\item[BLS]      Basic Life Support
%\item[BNF]      Backus-Naur Form
%\item[BPG]      Best Practice Guideline
%\item[CAST]     Commercial Aviation Safety Team
%\item[CDSS]     Clinical Decision Support System
%\item[CPR]      Cardiopulmonary Resuscitation
%\item[CSM]      Communicating State Machines
%\item[CTL]      Computation Tree Logic
%\item[DeGeL]    Digital Electronic Guideline Library
%\item[EHR]      Electronic Health Record
%\item[EVM]      Ethereum Virtual Machine
%\item[GEE]      Guideline Execution Engine
%\item[GEODE-CM] Guided Entry of Data Elements for Clinical Management
%\item[GLARE]    Guideline Acquisition, Representation, and Execution
%\item[GLIF]     Guideline Interchange Format
%\item[GOSpeL]   Guideline prOcess Specification Language
%\item[GPROVE]   Guideline PRocess cOnformance VErification Framework
%\item[GP]       General Practitioner
%\item[HCP]      Healthcare Practitioner
%\item[ITL]      Interval Temporal Logic
%\item[LTL]      Linear Temporal Logic
%\item[MBA]      Model-based Architecture
%\item[MLM]      Medical Logic Module
%\item[NAM]      National Academy of Medicine
%\item[NHS]      National Health Service
%\item[ONC]      Office of the National Coordinator for Health Information Technology
%\item[P-CAPE]   Partners Computerized Algorithm Processor and Editor
%\item[PCAST]    President's Council of Advisors on Science and Technology
%\item[PME]      Preventable Medical Error
%\item[ROSC]     Return of Spontaneous Circulation
%\item[SAGE]     Standards-based Guidelines Environment
%\item[SMT]      Satisfiability Modulo Theories
%\item[SOS]      Structural Operational Semantics
%\end{abbrevlist}
%
%\chapter{List of Symbols}
%
%\begin{symbollist}[0.7in]
%\item[$\tau$] Time taken to drink one cup of coffee.
%\item[$\mu$g] Micrograms (of caffeine, generally).
%\end{symbollist}

\tableofcontents
\mainmatter

\section{Introduction}
Medical Errors characterized by adverse events due to human and system factors
(as opposed to diseases) are the third leading cause of mortality in the United
States. Referred to as Preventable Medical Errors (\PME{}), they are defined as
incorrect intended treatment, or incorrect executions of intended treatment
\cite{DonaldsonBook00}.
To mitigate such errors, hospitals and Medical Associations publish
evidence-based statements that codify recommended best practice for
various scenarios called Best Practice Guidelines (\BPGs{}) \cite{field1990clinical}.
\BPGs{} are routinely updated based on results from clinical trials and advances in medical science,
providing Healthcare Professionals (\HCPs{}) with the latest diagnosis and treatment related information.

\BPGs{} can only be effective at reducing \PMEs{} if \HCPs{} adhere to them.
For example, consider Advanced Cardiac Life Support (\ACLS{}): a \BPG{} published
by the American Heart Association (AHA) for management
of In Hospital Cardiac Arrest (IHCA): a life-threatening condition \cite{AHAGuidelineAdult, AHAGuidelinePed}.
Studies suggest that management
of IHCA in 30\% of adult, and 17\% of pediatric cases deviates from the
AHA-prescribed \BPG, resulting in worse patient outcomes \cite{Ornato2012DeviationAdult,Wolfe2020DeviationPediatric,
Crowley2020DeviationAdult,Honarmand2018Adherence,Mcevoy2014Adherence}.
Thus, ensuring \BPG{}-adherence is vital. However, it has been shown that
having clinicians adhere to \BPGs{} is difficult to achieve in practice
\cite{RandJAMA99,DavisCMAJ97}. To address this, computerized Decision Support Systems that codify \BPGs{} and supports \HCPs{} with
situation-specific advice, called Clinical Guidance Systems (\CGSs{}), can be utilized.
Such systems have demonstrated effectiveness at improving \BPG{} adherence
\cite{KwokEMA09}. Moreover, evidence from multi-center clinical trials
suggests that such systems are effective at reducing \PMEs{} \cite{BenettJAMIA16,SahotaJIS11}.

We briefly provide an overview of existing research on \CGSs{} to explain why
it doesn't addressed problems we intend to address in this proposal.
There exists a large body of research on \CGSs{}. In
\cite{PelegJBI13}, the author provides a methodological review of
existing work on Computer Interpretable Guidelines (\CIGs{}): executable
formalizations of \BPGs{} used to construct \CGSs{}.
Existing work is classified into one of eight themes spanning
the entire development cycle of a \CIG{}. The themes and relations between them
are shown in \figurename{} \ref{fig:cpg-research-topics}.

\begin{wrapfigure}{l}{0.5\textwidth}
  \includegraphics[width=\linewidth]{cpg-topics}
  \caption{\CGSs{} Research Themes}\label{fig:cpg-research-topics}
\end{wrapfigure}

According to the author in \cite{PelegJBI13}, \CIGs{} are usually based on previously published non-executable
\BPGs{}. To develop a \CIG{}, a language is identified in (1). Teams of
software developers and clinicians then collaborate to express medical knowledge
in the \BPG{} in the identified language. In (3), the \CIG{}
is integrated with components such as a Graphical User
Interface (\GUI{}), Electronic Health Records (\EHRs{}) and external devices
(such as monitors for patient parameters) to obtain a \CGS. Before adoption
in the real-world, it is imperative to ensure that the \CIG{} \emph{mirrors}
the underlying \BPG{}. This validation occurs by \emph{testing} the \CGS{}
using execution capabilities of the modeling language from (1) in (5).
Additionally, formal verification may be used to establish other desired
properties hold. Inconsistencies identified in (5) are fixed through developer-clinician
collobaration in (2),  re-validation and
verification. While the aforementioned development cycle has resulted in several
effective \CGSs{}, it has some limitations:

\paragraph{Gap between specification and implementation:}

To develop the \CIG{}, software developers rely on clinicians to interpret the
non-executable \BPG{} and communicate
the intended semantics to them. Thus, the non-executable \BPG{} serves as a functional specification for
the \CIG{}, i.e. the implementation. In such safety-critical systems, it is
imperative that the implementation, i.e. the \CIG{}, conforms to its
specification, i.e., the \BPG{}. To address this, the \CIG{} is tested by
putting the \CGS{} through clinical simulations. But, while testing reduces
the risk of non-conformance, it does not completely eliminate it.

\paragraph{Safe Modularity:}

While developing \CGSs{} is both complex and cost-intensive,
the development effort can be reduced by sharing \CGSs{} across hospitals \cite{PelegAMIA00}.
But, even for the same \BPG{}, hospitals develop their own \CGSs{} to address
their needs, resulting in duplicated work.
For instance, for the \ACLS{} \BPG{}, multiple \CGSs{} have been developed by different
different hospitals in a span of just six years years \cite{FullCodePro,PediAppRREST2020,
PediAppRREST2021,GuidingPad2017,GuidingPad2019, GuidingPad2020,DST2014,DST2019,ROSCo2021,TeamScreen2019,Wu2017}.
\CGSs{} based on the same \BPG{} typically have the same \CIG{}, but may differ
in their Graphical User Interfaces (\GUIs), or integration with external
devices, to address hospital-specific needs.
To enable safe sharing of knowledge, we need a mechanism that:
\begin{enumerate*}[label=(\alph*)]
  \item allows a stable, formally-verfied \CIG{} that is \emph{decoupled} from other
    components, and,
  \item supports hospital-specific customizations without compromising system
    \emph{safety}.
\end{enumerate*}

\paragraph{Holistic System Safety:}

Actions performed by a \CGS{} can either be \emph{programming-oriented}
or \emph{clinically-oriented} \cite{BoxwalaJBI04}. \emph{programming-oriented}
actions are peformed by executing the \CIG{} itself. For example,
using patient parameters, or health records to make a reccomendation or diagnosis,
or to raise a warning. \emph{clinically-oriented} actions on the other hand
are ones that involve a clinician. For example, in the case of \ACLS{},
the \CGS{} recommends that Cardiopulmonary Resuscitation (\CPR{}) be performed
for a certain length of time. Such actions can only be performed by clinicians,
an the \CGS{} assumes that the recommended action was indeed performed before
moving resuming guidance.

For holistic system safety, both categories of actions must be completed
successfully. While traditional formal reasoning techniques can be employed
to establish correctness of \emph{programming-oriented} actions, the same
cannot be used to reason about \emph{clinically-oriented} ones.
Thus, a mechanism that allows some guarantees about clinically-oriented is
desirable.


\paragraph{Formal Semantics:}

Since \CIGs{} are safety-critical, it is vital that the
language in which they're developed has complete formal semantics that
can be used to derive tools such as semantically-correct interpreters,
deductive program verifiers, and model-checkers.

\subsection{Limitations of Existing Approaches}

While existing approaches have been imperative to increasing \CGS{} adoption, to the
best of our knowledge, none of them address all of the aforementioned
limitations. We briefly describe notable approaches, their successes, and their limitations.

The Arden Syntax \cite{HripcsakCBM94} a widely used medium for
expressing \CIGs{}.  Guidelines as described using Medical
Logic Modules that contains information related to guideline's purpose
, maintainance, and medical knowledge. The modules are modular to allow
re-use and sharing across hospitals. But, Arden Syntax
is focused on describing simple, modular, and independent
guidelines (such as reminders), and not on guidelines with complex logic (such
as treatment protocols) \cite{PelegJBI01}.
Arden Syntax's limitation in modeling complexity is addressed by
GLIF \cite{BoxwalaJBI04}: a language that uses flowcharts to expressed
guidelines. A multi-level approach is
employed to manage complexity: at the top is the conceptual level, where
only high-level details relevant for human-comprehension are present. In the
middle is a computable-level, where details of guideline execution flow
and patient data elements are specified. At the bottom is the implementable
level, where institution-specific details and mappings into patient data are
specified. Both Arden Syntax and GLIF  eliminate
the gap between the \BPG{}, i.e. the specification, and the \CIG{}, i.e. implementation as
they're meant to be either directly used by clinicians (or in collaboration with
computer scientists) to express \BPGs{} in an executable medium. \CIGs{}
expressed in them are meant to be shared across hospitals, and are thus modular.
However, neither formalism has complete formal semantics, or comprehensiev support for
rigorous formal analysis.

The need for formal analysis is identified by Asbru: a formalism with formally
defined syntax and semantics \cite{ShaharAMIA96}. In Asbru, a guideline is modeled as a plan
that contains:
\begin{enumerate*}[label=(\roman*)]
  \item intentions that define aims,
  \item conditions that specify when the plan is applicable,
  \item effects that define expected behavior during execution, and,
  \item a body containing other subplans.
\end{enumerate*}
Apart from an execution engine, the Asbru ecosystem also contains
other tools, such as a model checker for verification \cite{BaumlerSPIN06}.
However, the formal semantics of Asbru have been only partially defined, and
is insufficient to implement tools for the language \cite{SuttonAMIA03}.
The importance of a complete formal-semantics is identified and addressed
by PROforma \cite{SuttonAMIA03}, another formalism that uses plans to
model guidelines. A PROforma plan is made of a sequence of tasks.
The plan defines constraints on their enactment, and circumstances
for termination (for example, exceptions) \cite{SuttonAMIA03}. But, despite
having complete formal semantics, it does not have a comprehensive suite of
formal analysis tools such as model checkers, deductive verifiers.


The SAGE guideline model \cite{TuSAGE04} uses the Prot\'eg\'e knowledge
representation framework \cite{NoyAMIA03} to model guidelines,
and improves on aforementioned approaches by
enabling seamless integration into hospitals' existing Clinical Information Systems
(\CISs). But, it lacks complete formal semantics, and analysis tools
such as deductive verifiers and model checkers.
The GLARE formalism \cite{TerenzianiBook04} uses an actions based approach
to represent guidelines, and addresses clnician-comprehensibility and
modularity. For formal analysis, GLARE guidelines can be translated to
Promela: the SPIN model checker's specification language \cite{GiordanoAMIA06}.
The approach partly addresses holistic safety as
external agents (such as clinicians) can be modelled and analyzed.
But, the scenario where the external agent's behavior
deviates from the model during system execution isn't addressed.
Non medical-domain specific languages can also be used to reason about
medical systems. For example, in \cite{ArcainiMEMCODE15}, Abstract State
Machines (\ASMs) are used to validate and verify a system for measuring
patients' stereoacuity in the diagnosis of amyblyopia. But such a
formalism, while suitable for formal verification, may
not be easily comprehensible to clinicians for validation.

In \tablename{} \ref{table:existing-approaches}, we provide an overview of
the strengths and limitations of existing approaches. Note that we use
\greencheck{}, \cancelcheck{}, and \redcross{} to depict that an approach
fully-addresses, partly-addresses, or doesn't address a limitation respectively.
To the best of our knowledge,
none of the aforementioned approaches have:
\begin{itemize}[leftmargin=*]
  \setlength\itemsep{0em}
  \item An interpreter or execution engine with \emph{\underline{correctness guarantees}}.
  \item A Rich Suite of \emph{\underline{formal analysis tools}} such as a deductive program
    verifier, model checker, and symbolic execution engine that can be used to
    reason about the guidelines. Since a \CIG{} can comprise multiple
    processes that are \emph{parallel}, \emph{sequential}, or a mix of both,
    reasoning about them can be challenging. But, existing work in modeling
    and reasoning about distributed systems can provide solutions.
  \item Ability to reason about agents that perform \emph{\underline{external
    agents}}. These include clinicians responsible for \emph{clinically-defined}
    actions, or \emph{monitors} for \emph{patient parameters}. While it may seem
    impossible to reason about systems that depend heavily on actions of
    external agents, solutions to similar problems in other domains, such as
    the Simplex Architecture \cite{BakRTAS09} in Real-Time Systems (\RTSs), can be looked at for directions.
\end{itemize}

\begin{center}
\renewcommand{\arraystretch}{0.5}
%\setlength\extrarowheight{-9pt}
  \begin{table}
  \begin{tabularx}{\textwidth}{
      >{\centering\arraybackslash}X
    || >{\centering\arraybackslash}X
    | >{\centering\arraybackslash}X
    | >{\centering\arraybackslash}X
    | >{\centering\arraybackslash}X
  }
                 & Implementation-Specification Gap & Complete Formal Semantics & Formal Analysis Tools & Holistic Safety  \\
    Arden Syntax & $\greencheck$                               & $\redcross$               & $\redcross$           & $\redcross$ \\
    GLIF         & $\greencheck$                               & $\redcross$               & $\redcross$           & $\redcross$ \\
    Asbru        & $\greencheck$                               & $\cancelcheck$            & $\greencheck$         & $\redcross$ \\
    PROForma     & $\greencheck$                               & $\greencheck$             & $\redcross$           & $\redcross$ \\
    GLARE        & $\greencheck$                               & $\cancelcheck$            & $\cancelcheck$        & $\cancelcheck$ \\
    Promela/SPIN & $\redcross$                                 & $\greencheck$             & $\greencheck$         & $\cancelcheck$ \\
    AMSs         & $\redcross$                                 & $\greencheck$             & $\greencheck$         & $\redcross$ \\
    SAGE         & $\greencheck$                               & $\redcross$               & $\redcross$           & $\redcross$ \\
  \end{tabularx}
  \caption{Comparison of Existing Approaches}\label{table:existing-approaches}
  \end{table}
\end{center}




%Clinical \emph{Best Practice Guidelines (BPGs)} are evidence-based statements
%developed to assist healthcare providers in diagnosis and treatment\cite{field1990clinical}.
%Best practice is identified and periodically updated by professional associations
%based on multi-center clinical trials and advances in medical science.
%In practice, strict compliance to guidelines is difficult to achieve,
%and deviations may occur, especially in acute care. Medical errors are defined as
%incorrect intended action and incorrect execution of intended action
%according to Institute of Medicine (IOM)\cite{ToErrIsHuman2000}.
%Such compliance-related \emph{preventable} medical errors are the
%third-leading cause of mortality in the United States, accounting for more than
%250,000 deaths every year\cite{MakaryBMJ16}.
%For example, Cardiac Arrest is a common life-threatening condition.
%American Heart Association (AHA) publishes Advanced Cardiac Life Support (ACLS)
%guidelines\cite{AHAGuidelineAdult,AHAGuidelinePed} for management of
%In-Hospital Cardiac Arrest (IHCA).
%Studies report that management of IHCA in 30\% of adult, and 17\% of
%pediatric cases contains deviations from AHA-prescribed \BPG{}, associated with
%worse patient outcomes\cite{Ornato2012DeviationAdult,Wolfe2020DeviationPediatric,
%Crowley2020DeviationAdult,Honarmand2018Adherence,Mcevoy2014Adherence}.

%Preventable errors arising from deviations from the \BPG{} can be mitigated using
%software that computerizes the \BPGs{} and supports physicians with
%situation-specific advices.
%This type of Clinical Decision Support Systems, called \emph{Clinical Guidance
%Systems (GCSs)}, have shown effectiveness in reducing preventable errors in
%small scale simulations. However, the current design and development process of
%\CGSs{} lacks standardization and safe customization,
%which constrains usability and wide adoption. Thus, new research effort is needed.

%\subsection{Limitations of Current \CGSs{}}
%\paragraph{Closing the gap between \BPG{} specification and implementation}
%
%  Development of \CGSs{} requires collaboration between clinicians and computer scientists.
%  Clinicians produce interpretations of the paper-based \BPGs{} and communicate
%  to developers, then developers translate them into executable code (\BPGLogic{}).
%  This process is prone to discrepancy between the intended \BPG{} specification
%  and the implemented \BPGLogic{} due to interdisciplinary barriers.
%  Moreover, as safety critical software, the correctness of \CGSs{} is currently
%  only validated through manual inspection during clinical simulations.
%  New technology is needed to standardize the practice of coming up with \BPG{}
%  models, and provide formal verifications for safety. Ideally, the \BPG{} model
%  should be easily comprehensive for both clinicians and developers,
%  and even better, present correct-by-construction properties,
%  where the model is directly executable, i.e., \emph{the model is the implementation}.
%
%
%\paragraph{Modeling errors}
%
%  The essential goal of \CGSs{} is minimizing preventable errors. To maximally
%  achieve this, \CGSs{} should not only provide suggestions for the next correct
%  actions, but also identify errors in intended user action and proactively
%  prevent them. From the model level, this means the \CGSs{} should also model for
%  deviations from guidelines. This can be done collaboratively in two ways.
%  First, clinicians provide a list of common errors based on experience and studies,
%  and developers implement them as part of the \BPGLogic{}. Second, once the \BPG{}
%  is modeled, there should be ways to systematically discover classes of errors.
%  For example, timing error can be formalized as a user action is indicated in an
%  improper system state, and dosing error as user action data out of safety ranges
%  indicated by the \BPGs{}. Techniques and strategies are needed to model and
%  automatically discover errors.
%
%
%\paragraph{Supporting safe customization}
%
%  \CGSs{} consist of \BPGLogic{} and \emph{Graphical User Interface (GUI)}.
%  To achieve maximum effectiveness and usability, developer team iteratively
%  prototype, review, and improve the \GUI{}.
%  During this process, it's vital to ensure that the core-\BPGLogic{} is unaltered.
%  Moreover, there may exist multiple guidance systems with fine-tuned \GUIs{} to
%  address hospital-specific needs for the same \BPG{}. For example,
%  for the ACLS \BPG{}, 7 different guidance systems have been developed by
%  different hospitals in a span of 6 years \cite{FullCodePro,PediAppRREST2020,
%  PediAppRREST2021,GuidingPad2017,GuidingPad2019,
%  GuidingPad2020,DST2014,DST2019,ROSCo2021,TeamScreen2019,Wu2017}.
%  This necessitates the need for a trusted, stable, and formally
%  verified safe and correct \BPGLogic{} software module that can be used with
%  multiple hospital-specific \GUIs{}, as well as techniques to support safe
%  customization for versatile \GUIs{} implementations. Changes made in the
%  \GUIs{} should be guaranteed not to corrupt the behavior of \BPGLogic{}.
%
%
%\subsection{Challenges}
%  Standardized safe approach to model \BPGs{} requires capability to handle
%  complex \BPGLogic{} as well as formal definition and proof for safety properties
%  of such complex models. \BPGs{} involve multiple concurrent workflows.
%  The workflows can be \emph{sequential}, \emph{parallel}, or a mix of both.
%  Safety rules such as mutual exclusion, coupling, branches, and numerical bounds
%  need to be enforced among workflows. It is vital to ensure that the \BPGLogic{}
%  faithfully implements these dependencies and is provable analytically.
%
%  To assure safety, it is essential to constrain the propagation of faults
%  in the system and trace impact of changes. This is challenging when involve
%  heterogeneous devices and interfaces, especially when we want to support
%  versatile \GUIs{}. On one hand, software modules that involve user actions are
%  traditionally more complex to formally verify safety because of the uncertainty
%  from external sources; On the other hand, \GUIs{} are subject to much higher
%  frequency of changes, it is not efficient to re-verify the whole system every
%  time it changes. Reasonable design of software architecture and safety tool
%  chain is needed to efficiently promote overall system safely.


\chapter{Background}\label{chapter:background}

\autoref{chapter:introduction} conceptually introduced
best practice guidelines (\BPGs{}) and
clinical decision support systems (\CDSSs{}).
This chapter introduces relevant background details
on \BPGs{} and \CDSSs{}.
In \autoref{sec:bpg-background}, we utilize a real-world \BPG{}
to explain the motivation behind codifying treatment
in the form of clinical guidelines. We also briefly discuss
common characteristics of such guidelines that enable medical knowledge
to be represented efficiently and accurately.

\BPGs{} are usually published by hospitals,
research institutions and medical associations with the aim to improve quality of care by
\begin{enumerate*}[label=(\alph*)]
  \item reducing medical errors due to preventable causes,
  \item standardizing knowledge from latest evidence-based research, and,
  \item enabling access to aforementioned knowledge at medical establishments
  that lack resources to conduct research.
\end{enumerate*}

While in theory, following \BPGs{} should improve clinical outcomes,
their effectiveness in practice is dictated by whether healthcare practitioners
follow them or not. In \autoref{sec:cdss-background}, we present
challenges that practitioners encounter in following \BPGs{}. We then argue
that non-conformance results in worse patient outcomes.
Next, we show how computerized systems that utilize data from available
heterogeneous sources such as electronic health records and sensors for
patient parameters can improve patient outcomes by addressing challenges
to following \BPGs{} encountered by practitioners.

\section{Clinical Best Practice Guidelines}\label{sec:bpg-background}

Clinical best practice guidelines are evidence-based statements
published by hospital and medical associations that codify recommended
interventions for various clinical scenarios \cite{field1990clinical}.
It has been recognized that, if implemented correctly, \BPGs{} have the potential to:

\begin{enumerate}
  \item Reduce unwarranted variations in clinical practice.
  \item Improve healthcare safety and quality.
  \item Enhance translation of research into practice.
  \item Reduce healthcare costs.
  \item Enable development of performance measures for diseases
    \cite{GuerraInjury23, BusseWHO19}.
\end{enumerate}

Guidelines in the form of statements and recommendations
were first introduced in the 1970s,
and were mostly based on expert opinion
In the 1980s and the early part of 1990,
a significant increase in evidence from proliferation of randomized controlled
trails occured. This co-incided with the introduction
of computers at medical instutions that enabled available evidence
to be quickly accessed to aid decision making \cite{GuyattAMA92,SackettJPH95}.
These developments facilitated a shift towards more rigorous
development of \BPGs{}, where evidence was prioritized over expert opinion
\cite{GuerraInjury23}.

To be able to administer optimum care, practitioners must take into
account evidence from ongoing research to guide treatment. This
can be particularly challenging, as available evidence can evolve rapidly.
\BPGs{} enable findings from latest research to be quickly translate in
practice. Well designed \BPGs{} are usually developed and published by trusted medical
establishments (such as medical associations and research institutions)
that base recommendations on best available evidence \cite{GuerraInjury23}.
As the body of evidence increases over time through research,
\BPGs{} must be updated accordingly to reflect latest findings.

\begin{figure}[t]
  \centering
  \includegraphics[width=0.5\textwidth]{sepsis-screening-osf}
  %\includegraphics[width=0.5\textwidth]{screening-vitals}
  \caption{Pediatric sepsis screening \BPG{}}\label{fig:sepsis-screening}
\end{figure}

To illustrate characteristics of \BPGs{}, we briefly go over a \BPG{}
for managing sepsis in pediatric cases used at OSF St. Francis Medical Center
in Peoria, Illinois -- a major pediatric hospital in the United States. Note
that for brevity, we refer to said hospital simply as OSF in the remainder of
this section.
Sepsis is life-threatening condition caused by the body's extreme response to
an infection \cite{RhodesICM17}, and is
a major cause of morbidity and mortality in children \cite{Eisenberg2021JP}.
Adverse outcomes can, however, be mitigated through timely
identification and prompt treatment with antibiotics and
intravenous (IV) fluids \cite{Weiss2014CCM,Evans2018JAMA}.
\BPGs{} for screening and management of sepsis in pediatric Emergency
Departments (EDs) have shown effectiveness in screening and management of sepsis \cite{Eisenberg2021JP},
leading to their adoption in many pediatric EDs \cite{Balamuth2017EM,Sepanski2014FP}.

In \autoref{fig:sepsis-screening}, we present a simplified version of
the screening section of OSF's sepsis mangement guideline.
In essence, when a patient arrives at the
\ED{} with a fever or an infection, the \HCP{} is supposed to obtain
\begin{enumerate*}[label=(\alph*)]
  \item the patient's age,
  \item any conditions, such as cancer, immunosuppresssion, etc,
    that increase likelihood of sepsis, and
  \item the patient's vital signs, such as heart rate, systolic blood
    pressure, respiratory rate, etc.
\end{enumerate*}

This information is then used to check for abnormalities
in clusters of linked information, called \say{buckets}. For instance, if
the patient's heart rate is abnormal, then \say{bucket 1} is said to
have an abnormal value.
Checking for such abnormalities often involves the use of tables, such as
\autoref{table:vital-signs} that contains normal ranges indexed by
\emph{age}.
%\footnote{For brevity, we omit some age ranges and vital signs from table
%\ref{table:vital-signs}}.
If the patient has at least one abnormal value in every \say{bucket},
then he/she is flagged as potentially septic.

The \BPG{}-recommended treatment for
sepsis involves multiple concurrent workflows, such as
screening for septic shock, fluid resuscitation, and administering antibiotics.
In \autoref{fig:fluid-therapy}, we provide
a version of the fluid resuscitation guideline used
at OSF. Briefly, if the patient is flagged as potentially septic, the guideline suggests
\begin{enumerate*}[label=(\roman*)]
  \item obtaining any fluid-overload risks,
  \item administering normal saline (typically over a period of 15 minutes),
    where the dosage is dictated by risks determined in previous step,
  \item assessing signs of fluid-overload,
  \item evaluating patient responsiveness to normal saline upon completion of
    the administering process, and,
  \item determining whether another fluid bolus should be administered based on
    information from previous steps.
\end{enumerate*}
\begin{figure}[h]
  \centering
  \includegraphics[scale=0.5]{FluidWorkflow-fmcad.pdf}
  \caption{Fluid Resuscitation Guideline}\label{fig:fluid-therapy}
\end{figure}

  \begin{table}
    \centering
    \begin{tabular}{ | c || c | c | c | }
      \hline
      \textbf{Age}            & \textbf{Heart Rate}   & \textbf{Systolic BP} & \textbf{Temp}  \\
      \hline
      $0d - 1m$               & $>205$                & $<60$                & $<36 \text{ or } >38$ \\
      \hline
      $\geq 1m - 3m$          & $>205$                & $<70$                & $<36 \text{ or } >38$ \\
      \hline
      $\geq 3m - 1y$          & $>190$                & $<70$                & $<36 \text{ or } >38.5$ \\
      \hline
      $\dots$                 & $\dots$               & $\dots$              & $\dots$ \\
      \hline
      $\geq 13y$              & $>100$                & $<90$                & $<36 \text{ or } >38.5$ \\
      \hline
    \end{tabular}
    \caption{Vital Signs Chart}\label{table:vital-signs}
  \end{table}

This real-world \BPG{} exhibits characteristics common
across many \BPGs{}. Specifically \BPGs{} typically:
\begin{itemize}
  \item Involve \stress{concurrent} workflows, such as administering drugs,
    monitoring vitals, performing treatment, etc. There may also be
    inter-workflow interactions. For instance, a diagnosis of sepsis during the
    screening may require modifications to an ongoing course antibitiotics.
  \item Often specified in a \stress{flowchart-like}
    notation. See \cite{AHAFlowcharts} and \cite{CancerCareFlowcharts} for other flowchart-based \BPGs{} for management of \emph{cardiac arrest}, and
    screening, risk-reduction, treatment and survivorship in
    cancer care respectively.
  \item Require communication between \stress{heterogeneous agents} such as
     monitors and Electronic Health Records (EHRs).
  \item Often use \stress{tables} indexed by parameters such as age, weight,
    etc to present normal/abnormal ranges for measurements, or recommended dosages for drugs.
\end{itemize}

%Note that the aforementioned characteristics are \emph{not} specific
%to one guideline. According to a review paper on \CIGs{} \cite{ClerqAIM03},
%such \DSLs{} should additionally
%\begin{enumerate*}[label=(\alph*)]
%  \item be formally defined, i.e, have a formal syntax and semantics, and
%  \item have an execution engine to provide decision support.
%\end{enumerate*}

\section{Clinical Decision Support Systems}\label{sec:cdss-background}

%\BPGs{} are developed with the intention of improving patient outcomes
%by reducing preventable medical errors.
%But, such guidelines can only be
%effective if they are followed in practice, which can be challenging.
% For instance, consider the Advanced Cardiac Life Support (ACLS):
%a \BPG{} published by the American Heart Association (AHA) for management
%of a life-threatening condition called in-hospital cardiac arrest
%\cite{AHAGuidelineAdult, AHAGuidelinePed}. Studies suggest that management
%of IHCA in 30\% of adult, and 17\% of pediatric cases deviates from the
%AHA-prescribed \BPG{}, resulting in worse patient outcomes \cite{Ornato2012DeviationAdult,Wolfe2020DeviationPediatric,
%Crowley2020DeviationAdult,Honarmand2018Adherence,Mcevoy2014Adherence}.

\BPGs{} are developed with the intention of improving patient outcomes
by reducing preventable medical errors, and their
have resulted in them being widely adopted
in daily practice \cite{WoolfBMJ99}. However, \BPG{} adoption in
clinical settings has several challenges. \BPGs{} are complex documents,
which may also contain some vagueness.
Exact meaning of terms is not always defined and recommended
actions may not be clearly articulated. The long and cumbersome
nature of \BPGs{} may make them difficult to effectively apply
in regular practice. Additionally, such \BPGs{} need to be
periodically updated to take into account latest evidence, and
adapted according to local needs of the setting they're utilized
in \cite{DeClerqSHTI08}.

To mitigate aforementioned issues, systems that utilize available
patient data to assess patient state and issue guideline-based
decision support to practiioners can be utilized \cite{DeClerqSHTI08}.
Such computerized clinical decision support systems (\CDSSs{}) codify
medical knowledge in \BPGs{} and provide practitioners with
situation-specific advice that \say{guides} them towards adherence
to the underlying \BPG{}.
Well implemented \CDSSs{} have been shown to improve
\BPG{}-adherence \cite{GargJAMA06,KawamotoBMJ05}, and reduce
\PMEs{} based on evidence from multi-center clinical trials \cite{BenettJAMIA16,SahotaJIS11}.

\subsection{\CDSS{} Example}\label{sec:cdss-example}

In \autoref{sec:bpg-background}, we presented a real-world
\BPG{} for screening and management of sepsis in pediatric cases.
To illustrate how a \CDSSs{} can enable better \BPG{} adherence,
we utilize an example \CDSS{} that codifies the sepsis management \BPG{}
to administer decision support.

\begin{figure}[t]
  \centering
  \includegraphics[width=0.9\textwidth]{sepsis-tool-main}
  \caption{Sepsis Tool Main Screen}\label{fig:sepsis-tool-main}
\end{figure}

\autoref{fig:sepsis-tool-main} and \autoref{fig:sepsis-graphs-view} show an overview
of a \CDSS{} for sepsis management.
The \CDSS{} is intended to work as tablet at
the patient's beside in a hospital's emergency department (ED).
\autoref{fig:sepsis-tool-main} depicts main screen that the practitioner sees on the tablet.
The main screen consists of four panes (labeled and color coded for illustration
purposes). Pane 1 shows a grid with the patient's vitals pulled
from a combination of the patient's records, sensor measurements and
assessments made by practitioner through prompts that appear on the screen.
Red and green depict abnormality and normality respectively, while gray
indicates missing measurements, which are assumed to be normal. As more
measurements become available, corresponding boxes in the grid are updated
to reflect their status. The two boxes in the top left corner depict whether
the patient is suspected to have sepsis, and be in septic shock. The \CDSS{}
continually monitors these vitals to determine whether they satisfy the criteria
for sepsis or septic shock specified in the hospitals guideline, where the
guideline itself is a $\MediK{}$ program that the practitioners can view and
comprehend. Pane 2 shows the tasks the practitioner is expected to perform
once sepsis is detected. As these tasks are supposed to be performed within an
hour (referred to as golden hour in sepsis treatment), the \CDSS{} tracks time
since the first task is started by clicking the corresponding check-mark.
For some complex tasks, such as fluid resuscitation and antibiotics
administration, an auxiliary window guides the practitioner through the task. The
example scenario shows a patient with concerning vitals despite having
given multiple fluid boluses, indicating septic shock -- a condition that
demands immediate attention. Thus, as specified in the guidelines, the system
recommends urgently administering antibiotics alongside fluids, by highlighting
the relevant tasks in yellow.
Pane 3 contains medications grouped by type, where clicking the
\say{GIVE} button corresponds to administering the drug.
Pane 4 leads to toggleable windows with
important information. For instance, clicking on \say{HAEMODYNAMICS LINE GRAPHS}
toggles the window in \autoref{fig:sepsis-graphs-view} depicting
the patient's vitals over time punctuated by points at which
drugs were administered.


\begin{figure}[t]
  \centering
  \includegraphics[width=0.35\textwidth]{sepsis-graphs}
  \subcaption{Patient Vitals vs Time}\label{fig:sepsis-graphs-view}
\end{figure}

%\begin{figure*}[t!]
%  \begin{subfigure}[t]{0.69\textwidth}
%    \centering
%    \includegraphics[height=2.5in]{sepsis-tool-main}
%    \subcaption{Main Screen}\label{fig:sepsis-tool-main}
%  \end{subfigure}
%  \begin{subfigure}[t]{0.3\textwidth}
%    \centering
%    \includegraphics[height=2.5in]{sepsis-graphs}
%    \subcaption{Patient Vitals vs Time}\label{fig:sepsis-graphs-view}
%  \end{subfigure}
%  \caption{$\MediK{}$-based Sepsis Management \CDSS{}}\label{fig:sepsis-tool}
%\end{figure*}

\subsection{\CDSS{} Components}\label{sec:cdss-components}

\CDSSs{} were first introduced in the 1970s. Initial systems
didn't integrate well with existing patient care workflows, and
didn't find greater adoption outside academia. However,
wider adoption of electronic health records and digital systems for
patient parameters enabled better integration of \CDSSs{} with
workflows in everyday practice, enabling wider adoption.
Modern \CDSS{} can deliver recommendations
through a diverse spectrum of devices such
as systems at the patient bedsides, desktops or tablets carried by
practitioners and smartphones. This versatility
has further enhanced \CDSSs{} adoption \cite{SuttonNature20}.

In \autoref{fig:cdss-architecture}, we provide an overview
of the components of a typical \CDSS{}, and the underlying architecture.
A typical guidelines-based \CDSS{} conceptually consists of the
following components \cite{SuttonNature20}:

\begin{figure}[h]
  \centering
  \includegraphics[width=0.75\textwidth]{cdss-architecture}
  \caption{\CDSS{} Components}\label{fig:cdss-architecture}
\end{figure}


\paragraph{Knowledge Base:}

The knowledge base is the encoding of the underlying guideline in a
computable medium. Recall from \autoref{sec:bpg-background}
that \BPGs{} are typically developed experts in medicine and published
as textual documents. To develop a \CDSS{}, experts in medicine
collaborate with computer scientists or software developers to
develop requirements documentations that present the \BPG{}'s
semantics in a manner amenable to software development.
This is subsequently utilized by software developers to encode
knowledge from the textual \BPG{} into some programming language \cite{PelegJBI13}.
For instance, the \CDSS{} for sepsis management from \autoref{sec:cdss-example}
utilizes the textual \BPG{} shown in \autoref{fig:sepsis-screening}
and \autoref{fig:fluid-therapy} to diagnose and manage sepsis.
Thus, medical knowledge in the textual \BPG{} must translated into some
programming language to enable its use in a functioning \CDSS{}.

\paragraph{Clinical Data:}

\CDSSs{} utilize clinical to assess patient state and
generate recommendations as specified in the underlying \BPG{}. Typically this involves
utilizing available data from various sources such as electronic health records,
devices that monitor various patient vitals, and inputs from \HCPs{}.

Consider the sepsis screening \BPG{} in \autoref{fig:sepsis-screening}.
In order to screen a patient for sepsis, data and
measurements such as the patient's age, associated high-risk conditions,
mental status, heart rate, etc. is required. The origins
of the data is generally very diverse. Information such
as the patient's age, weight and associated high risk conditions
remain static over the duration of sepsis management, and can be
obtained from the patient's health records.
On the other hand, information such as the patient's mental
state or the condition of the patient's skin necessitates an assessment from an
\HCP{} at the patient's bedside before it's manually entered into the system.
Finally, for patient parameters such blood pressure or mean arterial pressure
that change dynamically during the course of treatment, sensors or monitors
that provide them in real-time should ideally be utilized.
Typically, \CDSS{} must take said diversity in data sources into account
to be effective.

\paragraph{User Interface (\UI{}):}

The \UI{} is the interface the \HCPs{} utilize to interact with the system.
The UI typically:
\begin{itemize}
  \item Delivers guideline-specified information (recommendations, alerts, etc.).
  \item Facilitates input of necessary data from the practitioner.
  \item Displays relevant patient information (vital signs, treatment progress).
\end{itemize}

The \UIs{} need to account for the diversity in devices used in modern
healthcare. Interaction can occur through desktops, tablets, or even cell
phones. The \UI{} of the our running example \CDSS{} for managing \CDSS{}
is meant to be displayed on tablet at the patient's bedside, and is shown in
\autoref{fig:sepsis-tool-main}.
The displayed information is organized into distinct \say{panes}:
Pane 1 for patient state, Pane 2 for treatment progress,
Pane 3 for medication details, and Pane 4 for textual guideline references.
Time-sensitive and critical information appears as overlaid popups/prompts.

\paragraph{Additional Infrastructure}

Finally, additional infrastructure is required to glue all aforementioned
components into a functioning \CDSS{}. Said infrastructure typically consists
of software to:
\begin{itemize}
  \item Integrate with a healthcare establishment's existing Information Technology
    stack for access to electronic health records, and systems for services such
    as drug order and delivery, \HCP{} communication, etc.
  \item Interface with various devices, including sensors and monitors,
    that track patient parameters like heart rate, blood pressure, etc.
  \item Control treatment-related devices such as infusion pumps, ventilators, etc.
\end{itemize}
For instance, the example \CDSS{} for sepsis management shown in
\autoref{fig:sepsis-tool-main} integrates requires additional software for integration with
electronic health, several monitors for patient parameters
such as heart rate, blood pressure, mean arterial pressure, etc. and the
hospital's drug order and delivery systems to be effective.


\chapter{Hurdles to \CDSS{} Adoption}\label{chapter:hurdles-cdss-adoption}

There is now increasing evidence to suggest that
well implemented \CDSSs{} can significantly improve quality of care
\cite{GargJAMA05,WellsEJPC08}. However, despite several advantages,
several challenges continue to inhibit wider \CDSS{} adoption \cite{Nam17}.
Some challenges are non-technical, i.e., require changes to legislation,
incentive mechanisms and practitioner education and training, and are beyond the
scope of this work. But, several limitations in existing \CDSS{} technology have also
inhibited further adoption. This chapter discusses said challenges, and
the progress made by existing state of art towards addressing them.

Recall, from section \ref{sec:hurdles-cdss-adoption}, that in 2017,
the National Academy of Medicine published a report on \CDSSs{} that
laid out a roadmap to optmize \CDSS{} uptake in medicine. According to the
report, several challenges need to be tackled for wider adoption, such as:


\begin{enumerate}[label=C\arabic*.]
\itemsep0.0em
\item Absence of systematic ways of \emph{validating content}
in a \emph{reliable}, \emph{accessible} and \emph{updateable} manner.
\item Lack of \emph{reliable}, \emph{shareable} \CDSS{} content
that can be easily adopted across healthcare organizations and their (Information
Technology) \IT{} systems.
\item Technical difficulties of sharing due to \emph{need for
  adaptation} to diverse Electronic Health Records (\EHR) systems.
\item \emph{Suboptimal} User Interfaces (\UIs), implementation choices and
workflows.
\end{enumerate}

\section{Addressing Adoption Hurdles}

\CDSS{} first appeared in the 1960s, and have evolved over time
to address aforementioned challenges. The following sections
describe progress made towards addressing aforementioned challnges.

\subsection{Monolithic \CDSSs{}}\label{sec:monolithic-cdss}

Early \CDSSs{} were developed as monolithic standalone systems
that were self-contained, requiring direct user input for clinical data
\cite{RodriguezBook16}. Such systems co-existed with
primitive electronic health records (\EHR{}) systems,
and thus had to rely on manual data entry before administering support.

Several successful \CDSSs{} implementations utilized a standalone
architecture. Early \CDSS{} implementations
such as MYCIN \cite{ShortliffeBook12} required the \HCP{}
to answer a set of questions to provide advice regarding microbial therapy.
Other early \CDSSs{} such as DXplain \cite{BarnettJAMA87} utilized a wide
range of findings (history, data, etc.) to come up with a diagnosis, and
is still in active development.

The reliance on manual entry made using such systems time-consuming. As support for \EHR{} matured,
\CDSSs{} implementations became better integrated with \EHR{} systems
for automated clinical data retrieval.
However, with monolithic systems, the integration was usually \EHR{}-specific \cite{RodriguezBook16}.
Migrating or sharing \CDSS{} content across medical establishments
presented significant challenges as a system designed
for an establishment's \EHR{} system couldn't be used with a different
establishment's \EHR{} system \cite{KawamotoJBI10}.

\subsection{Modular \CDSS{} Architectures}\label{sec:modular-architectures}

In section \ref{sec:cdss-components}, we presented the components that
every guidelines-based clinical decision support can conceptually be decomposed
into. Early \CDSSs{} from section \ref{sec:monolithic-cdss}
were not designed with modularity that enabled sharing components
between different implementations. As the need for scaling \CDSSs{}
across institutions grew, component-based architectures that
enabled \EHR{} agnostic systems to be developed became prevalent \cite{KawamotoJBI10}.

\EHR{}-agnostic architectures represent a significant step towards
addressing several challenges. Such architectures
address C3 as the knowledge-base can be shared across institutions with
different \EHR{} systems. C2 is partially addressed as the
knowledge-base can be independently developed, maintained and distributed.
C4 is also partially addressed as decoupled components, such as the \UI{},
are easier to adapt to \HCP{} preferences.

Over the years, several \CDSS{} implementations have utilized a
components-based architecture. For instance, in \cite{KawamotoJBI10}, the
authors utilize the service-oriented architecture \cite{ErlBook05} to build
a \CDSS{} web service that can be utilized in a completely \EHR{}-agnostic way.
Recent efforts include \CDSSs{} platforms such as
EvidencePoint that enable \CDSS{} to be integrated
closely with the hospital's \EHR{} without being tightly coupled \cite{SolomonJMIR23}.
\EHR{}-agnostic architectures allow decision support to be administered using the
\EHR{}'s \UI{}. Given their prevalence in modern medical establishments,
\EHR{} systems have become integrated into workflows,
and HCPs are accustomed to using them. Dispensing clinical decision
support through the \EHR{}'s \UI{} is vital for adoption, as better workflow
integration can lead to higher adoption \cite{PressJMIR16,LiJMI16}.

Approaches that utilize a component-based architecture
enable medical knowledge to be shared more efficiently.
But, it's possible for medical knowledge itself to be incorrectly
encoded. \BPGs{} are generally expressed as long textual documents meant to
be understood by \HCPs{} \cite{SchiffmanYMI13}. To build a \CDSS{}, the \BPG{} has to be
systematically expressed in a computable medium. This translation process,
referred to as knowledge formalization, is generally ad-hoc, and can be
the source of inconsistencies in encoded medical knowledge
resulting in \CDSSs{} that render wrong advice \cite{ShaharIOS04}.

Typically, in order to translate textual guidelines to a computable
medium, experts in medicine collaborate with computer scientists and
software developers to come up with a requirements document.
This is subsequently utilized to develop the knowledge base, i.e.,
the computer interpretable encoding of the \BPG{} \cite{PelegJBI13} in a
traditional programming language. Thus, the \BPG{} as a
functional specification for the knowledge base.
For instance, in the case of the aforementioned EvidencePoint,
the knowledge base is expressed in  Javascript \cite{SolomonJMIR23}.

However, since computer scientists/software
developers don't understand medicine, and experts in medicine typically
don't understand programming languages,
it's possible for the functional specification, i.e., the \BPG{}, to
diverge from its implementation, i.e., the knowledge base. Thus,
C1 from section \ref{chapter:hurdles-cdss-adoption} remains unaddressed.

\subsection{Computer-Interpretable \BPGs{}}

\CDSSs{} are safety-critical systems, where bugs can have serious consequences.
Thus, ensuring correctness of \CDSSs{} is vital to widespread adoption.
To ensure bugs arising out of divergences between the textual \BPG{}
and its computable translation, i.e., the knowledge base, the gap between
the \BPG{} and the knowledge base must be eliminated. To this end, instead of
encoding \BPGs{} in a conventional programming language, domain-specific
languages (\DSLs{}) for directly expressing \BPGs{} in a computer-interpretable manner can
be utilized. By emphasizing comprehensibility to \HCPs{}, these \DSLs{} enable
\HCPs{} to validate the accuracy of encoded medical knowledge.
A computer-interpretable \BPGs{} can both as the specification, i.e. the textual \BPG{},
and the implementation, i.e., the knowledge base, thereby eliminating the
specification-implementation gap.

In 1989, the Arden Syntax was developed as a result of incompatibility
of medical knowledge between medical institutions \cite{HripcsakCBM94}.
While it was designed to represent simple guidelines,
such as those related to reminders, more complex treatment protocols couldn't
be represented. Arden syntax had a formal Backus-Naur Form (\BNF{})
syntax definition, but no formal semantics, or a standardized execution engine.
Instead, ad-hoc execution engines for the language have been developed over
time \cite{ClerqAIM03}.

Shortcomings regarding ability of the Arden Syntax to represent complex
treatment protocols were addressed by formalisms such as GLIF \cite{PelegAMIA00}
that permit encoding complex guidelines through the use of a multi-layer
approach to represent both high-level medical knowledge and low-level
implementation details. However, just like the Arden Syntax, GLIF lacks
a formal semantics or formal analysis tools.

The need for formal analysis is identified by Asbru: a formalism with formally
defined syntax and semantics \cite{ShaharAMIA96}. In Asbru, a guideline is modeled as a plan
that contains:
\begin{enumerate*}[label=(\roman*)]
  \item intentions that define aims,
  \item conditions that specify when the plan is applicable,
  \item effects that define expected behavior during execution, and,
  \item a body containing other subplans.
\end{enumerate*}
Apart from an execution engine, the Asbru ecosystem also contains
other tools, such as a model checker for verification \cite{BaumlerSPIN06}.
However, the formal semantics of Asbru have been only partially defined, and
is insufficient to implement tools for the language \cite{SuttonAMIA03}.
The importance of a complete formal-semantics is identified and addressed
by PROforma \cite{SuttonAMIA03}, another formalism that uses plans to
model guidelines. A PROforma plan is made of a sequence of tasks.
The plan defines constraints on their enactment, and circumstances
for termination (for example, exceptions) \cite{SuttonAMIA03}. But, despite
having complete formal semantics, PROforma's semantics is not executable.
Therefore, an interpreter and analysis tools have to be implemented in an
ad-hoc manner.

Over the years, significant progress has been made towards making \CDSSs{}
safer, easier, effective and cheaper. In section \ref{sec:monolithic-cdss},
the earliest \CDSSs{} attempted to increased adherence to evidence-based best practices
at medical establishments. To ensure \CDSSs{} could be scaled better,
components-based architectures, discussed in section
\ref{sec:modular-architectures} enabled medical knowledge to
be developed and maintained independently of other components (such as system \UI{}).
To ensure that medical knowledge is indeed encoded correctly,
several \DSLs{} that eliminate the gap between a textual \BPG{} and
the \CDSS{}' knowledge base have been developed. However, while the need
the following challenges remain unaddressed:
\begin{itemize}
  \item Interpreters and compilers for said \DSLs{} are developed in an
    ad-hoc way, and can be prone to bugs that manifest during execution.
  \item There is a lack of formal analysis tools such as model checkers and
    program verifiers that can be utilized to establish that the
    computer-interpretable \BPGs{} satisfy desired safety and liveness
    properties.
\end{itemize}



\chapter{Related Work}

In chapter \ref{chapter:hurdles-cdss-adoption}, we provided a brief
overview of existing approaches and their limitations. This chapter
provides a comprehensive discussion of related approaches. Recall
from section \ref{sec:modular-architectures} that implementing a guidelines-based
clinical decision support system requires collaboration between
experts in medicine and software engineers for knowledge formalization, i.e.,
the process of encoding medical knowledge in textual
\BPGs{} in some programming language. Using a conventional programming
language for knowledge formalization can lead to an inaccurate
encoding medical knowledge, as experts in medicine, being unaccustomed
to computer programming, cannot validate the accuracy of the encoding.
To address this, \DSLs{} designed specifically for expressing
\BPGs{} in a computer-interpretable format are utilized. \BPGs{} expressed
in such languages can serve as both textual guideline documents and knowledge
bases in \CDSSs{}. But, medical knowledge in \BPGs{} has also been expressed
and formalized using other non domain-specific approaches. The discussion in
this chapter has also been split along the same lines. In section
\ref{sec:dsl-based-approaches}, we discuss approaches involving \DSLs{} for
computer interpretable guidelines. In \ref{sec:general-approaches}, we
discuss other approaches that aren't specific to \BPGs{}.

\section{\DSL{}-based Approaches}\label{sec:dsl-based-approaches}

\subsection{Arden Syntax}\label{sec:arden-syntax}

The Arden Syntax is among the earliest and most widely-used
standards for expressing medical logic, with the first
draft of the standard appearing in 1989.
It was also among the earliest attempts to create a domain
specific language specifically for use in \CDSSs{} \cite{SamwaldJBI12}.

In Arden Syntax, code is organized into self-contained medical
logic modules (\MLMs{}) that have a well-defined structure to
separate higher-level medical logic from low-level implementation
details such as variable declarations. The language continues to
evolve to accommodate diverse uses, and is supported by multiple execution
engines \cite{AnandMed04,KaradimasAMIA02}.

The maturity of the Arden Syntax platform is reflected in its
use to implement a large and diverse set of \CDSSs{}. These
include:
\begin{itemize}
  \item Systems for monitoring and infections surveillance \cite{BlackyACI12,SteinbrecherDC02}.
  \item Treatment planning and decision support \cite{EngeleHealth11,BoeglAMIA05}.
\end{itemize}

Being one of the earliest \DSLs{} specifically designed for \CDSSs{}, the
Arden Syntax has found widespread adoption. But, the language also has several
limitations. Notably:
\begin{itemize}
  \item Support on simple, independent guidelines instead of complex treatment
  workflows \cite{ClerqAIM03}.
  \item Lack of formal semantics and clarity in the language specifications \cite{SamwaldJBI12}.
\end{itemize}

\subsection{\GEODECM{}}\label{sec:geodecm}

Guided Entry of Data Elements for Clinical Management (\GEODECM{}) was
an early \DSL{} developed by the Decision Systems Group at Harvard University
Medical School. The \GEODECM{} \DSL{} utilized a
state machine-based architecture to represent medical knowledge.
Clinical problems in \GEODECM{} were broken down into clinical management
states, where entry, exit and transitions between states were determined
by data collected during execution \cite{StoufletJAMIA96}. Each
clinical management state was represented as state machine nodes,
and edges between nodes represented transitions between the various clinical
mangement states \cite{MachadoJAMIA98}.

\subsection{EON}\label{sec:eon}

The EON langauge, developed at Stanford University, also
utilizes a state machine-based architecture for representing medical guidelines.
Treatment protocols are first recursively composed into smaller granular
elementary protocols that cannot be decomposed any further.
The protocols are then represented via directed multigraphs, where
nodes represent both patient and treatment state, and transitions represent
actions or changes in patient conditions \cite{TuAMIA96}.

EON{} has many similarities to \GEODECM{}. Medical knowledge in both \DSLs{}
is expressed using a finite-state machine notation, where nodes are
patient or treatment states and edges represent actions or changes in state.
However, EON, unlike \GEODECM{}:
\begin{itemize}
  \item Has an informal, yet comprehensive operational semantics.
  \item Support for sequencing, looping and synchronization constructs to support
    guidelines with multiple concurrent actions \cite{TuAMIA96}.
\end{itemize}

While Arden Syntax emphasizes expressing medical logic using independent, modular
medical logical modules, EON state machines are more tightly coupled, enabling
representation of complex protocols and guidelines \cite{TuAMIA96}. EON has
been used to implement complex \CDSSs{} for management of conditions such as
AIDS \cite{MusenJAMIA96} and Breast Cancer \cite{TuAMIA96}.

\subsection{\GLIF{}}\label{sec:glif}

The Guideline Interchange Format (\GLIF{}) was introduced in 1998 as a
result of a collaboration between researchers from Columbia University,
Harvard University and Stanford \cite{ClerqAIM03}. Good medical
guidelines improve healthcare quality, but require substantial work
to develop and maintain. At the time, guidelines were published through
unstructured text documents that were not easily shareable. The motivation
behind the \GLIF{} project was to enable \BPG{} between institutions sharing
through the development of:
\begin{itemize}
  \item A standardized electronic format for rapid dissemination.
  \item A repository of guidelines to prevent duplicated effort.
  \item Tools that \HCPs{} could utilize to retrieve and
    execute relevant guidelines.
  \item Analysis tools that enabled authors to develop
    and publish high-quality, unambiguous guidelines.
\end{itemize}

The \GLIF{} team comprised of research groups from various universities
where other languages, including Arden Syntax (section \ref{sec:arden-syntax}),
\GEODECM{} (section \ref{sec:geodecm}) and EON (section \ref{sec:eon}),
were developed. Thus, perspectives and learnings from existing efforts guided
the development of \GLIF{}, including borrowing certain useful constructs
directly from EON \cite{MachadoJAMIA98}

In \GLIF{}, guidelines are expressed using the \GLIF{} class containing
relevant attributes corresponding to treatment information. One such
attribute is an unordered list of all guideline steps. The guideline
specification itself is a directed graph defined over collection of
said steps. A step can be one of the following:
\begin{itemize}
    \item Action steps: Specify clinical-care actions during treatment.
    \item Conditional steps: Determine control flow based on logical statements.
    \item Branch steps: Enable flow to multiple guideline steps concurrently.
    \item Synchronization steps: Converge concurrent flow back to a single
      guideline step \cite{MachadoJAMIA98}.
\end{itemize}
\GLIF{} has been used to implement several \CDSSs{}, including
systems for depression screening and management \cite{ChoiJMI07} and
hyperkalemia patient screening \cite{WangBook04}.

\subsection{PROforma}\label{sec:proforma}

PROforma was initially developed at the Cancer Research UK Advanced Computation
Laboratory, and has been utilized to develop several \CDSSs{} \cite{ClerqAIM03}.
In PROforma, guidelines are modeled as tasks and data items. Tasks are
hierarchically organized into plan, and may be further divided into:
\begin{itemize}
  \item Actions: Procedures that must be performed in an external environment.
  \item Enquiries: Guideline points at which data must be obtained.
  \item Decisions: Points at which a choice determines further control flow
    \cite{SuttonAMIA03}.
\end{itemize}

PROforma guidelines can be visualized as directed graphs, with nodes representing
actions and edges constraints on control flow. Pictorially,
actions are represented as squares, enquiries as diamonds and
decisions as circles. This enables PROforma guidelines' visual representations
to resemble flowcharts and medical algorithms that \HCPs{} are already familiar
with.

A PROforma task has associated properties that determine how it's interpreted.
These include:
\begin{itemize}
  \item Captions and descriptions that improve guideline comprehensibility.
  \item Preconditions that specify conditions for a task's execution to begin.
  \item Scheduling constraints that enable synchronization between concurrent
    tasks.
\end{itemize}
Additionally, plans have termination and abort conditions that specify
successful end of the current plan to facilitate execution of successor
plans.

Using PROforma requires a suite of tools called Tallis \cite{TallisUrl}, written in Java,
that include a composer for creating and viewing guidelines, a tester for
debugging them, and an execution engine for running them \cite{SuttonAMIA03}.

\section{General Approaches}\label{sec:general-approaches}





\chapter{Semantics-First Approach to Clinical Decision Support}

In chapter \ref{chapter:introduction}, we explained that, despite
advances in medicine, mortality and costs associated with preventable
medical errors (\PMEs{}) remain unacceptably high. In chapter
\ref{chapter:background}, we explained how systems that
assist healthcare practitioners (\HCPs{}) with situation-specific
advice based on evidence-based best practice guidelines (\BPGs{}),
called clinical decision (\CDSSs{}) can reduce both mortality
and costs associated with \PMEs{}. But, despite their potential,
the uptake of such systems in practice is hindered by challenges
that were introduced in section \ref{sec:hurdles-cdss-adoption}, and
discussed in depth in chapter \ref{chapter:hurdles-cdss-adoption}.
In brief, the following challenges (Cs) were outlined:
\begin{enumerate}[label=C\arabic*.]
\itemsep0.0em
\item Absence of systematic ways of \emph{validating content}
in a \emph{reliable}, \emph{accessible} and \emph{updateable} manner.
\item Lack of \emph{reliable}, \emph{shareable} \CDSS{} content
that can be easily adopted across healthcare organizations and their (Information
Technology) \IT{} systems.
\item Technical difficulties of sharing due to \emph{need for
  adaptation} to diverse Electronic Health Records (\EHR) systems.
\item \emph{Suboptimal} User Interfaces (\UIs), implementation choices and
workflows.
\end{enumerate}

\begin{figure}[th!]
  \centering
  \includegraphics[width=0.5\textwidth]{pyramid}
  \caption{Existing \DSLs{} for Computer Interpretable Guidelines}\label{fig:existing-work-pyramid}
\end{figure}

Over the years, significant progress has been made towards
addressing these challenges. In chapter \ref{chapter:related-work},
we discussed how existing approaches have attempted to
address said challenges, and their limitations. Specifically,
in section \ref{sec:related-work-discussion}, we outlined major
themes that these approaches adopt to tackle these challenges.
This is further illustrated by the pyramid diagram in \figurename{}
\ref{fig:existing-work-pyramid}, where aforementioned themes are
underlined in the pyramid's various rungs.
As is typical, approaches that appear in higher rungs also
have characteristics of ones below them. For example, while guidelines expressed in
the Arden Syntax eliminate the specification-implementation gap by being
both \HCP{}-comprehensible and interpretable, they cannot be formally analyzed
due to lack of analysis tools in the ecosystem. Asbru-based guidelines
on the other hand not only eliminate the specification-implementation gap, but can also be
formally analyzed using support for KIV-based verification in the Asbru
ecosystem (see section \ref{sec:kiv-verification}).

As is evident in \figurename{} \ref{fig:existing-work-pyramid}, no
existing approach covers the \say{holistic safety} rung of the pyramid.
Recall from section \ref{sec:related-work-discussion} that we say an
approach tackles \say{holistic safety} if,
besides support for analyzing guidelines,
analysis and execution tools also have correctness guarantees.
In this work, we argue that such guarantees are necessary for
trustworthy \CDSSs{}. We attempt to address \say{holistic safety}
systematically by developing a \emph{semantics-first approach} for
building clinical decision support systems. In this context, by semantics-first
we mean that:
\begin{itemize}
  \item The semantics of the programming language for defining said knowledge is
    formally defined, from which execution and analysis tools are derived in a
    correct by construction manner, leading to holistic safety.
  \item The semantics of medical knowledge are expressed accurately.
\end{itemize}
At the core of our approach is a novel domain-specific language for expressing
medical knowledge called $\MediK{}$ (pronounced Medi-Kay). By being comprehensible to domain experts
in medicine, $\MediK{}$-based computer interpretable guidelines can serve
both as a guideline's non-executable \HCP{}-comprehensible description, i.e.,
the specification, and its encoding in a computable medium, i.e., the
implementation, thereby eliminating any specification-implementation gap.

The remainder of this chapter is structured as follows: Section
\ref{sec:semantics-first} briefly describes the semantics-first philosophy.
Next, Section \ref{sec:k-framework} describes $\K$ -- the language semantic
framework that $\MediK{}$'s are expressed in. Finally,
section \ref{sec:semantics-first-pitfalls} describes potential pitfalls
of following the semantics-first philosophy.

\section{Semantics-First Approach}\label{sec:semantics-first}

The semantics-first approach prescribes a systematic way of
developing programming languages. Instead of implementing
tools for a language, such as interpreters, compilers and
model checkers in ad-hoc manner, the approach states that the
first step in developing said tools must be to formally define
the language's semantics. As show in \figurename{} \ref{fig:semantics-first},
once defined, all tools for the language
can then be automatically derived from the semantics. Moreover, since
the tools utilize the semantics, they are, by definition,
correct-by-construction.

While following the semantics-first philosophy might seem like an obvious choice
in language design, its adoption in practice is far from ideal.
Conventional practice in the programming language and formal
methods community is still to develop analysis and execution tools for each
programming language from scratch \cite{ChenSETSS19}, as illustrated
in \figurename{} \ref{fig:conventional-pl-development} from
\cite{ChenSETSS19}. But, this approach has several
disadvantages:
\begin{itemize}
  \item Implementing tools that perform the same function for
    different languages incurs unnecessary development and maintenance cost.
    As shown in figure \ref{fig:conventional-pl-development}, if there are
    $l$ languages, where each has $t$ tools, then a total of $l \times t$
    tools have to be developed and maintained over time.
  \item Tools are often based on informal descriptions of language semantics,
    leaving developers to extrapolate finer details of the language's semantics,
    leading to inconsistencies.
    For instance, in \cite{ParkPLDI15}, it was found
    that ECMAScript 5.1-compliant JavaScript engines
    in mainstream web browsers behaved differently from each other
    for certain complex JavaScript programs.
  \item As newer versions of a language are introduced, each
    tool for the language has to be updated to ensure support for the latest
    version. This again results in duplicated work.
\end{itemize}
\begin{figure}[t!]
  \centering
  \includegraphics[width=0.6\textwidth]{conventional-pl-development}
  \caption{State-of-Art in Programming Language Design}\label{fig:conventional-pl-development}
\end{figure}

\subsection{Why build \CDSSs{} using Semantics-First?}

In section \ref{sec:semantics-first}, we described benefits
of using the semantics-first approach for developing regular programming
language. But, these differences become starker when semantics-first
is compared against the conventional approach shown in \figurename{}
\ref{fig:conventional-pl-development} in context of domain-specific language for
expressing medical guidelines. Specifically, as such a language will be utilized
in safety-critical settings, it is vital that the language:
\begin{itemize}
  \item Has an \emph{unambiguous}, \emph{formal}
    semantics that can serve as a reference for developing tool support for it.
    This is necessary to ensure that
    tools are free of behavioral inconsistencies due to ambiguities in the
    semantics.
  \item Is supported by a rich formal analysis tools that can
    be used to analyze programs.
    Implementing such tools from scratch would require significant effort.
  \item Can evolve quick to incorporate
    \begin{enumerate*}[label=(\roman*)]
      \item lessons from expressing medical guidelines in it, and,
      \item \HCP{} feedback, specifically regarding comprehensibility.
    \end{enumerate*}
    This can be challenging when using the approach shown in \figurename{}
    \ref{fig:conventional-pl-development}, as
    every change to the language's semantics would require corresponding
    changes to all relevant tools, and additional effort to maintain
    different versions, making the development process extremely tedious.
\end{itemize}

\section{The $\K{}$ Framework}\label{sec:k-framework}

In this section, we introduce $\K{}$: a rewrite-based executable semantics
in which programming languages can be defined through configurations and rules
\cite{KframeworkUrl}. Once the semantics of a programming language has been
defined, $\K{}$ automatically generates all tools depicted in \figurename{}
\ref{fig:semantics-first}, such as an interpreter, compiler,
model-checker and deductive verifier for the language. $\K{}$ has been successfully
utilized to formalize semantics of large real-world languages, such as
C \cite{HathhornPLDI15}, Java \cite{BogdanasPOPL15} and
Javascript \cite{ParkPLDI15}, and analyze non-trivial programs
\cite{StefanescuOOPSLA16,ParkFSE18}.

The remainder of this section introduces $\K{}$ in greater detail by
going over the semantics the $\K$ semantics of IMP: a imperative
language whose syntax is inspired by C and Java. This introduction
is imperative to understanding the details of the $\MediK{}$ language
discussed in upcoming chapters that often rely on $\K{}$ notation.

\subsection{Defining Semantics in $\K$}\label{sec:semantics-in-k}

A typical $\K$ definition consists of the following components:
\begin{itemize}
  \item Syntax: Defined in \BNF{}-like notation, and utilized by $\K$
    to generate a parser for the language.
  \item Configuration: Organizes the program execution state
    into units called \emph{cells} that may be nested.
  \item Rules: Operate over configuration segments and define program
    evolution via rewrites.
\end{itemize}

In \autoref{listing:imp-syntax}, we provide the syntax
definition for IMP.

\begin{lstlisting}[float=ht,
  frame=single,
  style=ksty,
  language=k,
  numbers=left,
  numbersep=5pt,
  caption={Syntax Definition in $\K$},
  label={listing:imp-syntax}
]
module IMP-SYNTAX
  imports DOMAINS-SYNTAX

  syntax Exp ::=
      Int
    | Id
    | Exp "+" Exp                   [left, strict]
    | Exp "-" Exp                   [left, strict]
    | "(" Exp ")"                   [bracket]

  syntax Stmt ::=
      Id "=" Exp ";"                [strict(2)]
    | "if" "(" Exp ")" Stmt Stmt    [strict(1)]
    | "while" "(" Exp ")" Stmt
    | "{" Stmt "}"   [bracket]
    | "{" "}"
    > Stmt Stmt                     [left, strict(1)]

  syntax Pgm ::= "int" Ids ";" Stmt
  syntax Ids ::= List{Id,","}
endmodule
\end{lstlisting}



\section{Pitfalls of the Semantics-First Approach}\label{sec:semantics-first-pitfalls}

\chapter{Evaluating \K{}}\label{chapter:evaluating-k}

In \autoref{chapter:semantics-first-cdss}, we introduced the
semantics-first approach to building systems, and how $\K$,
a framework for defining semantics of programming languages and
type systems, enables it. Specifically, \autoref{sec:why-use-semantics-first}
discusses the rationale for using the semantics-first approach over
the traditional state-of-art for programming language design that
leads to tools based on ad-hoc language semantics and duplicated effort.
In \autoref{sec:semantics-first-pitfalls}, we also touched upon
some additional considerations for using $\K$ in the context of
developing a language for developing a computer interpretretable giudeline DSL.
We outlined the following concerns:
\begin{itemize}
 \item Can the $\K$ generated tools be performant enough to serve as
 the \text{sole} tools?
 \item Can the $\K$-based semantics first approach be used to develop
 a semantics from scratch, wihout even an informal description of the language,
  and, comprehesive tests to go long with the same? Can the semantics
  serve as a comprhenesive docuemnt to implement other tools for the language?
\end{itemize}





\chapter{Separating Concerns: Modular and Safe Clinical Decision Support}\label{chapter:separating-concerns}

In \autoref{sec:cdss-components}, we briefly mentioned components that
conceptually make up a \CDSS{}. This chapter describes the motivations
behind decomposing the system as such. First, we describe where
\CDSSs{} may loosely fit within the larger context of control
theory and motivate the need for tailored architectures for \CDSSs{}.
Then we briefly identify some unique challenges of encoding medical knowledge.
Finally, we talk about encoding medical knowledge directly through
rewriting, and associated issues.

\section{Control Theory and \CDSSs{}}
Control theory deals with the design and analysis of \emph{closed-loop}
systems, i.e., systems where the inputs are affected at-least in part by
by outputs. Typically, such systems are characterized by:
\begin{itemize}
  \item The \emph{plant} represents the part of the system \emph{to be
  controlled}.
  \item The \emph{control} or \emph{compensator} is the part of the system
  that provides \emph{satisfactory characteristics} or \emph{regulation}
  \cite{SimrockTR08}.
\end{itemize}

When viewed from the lens of a closed-loop system, the patient behaves
as the plant, and the goal is optimizing patient outcomes
\cite{SongSMC23}\footnote{Joint work with Song et al.}.


\chapter{\K{}-based Computer Interpretable Guidelines}\label{chapter:k-based-guidelines}

In \autoref{chapter:separating-concerns}, we described an
architecture for building safe and modular clinical decision support systems.
We split \CDSSs{} into three separate components:
\begin{enumerate*}[label=(\alph*)]
  \item a frontend that facilitates interaction with \HCPs{},
  \item a backend that serves as a computer-interpretable encoding of the \BPG{}, and,
  \item additional infrastructure that wires components together.
\end{enumerate*}
This, and upcoming chapters, focus on building backends for \CDSSs{}
utilizing the \emph{semantics-first} approach described in
\autoref{chapter:semantics-first-cdss}. By semantics-first, we
mean that:
\begin{enumerate*}[label=(\alph*)]
 \item the semantics of medical knowledge in the \BPG{} is
 accurately captured, and,
 \item the semantics of the language used to describe the
 \BPG{} is formally defined.
\end{enumerate*}

In \autoref{sec:generic-bpg}, we briefly described characteristics
of \BPGs{} that a \CIG{} language must accommodate.
To this end, we attempted to come up with a framework that
can accommodate expressing a large number of diverse \BPGs{}.
Thus, we broke \BPGs{} into smaller statements that we organized
into:
\begin{itemize}
  \item a workflow containing statements that are executed sequentially, and,
  \item workflows within a guideline may be executed concurrently.
\end{itemize}

In this chapter, we describe a methodology to encode real-world \BPGs{}
as $\K$ definitions. Execution in $\K{}$ is inherently concurrent---
if more than one $\K{}$ rule can apply, then $\K$ non-deterministically
chooses one. Thus, we describe a way of systematically encoding
medical knowledge in \BPGs{} as $\K$ definitions. First, in
\autoref{sec:acls}, we introduce a real world \BPG{} that will be
utilized as running-example in the rest of this chapter. Note that
we intentionally choose real-word examples, instead of small toy cases,
to highlight that our philosophy scales to work in the real-world.
Next, \autoref{sec:kacls-cdss} describes $\KACLS{}$, a $\K$-based tool
to assist \HCPs{} conform to \ACLS{} guidelines that attempts to
follow the \emph{semantics-first} approach. Specifically,
we come up with an abstract representation that captures the semantics
of the \BPG{} from \autoref{sec:acls} with enough details to
enable computer-interpretation. Finally, in \autoref{sec:kacls-backend},
we describe a way to embed said abstract representation into a concrete
$\K$ definition for execution.

\section{Advanced Cardiovascular Life Support Guidelines (\ACLS{})}\label{sec:acls}

\begin{figure}[t!]
  \centering
  \includegraphics[width=0.5\textwidth]{acls-algorithm}
  \caption{Advanced Cardiac Life Support Guidelines}\label{fig:acls-algorithm}
\end{figure}

Advanced Cardiovascular Life Support (\ACLS{}) are a set of \BPGs{} periodically
published by the American Heart Association (\AHA{}) for management of
life-threatening cardiac conditions that will cause of have caused cardiac
arrest. The conditions that the guidelines treat range from dangerous arrhythmias, i.e.,
irregularities in heart's rhythm, to cardiac arrest.---a cardiac emergency where
the heart stop pumping \cite{ACLSWikiEntry}.
The guidelines, according to the AHA{}, \say{are based on the most current
and comprehensive review of resuscitation science, systems, protocols, and
education} \cite{ACLSUrl}.

\autoref{fig:acls-algorithm} shows the \AHA{}'s guidelines for advanced
life support (\ACLS{}) for managing cardiac arrest in adults \cite{ACLSGuidelineUrl}.
\AHA{} publishes guidelines
for basic life support (\BLS{}), and pediatric counterparts of both. While
\BLS{} is meant for a first responder to provide treatment with
limited resources, such as an automatic emergency defibrillator (\AED{}),
\ACLS{} is supposed to be followed by teams of
trained professionals with advanced equipment for airway management,
drug delivery, etc.

\subsubsection{Why build \CDSSs{} for \ACLS{}?}

Cardiac arrest treatment is extremely time-critical, and outcomes
become significantly worse with every passing minute. This
gathering relevant data about the patient infeasible. Moreover, as
the seriousness of the condition necessitates multiple, simultaneous treatments
to be administered rapidly, intervention is usually executed following
standardized \ACLS{} algorithms \cite{ACLSWikiEntry}.

Prompt \ACLS{} guidelines-conformant treatment has been
shown to improve patient outcomes \cite{HonarmandResuscitation18}.
Moreover, deviations from the guidelines in in-hospital cases has been
associated with decreased rates of return of spontaneous circulation (\ROSC{}),
survival to discharge, and survival to discharge with favorable neurological
outcomes \cite{CrowleyResuscitation20}. Studies have also found that
inadequate \ACLS{} training can lead to poor retention, resulting
in reduced \ACLS{} conformance \cite{KiddJCN07}. Thus,
a \CDSS{} that can provide situation-specific guidance about
the next steps, especially in cases where \HCP{} training might be inadequate,
can potentially improve compliance, and consequently outcomes.

\subsubsection{A Brief Overview of Advance Life Support Guidelines for Cardiac Arrest:}

As we aren't concerned with intricacies of medical knowledge in the guideline,
we present a brief and simplified description of the guideline-prescribed treatment.
As \ACLS{} is supposed to be performed in settings where necessary equipment is
available, a electrocardiogram (EKG) machine is utilized to
monitor the patient's cardiac rhythm. Certain cardiac arrhythmias, characterized
by cardiac rhythms referred to as \emph{shockable}, must be treated
by delivering a therapeutic electric shock
\cite{DefibrillationWikiEntry}. As shown in \autoref{fig:acls-algorithm},
treatment has several parallel workflows such as:
\begin{itemize}
  \item Periodic monitoring of the cardiac rhythm, and in case of a
    \emph{shockable} rhythm, using a defibrillator.
  \item Continuous cardiopulmonary resuscitation (\CPR{}).
  \item Administration of vital drugs such as
    epinephrine every few minutes.
\end{itemize}
As \autoref{fig:acls-algorithm} suggests, these workflows must be
performed rapidly and periodically. As the duration of the intervention
is typically a few minutes, making optimal use of available time is critical to
outcomes.

\section{The \KACLS{} System}\label{sec:kacls-cdss}

\begin{figure}[t!]
  \centering
  \includegraphics[width=0.85\textwidth]{acls-tool}
  \caption{Snapshot of \KACLS{}}\label{fig:kacls-snapshot}
\end{figure}

\autoref{fig:kacls-snapshot} shows a snapshot of our \CDSS{} for
enabling compliance to the advanced life support guidelines described in
\autoref{sec:acls}. Advanced life support is supposed to be performed by
a team of \HCPs{}, that take on roles such as a leader, \CPR{}-provider,
airway/respiratory specialist, Intravenous access (IV) and drug administration
specialist, pharmacist, defibrillator attendant and members that serve as
backups \cite{ACLSWikiEntry}. The leader is responsible for coordinating
treatment, our tool render support through a handheld tablet held by the leader.
\autoref{fig:kacls-snapshot} show a snapshot of said tablet's main screen.
Decision support, in the form of time-sensitive reminders, warnings when
procedures are stopped prematurely, or exceed stipulated limitations, and
confirmations regarding the patient rhythm's are provided on the screen
through popups and progress bars. For example, when the leader instructs
the team to start CPR, and presses the \say{Start} button under the \say{CPR}
section of the screen. A timer displays the duration and appropriate warnings
about prematurely stopping CPR or exceeding the time for a \CPR{} cycle are
displayed.

In \autoref{sec:modular-cdss-architecture}, we described
conceptual components of \CDSSs{}, and argued in favor
of developing \CDSSs{} using independently developed and maintained codebases
aligned with these components. The \KACLS{} system follows
this philosophy and has:
\begin{enumerate*}[label=(\roman*)]
  \item a simple \emph{frontend} written in Javascript using React \cite{ReactJSUrl} that handles
    user interaction, and,
  \item a $\K$-based HTTP server that encodes the \ACLS{} guideline
    and interacts with the frontend.
\end{enumerate*}
The use of the Javascript based frontend allows our application
to run on any modern web-browser. As shown in figure \ref{fig:kacls-snapshot},
our frontend consists of forms and buttons that correspond to procedures
in the algorithm in figure \ref{fig:acls-algorithm}. For example, the
\say{Start} button in the CPR box results in a two-minute timer corresponding
to the \say{2 minute continuous CPR} procedure in the informal description.
The frontend is simple and doesn't contain any guideline conformance related
logic. Interaction with the frontend simply results in dispatch of
\textit{events} to the backend. For example, when the \say{Start} button is pressed, a
\say{StartCpr} event is sent to the backend.


\subsection{Capturing Execution-specific Details}\label{sec:execution-specific-details}

A computer-interpretable version of a guidelines, unlike its
textual counterpart, may require specification of additional details
for execution. Such details can often be gathered from context in case
of textual guidelines or may be unintentionally missing. This was also
discussed in context of related approaches in \autoref{chapter:related-work},
especially in \autoref{sec:kiv-verification}.

As discussed in \autoref{sec:generic-bpg}, \BPGs{} can be notionally
represented using a collection of workflows. Each workflow has steps
executed sequentially but may be concurrently executed with steps from
other workflows. Consider the \ACLS{} guidelines discussed in
\autoref{sec:acls}. Several procedures, such as administering drugs,
and performing \CPR{} occur concurrently, where each procedure has
a set of tasks that need to be performed sequentially.
There may also be some implicit order across workflows, but we
address that in later chapters.

\subsubsection{Modeling \BPGs{} as State Machines}

In this work, we choose to express medical logic using
concurrently executing state machines with implicit queues for inter-machine communication.
Communicating state machines (\CSMs{}) is a well-understood model of concurrency
\cite{BrandJACM83} and is well-suited for representing medical guidelines in a
computer-interpretable format that is also \HCP{}-friendly.
We shall discuss the motivations behind using \CSMs{} in upcoming
chapters. For now, it suffices to describe how our choice addresses
issues mentioned in \autoref{sec:execution-specific-details}.
Given a guideline represented as a collection of workflows, we
describe each workflow via a state machine. The fact that
\CSMs{} can execute concurrently conveniently captures the
inherent concurrency in workflows.

\subsubsection{\ACLS{} Workflows as State Machines}

\begin{figure}[tb!]
\centering
\begin{subfigure}{\textwidth}
  \centering
  \includegraphics[width=0.55\linewidth]{cpr-machine}
  \caption{CPR Machine}
  \label{fig:cpr-machine}
\end{subfigure}%
\hfill\newline\hfill\newline\hfill
\begin{subfigure}{\textwidth}
  \centering
  \includegraphics[width=0.55\linewidth]{epi-machine}
  \caption{Epinephrine Machine}
  \label{fig:epi-machine}
\end{subfigure}
\caption{Formal Machine Definitions}
\label{fig:machine-defs}
\end{figure}

At the start of \autoref{sec:kacls-cdss}, we mentioned that
the \emph{frontend} of our \CDSS{} is a simple Javascript-based
application to facilitate user interaction through buttons
and popups. We can now elaborate what we mean by \emph{simple}.
Button presses on the frontend correspond to \emph{events}
that trigger transitions in state machines shown in \autoref{fig:machine-defs}.
For instance, pressing the \say{Start} button under
the CPR part of the tool shown in \autoref{fig:kacls-snapshot} results in a
\inlinek{StartCPR} event that triggers the corresponding transition in
the machine in \autoref{fig:cpr-machine}.
Similarly, certain machine states result in events being dispatched
to the frontend that cause popups or messages to be displayed
on associated parts of the screen. Thus, the frontend itself contains
no medical logic.

\subsection{$\KACLS$ backend}\label{sec:kacls-backend}

As described at the start of \autoref{sec:kacls-cdss},
the backend receives events from the frontend, processes them and
dispatches the results back to the frontend.
In \autoref{sec:execution-specific-details}, we described the
process of details required for execution by expressing
the \BPG{} using communicating state machines. For a functioning
backend through, we need to translate the abstract state machines
into an executable medium. We do this by encoding the
state machines in a $\K$ definition. Before we describe the
encoding, we describe some additional challenges that we need
to address in order to use $\K$ as a \CDSS{} backend.

\subsubsection{Additional Challenges}

\paragraph{Asynchronous External Communication:}
Communication with external process in \K{} is facilitated
by built-in I/O support. \K{} has built-in
functions for many operations that map directly to their
POSIX counterparts \cite{KFrameworkIOUrl}. However,
execution in $\K$ is single-threaded. Thus, to make \K{}
behave as an HTTP \cite{HTTPUrl} server that accept messages and respond asynchronously,
we need to write additional C++ initialization code and link it against
\K's LLVM backend \cite{KFrameworkBackendsUrl}.

\paragraph{Handling Timers:}
The \ACLS{} workflows shown in \autoref{fig:acls-algorithm} require
tasks being performed periodically. For instance, the algorithm
calls for continuous CPR administration for two minutes
between checking rhythm and performing defibrillation if needed.
Similarly drugs like epinephrine must be administered every three minutes.
\K{} does not support setting timers that can interrupt the \K{} process
on expiration. Thus, we rely on external \say{timers} that, instead
of interrupting the \K{} process, place a \inlinek{timeout} event
at the head of the \inlinek{<inputBuffer>} cell on expiration, which
trigger corresponding transitions in state machines. For example,
consider the CPR machine in \autoref{fig:cpr-machine}. When
\CPR{} is started through the press of a button on the frontend,
an external two-minute timer is set, and the machine
transition from state \inlinek{idle} to \inlinek{lessThan2Min}.
When the timer expires, the external timer sends a \inlinek{Timeout}
event to the \K{}, makes the \CPR{} machine to transition from
state \inlinek{lessThan2Min} to \inlinek{morethan2Min}.

\subsubsection{$\KACLS{}$ Backend}
Recall from \autoref{sec:semantics-in-k} that in $\K$ computation
is described via rewrite \emph{rules} that operate over a \emph{configuration}
that organizes data in labeled and potentially nested units called cells.
\autoref{lst:initial-configuration} shows the initial configuration for
the definition $\K$ of our state machines-based representation.
Note that unlike a typical $\K{}$ definition, the initial \inlinek{configuration}
does not have a \inlinek{<k>} cell containing a \inlinek{$PGM} variable
replaced by the program \AST{} at runtime, as the $\K$ definition is the
program itself.

\begin{lstlisting}[float=ht,
  frame=single,
  style=ksty,
  language=k,
  numbers=left,
  numbersep=5pt,
  caption={Initial Configuration},
  label={lst:initial-configuration},
  xleftmargin=3ex
]
configuration
    <machines>                                @\label{lstline:machines-cell-start}@
        <machine multiplicity="*" type="Set"> @\label{lstline:machine-cell-start}@
            <id>    .K   </id>
            <state> .K   </state>
            <store> .Map </store>
        </machine>                           @\label{lstline:machine-cell-end}@
    </machines>                              @\label{lstline:machines-cell-end}@
    <inputBuffer>  .JSONs </inputBuffer>     @\label{lstline:input-buffer-cell}@
    <outputBuffer> .JSONs </outputBuffer>    @\label{lstline:output-buffer-cell}@
\end{lstlisting}

Lines \ref{lstline:machines-cell-start}-\ref{lstline:machines-cell-end} declare
a cell \inlinek{<machines>} that will be used to multiple
\inlinek{<machine> ... </machine>} cells, declared between Lines
\ref{lstline:machine-cell-start} and \ref{lstline:machine-cell-end}, where
each cell stores all data related to a particular state machine. The
\inlinek{<id>} and \inlinek{<state>} cells store the machine's name and
active state respectively. The \inlinek{<store>} cell
is a map used to store any additional data. Note the following:
\begin{enumerate}[label=(\roman*)]
  \item The \inlinek{<machine>} cell is declared with an attribute
    \inlinek{multiplicity=*}, signifying that the configuration can contain
    multiple instances of such cells that are added or removed dynamically
    during execution. At the start of execution, the configuration has
    zero such cells.
  \item As the number of machines to represent is known and remains constant
    during execution, one might ask why they are not declared as a part of the
    initial configuration itself. We clarify that we deliberate not to do, due
    to reasons we explain shortly.
\end{enumerate}

Lines \ref{lstline:input-buffer-cell} and \ref{lstline:output-buffer-cell}
declare cells \inlinek{<inputBuffer>} and \inlinek{<outputBuffer>} respectively
that hold comma-separated lists of
JSON objects. As their name suggested, they are used as buffer to communicate with
the frontend. Incoming events from the
frontend are placed at the end of the \inlinek{<inputBuffer>} and outgoing
events intended for the frontend are consumed from head of the \inlinek{<outputBuffer>}.

\subsubsection{Initialization}

Initialization involves populating the configuration with the
initial state of each state machine. \autoref{lst:machine-initialization}
depicts the template of a rule for initializing machine
any machine \inlinekmath{$\mathcal{M}$}, with initial state \inlinek{u}

%For instance, \autoref{lst:-initialization} shows the rule that initializes
%the CPR Machine to start in state \inlinek{idle} corresponding to the
%entry state of the machine in \autoref{fig:cpr-machine}.

\begin{lstlisting}[float=ht,
  mathescape=true,
  frame=single,
  style=ksty,
  language=k,
  numbers=left,
  numbersep=5pt,
  caption={CPR Initialization},
  label={lst:machine-initialization},
  xleftmargin=3ex
  ]
<machines>
  .Bag => ( <machine>                                      @\label{lstline:machine-add-begin}@
              <id> $\mathcal{M}$ </id>
              <state> u </state>
              ...
            </machine> )                                   @\label{lstline:machine-add-end}@
  ...
<machines>
<inputBuffer> $\mathcal{M}$.init, IN => IN </inputBuffer> @\label{lstline:init-inputBuffer}@
\end{lstlisting}

Instead of starting \inlinek{<machines>} cell that is pre-populated with
\inlinek{<machine>} cells, we dynamically initialize
machines once the corresponding frontend component is rendered and ready.
When the frontend component associated with machine $\mathcal{M}$
is successfully loaded, it sends a \inlinekmath{"$\mathcal{M}$.init"} event,
which is added to the head of the input buffer.
For instance,  when the CPR part of the tool from
\autoref{fig:kacls-snapshot} has successfully loaded, it sends a \inlinek{"cprMachine.init"},
which is added to the head of the input buffer, where \inlinek{cprMachine} is
the name of the machine in the encoded \K{} definition, stored under the \inlinek{<id>} cell.
Lines \ref{lstline:machine-add-begin}-\ref{lstline:machine-add-end} of the rule
introduce a new \inlinek{<machine>} cell with name \inlinekmath{$\mathcal{M}$} and initial
state \inlinek{u}. The \inlinek{.Bag} in $\K$, seen on
\autoref{lstline:machine-add-begin}, represents the absence of a cell, which
the rule rewrites to a \inlinek{<machine>} cell, resulting in its
addition to the configuration under the \inlinek{<machines>} cell.
On \autoref{lstline:init-inputBuffer}, the $\K$ variable \inlinek{IN} matches any list of
JSON elements. Thus, \inlinekmath{"$\mathcal{M}$.init" , IN} matches any list with
\inlinekmath{"$\mathcal{M}$.init"} at the head of the list (\inlinek{IN} matches the tail of
the list). When the rule fires, the list is rewritten to
just the tail, resulting in the removal \inlinekmath{"$\mathcal{M}$.init"}
event from the \inlinek{<inputBuffer>} cell. For instance, in case of the CPR
machine, the initialization rule is:

\begin{lstlisting}[float=ht,
  frame=single,
  style=ksty,
  language=k,
  ]
<machines>
  .Bag => ( <machine>
              <id> cprMachine </id>
              <state> idle </state>
              ...
            </machine> )
  ...
<machines>
<inputBuffer> cprMachine.init, IN => IN </inputBuffer>
\end{lstlisting}

For each machine $\mathcal{M}$ in the guideline, the $\K$ definition
contains a corresponding rule of the form shown in
\autoref{lst:machine-initialization} in the definition. Next,
we discuss each machine in detail to explain how we encode
their transition systems in $\K$.

\subsubsection{CPR Machine}

The \textit{CPR Machine} encodes the intended CPR procedure
shown in figure \ref{fig:cpr-machine}. As mentioned earlier, when the user clicks
the \say{Start} button in the CPR pane in figure \ref{fig:kacls-snapshot},
an external two-minute timer is started. If the user decides to stop CPR before
two minutes are up, a message is displayed to the user warning him of
deviation from the intended guidelines. If the user stops CPR after
two minutes, no warning message is displayed.
\autoref{lst:cpr-machine-start} shows the rule that
fires when the \say{Start} button in clicked.
The rule does the following:
\begin{enumerate}[label=(\alph*)]
  \item Consumes the \inlinek{"StartCpr"} event from the \inlinek{<inputBuffer>} in
    \autoref{lstline:read-inputBuffer}.
  \item Move the machine from state \inlinek{idle} to state \inlinek{lessThan2Min}
    in \autoref{lstline:idle-lessThan2Min}.
  \item Sends a \inlinek{startTimer} action to the frontend via the
    \inlinek{<outputBuffer>} (encoded as JSON) in
    Lines \ref{lstline:external-timer-start-begin}-\ref{lstline:external-timer-start-end},
    which results in the start of a two-minute timer in the frontend.
\end{enumerate}
Note how features of $\K$ discussed in \autoref{sec:k-framework}
make the definition \emph{concise} yet \emph{descriptive}. Specifically:
\begin{itemize}
  \item The \emph{configuration abstraction} mechanism enables only parts of the
  configuration that are not used in the rule to be easily ignored.
    The \inlinek{...} (dots) on
    \autoref{lstline:machine-abstraction-dots} contents of cell
    not used in the rule to be safely ignored. Other parts
    of the configuration, such as the \inlinek{<machines>} cell
    and the \inlinek{<outputBuffer>} cell need not be mentioned at all.
  \item Localized rewriting enables the rewrite symbol \inlinek{=>} to
    apply on deeply nested subterms. This enables de-duplication,
    as parts of the term that remain unchanged don't have to be
    specified both on the \LHS{} and \RHS{} of the rewrite. For instance,
    the \inlinek{<store>} cell on \autoref{lstline:store-abstraction-dots}
    contains is a map that holds information about the machine, such as
    execution status. Recall that a map with $n$ key-value pairs is
    depicted in $\K$ as
    \inlinekmath{($k_1$ |- $\ v_1$), ($k_2$ |- $\ v_2$), $\dots\ $, ($k_n$ |- $\
    v_n$)}. Localized rewriting enables us to only rewrite the
    value mapped to key \inlinek{"cprRunning"} from any value
    to \inlinek{true} and ignore other parts of the map using \inlinek{...}
    notation.
\end{itemize}
\begin{lstlisting}[float=t,
  frame=single,
  style=ksty,
  language=k,
  numbers=left,
  numbersep=5pt,
  caption={CPR Machine in \K{}},
  label={lst:cpr-machine-start},
  xleftmargin=3ex
]
rule <machine>
          <id> cprMachine </id>
          <state> idle => lessThan2Min </state>                    @\label{lstline:idle-lessThan2Min}@
          <store> ( "cprRunning" |-> (_ => true) ) ... </store>    @\label{lstline:store-abstraction-dots}@
          ...                                                      @\label{lstline:machine-abstraction-dots}@
     </machine>
     <inputBuffer> "StartCpr" , IN => IN </inputBuffer>            @\label{lstline:read-inputBuffer}@
     <outputBuffer>                                                @\label{lstline:external-timer-start-begin}@
         OUT => OUT +JSONs jsonResponse( cprMachine
                                       | lessThan2Min
                                       | startTimer( .JSONs ) )
     </outputBuffer>                                               @\label{lstline:external-timer-start-end}@
\end{lstlisting}

\subsubsection{Epinephrine Machine}
\begin{lstlisting}[float=h,
  frame=single,
  style=ksty,
  language=k,
  numbers=left,
  numbersep=5pt,
  caption={Epi Machine in $\K$},
  label={lst:epi-machine-rule},
  xleftmargin=3ex
]
rule <machine>
      <id> epiMachine </id>
      <state> givenLessThan3Min
          => earlyOrderWarningNotification
      </state>
      <store> .Map => ("tentativeOrder" |-> ORDERED) ... </store>
    </machine>
    <inputBuffer> { "event" : "orderEpi"                              @\label{lstline:input-buffer-json-read-begin}@
                  , "data"  : [ ORDERED:Int ] } , IN
              => IN
    </inputBuffer>                                                    @\label{lstline:input-buffer-json-read-end}@
    <outputBuffer> OUT                                                @\label{lstline:output-buffer-warning-begin}@
      => OUT +JSONs jsonResponse( epiMachine
                                | earlyOrderWarningNotification
                                | showEarlyOrderWarning( .JSONs ))   @\label{lstline:output-buffer-warning-end}@
    </outputBuffer>
\end{lstlisting}

The \textit{Epinephrine Machine} encapsulates instructions for administering
the drug Epinephrine to the patient. The corresponding machine is shown in figure
\ref{fig:epi-machine}. The user orders the drug using the
\say{Order 1mg} button and can administers it using the \say{Give} button in
Epinephrine pane in figure \ref{fig:kacls-snapshot}. Giving the drug results in the
start of a three-minute timer, during which if a new order is placed, a warning
is issued.

\begin{lstlisting}[float=b!,
  frame=single,
  style=ksty,
  language=k,
  label={lst:general-rewrite},
  caption={Transitions as $\K$-Rules}
]
  <machine>
    <id> @$\mathcal{M}$@ </id>
    <state> @$u$@ => @$v$@ </state>
    ...
    <machine>
  <inputBuffer> @$\sigma$@, IN => IN </inputBuffer>
\end{lstlisting}

\autoref{lst:epi-machine-rule} encodes the transition
$\textssf{givenLessthan3Min} \xrightarrow[]{\textssf{orderEpi}}
\textssf{earlyOrderWarningNotification}$ from \autoref{fig:epi-machine}.
The rule fires when it has been less than three minutes
before the last dose and a fresh order for
Epinephrine is put in (Lines
\ref{lstline:input-buffer-json-read-begin}-\ref{lstline:input-buffer-json-read-end}).
Note that the input event here also has a payload corresponding to the dosage,
depicted by a JSON object with fields \inlinek{"event"} for the event's name
and \inlinek{"data"} for the dosage.
As the guidelines dictate that Epinephrine must be administered
every three minutes, ordering fresh Epinephrine is a deviation from the
guidelines. Thus, an appropriate warning message is presented to the user (Lines
\ref{lstline:output-buffer-warning-begin}-\ref{lstline:output-buffer-warning-end}).

\subsubsection{Encoding Remaining Transitions}

Note that we only show a single rule from the \textit{CPR} and
\textit{Epinephrine} Machines as other rules resemble the ones
shown, and have a one-to-one correspondence to transitions in the
state machines in \autoref{fig:machine-defs}. For instance,
the rule in \autoref{lst:cpr-machine-start} encodes the transition
``$\textssf{idle} \xrightarrow[]{\textssf{StartCpr}} \textssf{lessThan2Min}$'' in the
CPR state machine in \autoref{fig:cpr-machine}. To encode the entire machine,
along with the initialization rule,
we encode every transition $u \xrightarrow[]{\sigma} v$
of machine $\mathcal{M}$ as a rewrite rule of the form shown in
\autoref{lst:general-rewrite}

\begin{lstlisting}[float=b!,
  frame=single,
  style=ksty,
  language=k,
  numbers=left,
  numbersep=5pt,
  caption={Shock Machine in $\K$},
  label={lst:shock-machine},
  xleftmargin=3ex
]
rule <machine>
        <id> shockMachine </id>
        <state>
          shockableRhythm => checkRhythm                         @\label{lstline:shock-state}@
        </state> ...
      </machine>
      <machine>
        <id> cprMachine </id>
        <store> ( "cprRunning" |-> false ) ... </store> ...      @\label{lstline:cpr-false}@
      </machine>
      <machine>
        <id> epiMachine </id>
        <store> ( "epiGiven" |-> true ) ... </store> ...        @\label{lstline:epi-true}@
      </machine>
      <inputBuffer> "administerShock" , IN => IN </inputBuffer>
      <outputBuffer> OUT                                        @\label{lstline:shock-response-begin}@
      => OUT +JSONs jsonResponse( shockMachine
                                | checkRhythm
                                | confirmShock("CPR for
                                                2 Minutes" ) )  @\label{lstline:shock-response-end}@
      </outputBuffer>
\end{lstlisting}

\subsubsection{Shock Machine}

We now describe the \textit{Shock Machine}. Intuitively, this
machine ensures:
\begin{enumerate*}[label=(\alph*)]
  \item shocks are only administered if the rhythm is shockable,
  \item CPR is administered every two minutes, and,
  \item drugs (such as Epinephrine) are periodically administered.
\end{enumerate*}
In case the of deviation appropriate warnings are issued. The \textit{Shock}
machine differs from other machines, as its transitions depend
on the status of both the \textit{CPR} and \textit{Epinephrine} machines.

\autoref{lst:shock-machine} shows a rule from the \textit{Shock} machine.
The rule fires when the rhythm is \textit{shockable}, and the user
wants to administer a shock. Additionally, CPR hasn't been administered
(\autoref{lstline:cpr-false}), but
Epinephrine has been administered (\autoref{lstline:epi-true}).
In such a situation, the user is informed that it is safe to
administer a shock, which should be followed by
two minutes of CPR (Lines
\ref{lstline:shock-response-begin}-\ref{lstline:shock-response-end}).
However, it is not safe to assume that the rhythm would remain
\textit{shockable} for subsequent shocks. Thus, the machine moves from
state \inlinek{shockableRhythm} to \inlinek{checkRhythm}. If the
user tries to administer another shock, a prompt to confirm the rhythm will be
displayed to ensure a shock is not administered in case of an \emph{unshockable}
rhythm.

We draw the user's attention to the \textit{succinctness} and \textit{simplicity}
of the rule. Despite expressing the interaction between three different machines,
$\K$'s configuration abstraction mechanism ensures that only relevant parts of
each machine are mentioned, making the transition \textit{comprehensible}.

\section{Discussion}\label{sec:rewriting-based-guidelines-discussion}

This chapter describes the process of systematically encoding
\BPGs{} expressed using informal flowchart-like notations as communicating
state machines (\CSMs{}). The \CSMs{} can then be encoded in a $\K{}$,
where, for every machine, $\mathcal{M}$, every transition
$u \xrightarrow[]{\sigma} v$
can be encoded as a rewrite rule of the form shown in
\autoref{lst:general-rewrite}. We discussed how
\emph{configuration abstraction}, \emph{localized rewriting} and
the inherent concurrency of rewriting enable the $\K$ encoding of
the guideline to be both \textit{succinct} and \textit{simple}.
But, comprehending such $\K$ based guidelines precludes comprehending
the $\K$ language, which is unreasonable to expect not only from domain
experts in medicine, but also from software developers in general.
The $\K$ language is designed for development and implementation of
programming languages, and can have an associated learning curve, especially
if one is unfamiliar with the theory of defining formal semantics.
Thus, understanding \K{} may require additional effort
for most software developers, who are more comfortable with
popular general purpose programming languages such as Java, C and python.
Moreover, the $\K$ definition can become harder to comprehend as
more complex guidelines are modeled, requiring rules that encode
transitions where multiple steps must be performed and complex
conditions to determine when the transitions can occur.

In upcoming chapters, we address shortcomings of using $\K$ directly
to model guidelines by coming up with a novel \DSL{} specifically
designed to encode medical logic. This language, called $\MediK{}$,
differs from $\K$ in many aspects. Notably $\MediK$:
\begin{itemize}
  \item Uses syntax that borrows from other languages for \CSMs{}, notably P
    \cite{DesaiPLDI13}, to improve comprehensibility and readability, albeit at
    the expense of succinctness.
  \item Provides support for interacting with heterogeneous external devices,
    such as monitors, patient sensors, etc.
\end{itemize}

\chapter{\MediK{}: Towards Safe Guidelines-based \CDSSs{}}\label{chapter:medik-safe-cdss}

In \autoref{chapter:k-based-guidelines}, we discussed the process
of encoding best practice guidelines (\BPGs{}) as \K{} definitions
to describe a systematic way of building \CDSSs{}. Specifically,
we described the process of encoding medical knowledge in guidelines
notionally via workflows that can be expressed abstractly as
concurrently executing finite state machines that communicate
via passing messages. Next, we described the process of
expressing state machines as \K{} definition, where \K{} features
such as configuration abstraction and local rewriting
enable \emph{conciseness}. We then combined the \K{} definition
with a simple javascript-based frontend to develop a \CDSS{}
for assisting healthcare practitioners (\HCPs{}) follow the
Advanced Life Support (\ALS{}) guidelines for managing
cardiac arrest published by the American Heart Association.

In \autoref{chapter:hurdles-cdss-adoption}, we described
that wider \CDSS{} adoption is incumbent on a having a
systematic way of developing guidelines with validated
content. The approach described in \autoref{sec:semantics-first}
attempts to enable such a way. It dictates that:
\begin{enumerate*}[label=(\roman*)]
  \item the semantics of the language be formally defined, from
  which tools are derived from a correct-by-construction fashion, and,
  \item ensuring that semantics of medical knowledge is accurately
  described.
\end{enumerate*}
As is typically the case, the \KACLS{} system from
\autoref{sec:kacls-cdss} through collaboration between
\HCPs{} and computer scientists. While the \K{}-based
representation of the \ALS{} guideline was concise,
it was not easily comprehensible to \HCPs{}, or to other
software engineers without prior experience of using \K{}.
This chapter addresses limitations of our work from
\autoref{chapter:k-based-guidelines}. Specifically, we describe
\MediK{} (pronounced \say{Medi-kay} \footnote{\MediK{} is a portmanteau of
Medicine and \K}) a novel domain-specific language (\DSL{}) for
expressing medical knowledge that is designed from the ground-up
with \HCP{}-comprehensibility in mind. As the name suggest, \MediK{}
has a formal $\K$ semantics, from which all tools for it are derived,
including its interpreter. \MediK{} has been used to implement a
complex \CDSS{} for management of sepsis in pediatric cases that
has multiple concurrent workflows. Our \MediK-based system has
been shown to satisfy desired safety properties, and to the
best of our knowledge, is the first such system with formal safety
guarantees.

The rest of this chapter is organized as follows. In \autoref{sec:sepsis-bpg}
we recall the example \BPG{} from section \autoref{sec:bpg-background}
for management of sepsis, to illustrate common requirements
that \DSL{} for modeling \BPGs{} should satisfy.
In \autoref{sec:medik}, we introduce the \MediK{} \DSL{}, and illustrate how
it has been specifically designed to address said requirements.
In \autoref{sec:evalution}, we evaluate the effectiveness of our approach
by utilizing it to build a \CDSSs{} intended for real-world use.
In section \autoref{sec:discussion}, we discuss how \MediK{} builds on
existing approaches in \autoref{chapter:related-work} to advance the
state-of-art in addressing challenges from \autoref{chapter:hurdles-cdss-adoption}.


\section{Pediatric Sepsis Management \BPG{}}\label{sec:sepsis-bpg}

In \autoref{sec:bpg-background}, we presented a best practice
guideline for management of sepsis in pediatric cases.
We briefly describe the guidelines here. Recall that
sepsis is life-threatening condition caused by the body's extreme response to
an infection, and is a major cause of morbidity and mortality
in children. Evidence indicates that timely
identification and prompt treatment with antibiotics and
intravenous (IV) fluids is \emph{vital} for avoiding
adverse outcomes \cite{Weiss2014CCM,Evans2018JAMA}.
The \BPG{} has several concurrent workflows with inter-workflow
dependencies, making it suitable to study common characteristics of \BPGs{},
which we recall here. Specifically, \BPGs{}:
\begin{itemize}
  \item Involve \stress{concurrent} workflows, such as administering drugs,
    monitoring vitals, performing treatment, etc. There may also be
    inter-workflow interactions. For instance, a diagnosis of sepsis during the
    screening may require modifications to an ongoing course antibitiotics.
  \item Often specified in a \stress{flowchart-like}
    notation. See \cite{AHAFlowcharts} and \cite{CancerCareFlowcharts} for other flowchart-based \BPGs{} for management of \emph{cardiac arrest}, and
    screening, risk-reduction, treatment and survivorship in
    cancer care respectively.
  \item Require communication between \stress{heterogeneous agents} such as
     monitors and Electronic Health Records (EHRs).
  \item Often use \stress{tables} indexed by parameters such as age, weight,
    etc to present normal/abnormal ranges for measurements, or recommended dosages for drugs.
\end{itemize}
%While section
%\autoref{sec:bpg-background} briefly discussed some sepsis workflows, here
%we discuss remaining workflows and their interdependencies in detail.

%Recall from \autoref{sec:bpg-background} that once sepsis has been detected,
%it is treated by administering fluids and antibiotics.
%Additionally, sepsis
%lead to septic shock---a condition characterized by acute cardiovascular
%distress.

%Adverse outcomes can, however, be mitigated through timely
%identification and prompt treatment with antibiotics and
%intravenous (IV) fluids \cite{Weiss2014CCM,Evans2018JAMA}.
%\BPGs{} for screening and management of sepsis in pediatric Emergency
%Departments (EDs) have shown effectiveness in screening and management of sepsis \cite{Eisenberg2021JP},
%leading to their adoption in many pediatric EDs \cite{Balamuth2017EM,Sepanski2014FP}.
%
%In \figurename{} \ref{fig:sepsis-screening}, we present a simplified version of
%the screening section of OSF's sepsis mangement guideline.
%In essence, when a patient arrives at the
%\ED{} with a fever or an infection, the \HCP{} is supposed to obtain
%\begin{enumerate*}[label=(\alph*)]
%  \item the patient's age,
%  \item any conditions, such as cancer, immunosuppresssion, etc,
%    that increase likelihood of sepsis, and
%  \item the patient's vital signs, such as heart rate, systolic blood
%    pressure, respiratory rate, etc.
%\end{enumerate*}
%\begin{footnotesize}
%  \begin{table}
%    \centering
%    \begin{tabular}{ | c || c | c | c | }
%      \hline
%      \textbf{Age}            & \textbf{Heart Rate}   & \textbf{Systolic BP} & \textbf{Temp}  \\
%      \hline
%      $0d - 1m$               & $>205$                & $<60$                & $<36 \text{ or } >38$ \\
%      \hline
%      $\geq 1m - 3m$          & $>205$                & $<70$                & $<36 \text{ or } >38$ \\
%      \hline
%      $\geq 3m - 1y$          & $>190$                & $<70$                & $<36 \text{ or } >38.5$ \\
%      \hline
%      $\dots$                 & $\dots$               & $\dots$              & $\dots$ \\
%      \hline
%      $\geq 13y$              & $>100$                & $<90$                & $<36 \text{ or } >38.5$ \\
%      \hline
%    \end{tabular}
%    \caption{Vital Signs Chart}\label{table:vital-signs}
%  \end{table}
%\end{footnotesize}
%
%This information is then used to check for abnormalities
%in clusters of linked information, called \say{buckets}. For instance, if
%the patient's heart rate is abnormal, then \say{bucket 1} is said to
%have an abnormal value.
%Checking for such abnormalities often involves the use of tables, such as
%\tablename{} \ref{table:vital-signs} that contains normal ranges indexed by
%\emph{age}.
%%\footnote{For brevity, we omit some age ranges and vital signs from table
%%\ref{table:vital-signs}}.
%If the patient has at least one abnormal value in every \say{bucket},
%then he/she is flagged as potentially septic.
%
%The \BPG{}-recommended treatment for
%sepsis involves multiple concurrent workflows, such as
%screening for septic shock, fluid resuscitation, and administering antibiotics.
%In \figurename{} \ref{fig:fluid-therapy}, we provide
%a version of the fluid resuscitation guideline used
%at OSF. Briefly, if the patient is flagged as potentially septic, the guideline suggests
%\begin{enumerate*}[label=(\roman*)]
%  \item obtaining any fluid-overload risks,
%  \item administering normal saline (typically over a period of 15 minutes),
%    where the dosage is dictated by risks determined in previous step,
%  \item assessing signs of fluid-overload,
%  \item evaluating patient responsiveness to normal saline upon completion of
%    the administering process, and,
%  \item determining whether another fluid bolus should be administered based on
%    information from previous steps.
%\end{enumerate*}
%\begin{figure}[b]
%  \centering
%  \includegraphics[scale=0.45]{FluidWorkflow-fmcad.pdf}
%  \caption{Fluid Resuscitation Guideline}\label{fig:fluid-therapy}
%\end{figure}
%
%This real-world \BPG{} exhibits characteristics common
%across many \BPGs{}. Specifically \BPGs{} typically:
%\begin{itemize}
%  \item Involve \stress{concurrent} workflows, such as administering drugs,
%    monitoring vitals, performing treatment, etc. There may also be
%    inter-workflow interactions. For instance, a diagnosis of sepsis during the
%    screening may require modifications to an ongoing course antibitiotics.
%  \item Often specified in a \stress{flowchart-like}
%    notation. See \cite{AHAFlowcharts} and \cite{CancerCareFlowcharts} for other flowchart-based \BPGs{} for management of \emph{cardiac arrest}, and
%    screening, risk-reduction, treatment and survivorship in
%    cancer care respectively.
%  \item Require communication between \stress{heterogeneous agents} such as
%     monitors and Electronic Health Records (EHRs).
%  \item Often use \stress{tables} indexed by parameters such as age, weight,
%    etc to present normal/abnormal ranges for measurements, or recommended dosages for drugs.
%\end{itemize}
%
%Note that the aforementioned characteristics are \emph{not} specific
%to one guideline. According to a review paper on \CIGs{} \cite{ClerqAIM03},
%such \DSLs{} should additionally
%\begin{enumerate*}[label=(\alph*)]
%  \item be formally defined, i.e, have a formal syntax and semantics, and
%  \item have an execution engine to provide decision support.
%\end{enumerate*}

In the following sections, we describe how these characteristics dictate the
design philosophy behind \MediK{}. We argue that this philosophy
makes \MediK{} both intuitive to \HCPs{}, and suitable for expressing
complex guidelines.

\section{\MediK{}}\label{sec:medik}

This section introduces the \MediK{} \DSL{} through its $\K$
syntax and semantics.
We developed $\MediK{}$ to realize
the semantics-first approach discussed in \autoref{sec:semantics-first}.
Thus, it has:
\begin{itemize}
  \item A formally defined, unambiguous semantics.
  \item A correct-by-construction interpreter derived from the semantics.
    As discussed in \autoref{chapter:evaluating-k},
  \item A comprehensive suite of formal program analysis tools.
  \item The ability to quickly adapt physician feedback, as only the semantics
    need to be changed an all tools evolve automatically.
\end{itemize}

The remainder of this section introduces the \MediK{} \DSL{}, and describes how it's
designed to accommodate characteristics of \BPGs{} discussed in
\autoref{sec:bpg-background}.

\subsection{\HCP{} Comprehensibility}

Our approach stresses that \HCPs{} must be able to comprehend the
guideline, and if necessary, making changes to on their own. As discussed
in \autoref{chapter:hurdles-cdss-adoption}, this has several hurdles, chief
among which is the fact that \HCPs{} are typically not trained to understand
conventionally programming languages. Thus, \HCPs{} need to work collaboratively
to translate the \BPG{} into a computable medium.
In essence, the \BPG{} serves as a functional
specification for implementing the \CDSS{}. But, this may lead to
a gap in the \HCPs{}' understanding of the system, and the actual
behavior.
To address this, we designed \MediK{}
s.t. encoded guidelines resemble their physical, non-executable counterparts,
with the intention that familiarity with non-executable guidelines
would also translate to computable \MediK{} ones.

Recall from \autoref{sec:generic-bpg} that \BPGs{}
typically involve concurrent workflows,
often expressed using a flowchart-like notation that may involve
inter-workflow interactions. In \autoref{sec:kacls-backend},
we discussed the merits and suitability of concurrently executing
state machines for modeling medical guidelines.
Thus, we looked at state-of-art languages for modeling large concurrent
systems using state machines, such as P \cite{DesaiPLDI13}, but made
adaptions to make expressing \BPGs{} easier.

In \MediK{}, like in P, programs are expressed as concurrently
executing instances of state machines that communicate via passing messages.
Given a \BPG{} where each workflow is expressed as a flowchart,
we express said flowcharts as State Machines in \MediK{}. Each flowchart node
in the \BPG{} is represented as a state in a state machine, and
edges are represented as state transitions. During execution,
instances of these machines are created, which interact with each other by
passing events. Note the distinction between
machine and its instance. A machine
is analogous to an Object Oriented Programming (OOP) class, whereas
its instance is analogous to an OOP object.

%We achieve this by defining \MediK{} (i.e., its syntax and semantics) in $\K$.
%$\K{}$ is a rewriting-based framework for defining executable
%semantics of languages, type systems and formal analysis tools.
%It has been successfully used to define executable semantics
%of many real world languages such as C \cite{HathhornPLDI15}, Java
%\cite{BogdanasPOPL15}, Javascript \cite{ParkPLDI15}, and the
%Ethereum Virtual Machine \cite{HildenbrandtCSF18}.
%We will introduce $\K$ by need while discussing \MediK{}. For more details on
%$\K$, we refer the reader to \cite{SerbanutaETNCS14} \cite{RosuJLAP10}.
%Once the executable
%semantics of a language have been defined in $\K{}$, it provides us with
%suite of tools such as an interpreter, deductive-verifier and a
%model-checker as shown in \figurename{} \ref{fig:k-overview}.

%The $\K{}$ ecosystem provides a suite of tools, such as an interpreter,
%model-checker, and deductive verifier that are parametric over the language's
%semantics, as shown in \figurename{} \ref{fig:k-overview}. Thus, by
%defining the semantics of \MediK{} in $\K{}$, we obtain aforementioned
%tools for it without any extra effort. Additionally:
%\begin{itemize}
%  \item The $\K$-based interpreter for \MediK{} essentially executes the language's
%    semantics rules, it is correct-by-construction.
%  \item Incorporating changes to \MediK{} only requires updating
%   the semantics. Since the tools are derived from the semantics,
%   they're automatically updated.
%\end{itemize}


%The remainder of this section introduces \MediK{} and describes
%how it's designed around characteristics of \BPGs{} from Section \ref{sec:motivating-example}.
%Recall that \BPGs{} typically involve concurrent workflows, often expressed using a
%flowchart-like notation that may involve
%inter-workflow interactions. To ensure \MediK{} programs are comprehensible
%to \HCPs{}, they must be representable in a flow-chart like notation that \HCPs{}
%are already comfortable with, and be capable of expressing inter-workflow
%interactions succinctly. To address these requirements, we borrow from
%from existing state-of-art languages for modeling large concurrent
%systems, like P \cite{DesaiPLDI13}, but make adaptions to make expressing and
%validating \BPGs{} easier. We explore the differences
%to existing techniques in section \ref{sec:related-works}.

Next, we describe \MediK{} using its $\K$-framework definition.
Recall from \autoref{sec:semantics-in-k}, the $\K{}$ definition
of a language has two components. The first is the language's
syntax, which is defined using a BNF-like notation. $\K{}$
utilizes this grammar to generate a parser for program in the language. We
describe \MediK{}'s syntax in depth in Section \ref{sec:syntax}.
The second is the semantics, which is defined using a $\K{}$-configuration and
rewrite rules. The $\K{}$-configuration
organizes the program's execution state. Rewrite rules
that operate over said configuration dictate the evolution
of program state during execution.
We describe the semantics in greater depth in Section
\ref{sec:semantics}\footnote{The complete executable semantics is available at
 \cite{medikUrl}.}

\subsection{Syntax}\label{sec:syntax}
We use the skeleton of a \MediK{} machine in \autoref{lst:machine-skeleton}, and use
it to describe the syntax.
Note that we use \inlinemedik{[...]} to denote
optional constructs, \inlinemedik{<...>} for mandatory constructs, lowercase for
terminals, and uppercase for non-terminals.
\begin{lstlisting}[
  float=ht,
  frame=single,
  style=mediksty,
  language=medik,
  numbers=left,
  numberstyle=\tiny,
  caption={Skeleton of a \MediK{} Machine},
  label={lst:machine-skeleton},
  xleftmargin=3ex
]
[init] machine <IDENTIFIER>                     @\label{lstline:machine-declaration}@
  receives <IDENTIFIER_LIST> {                  @\label{lstline:machine-receives}@
  vars <IDENTIFIER_LIST>;                       @\label{lstline:global-declarations}@

  [init] state <IDENTIFIER> {                   @\label{lstline:state-block-start}@
    entry [(IDENTIFIER_LIST)] {                 @\label{lstline:entry-block-start}@
      <STMT> // entry block
    }                                           @\label{lstline:entry-block-end}@
    on <IDENTIFIER> [(IDENTIFIER_LIST)] do {    @\label{lstline:handler-block-start}@
      <STMT> // event handler
    }                                           @\label{lstline:handler-block-end}@
  }                                             @\label{lstline:state-block-end}@
}
\end{lstlisting}
%A machine $m = \left(S, E, \Var, s_i\right)$,
%where $S$ is the set of states, $E$ the set of events,
%$\Var$, the set of instance variables, and $s_i$ the initial state.
%As shown in \figurename{} \ref{fig:machine-def}, a typical

A \MediK{} program consists of a set of machine definitions, where
a machine definition consists of:
\begin{itemize}
  \item (Lines \ref{lstline:machine-declaration}-\ref{lstline:machine-receives})
    The keyword \inlinemedik{machine}, followed by the name
    and a comma-separated list of identifiers signifying events that
    it \inlinemedik{receives} via the broadcast mechanism.
    One state in every machine, and one machine in a program can be
    prefixed with the keyword \inlinemedik{init}. On execution, an implicit instance
    of this machine the created, and the \inlinemedik{entry} block of
    initial state executed.
  \item (\autoref{lstline:global-declarations}) A set of instance variables.
  \item (Lines \ref{lstline:state-block-start}-\ref{lstline:state-block-end}) A set of state declarations. Each
    state has a name, an optional entry block (Lines
    \ref{lstline:entry-block-start}-\ref{lstline:entry-block-end}),
    and a set of event handlers (Lines \ref{lstline:handler-block-start}-\ref{lstline:handler-block-end}).
    The entry block begins with the keyword \inlinemedik{entry}, and may contain a list of variables
    that are bound to values when an instance enters the state during execution.
    One state in the machine may be prefixed with \inlinemedik{init}, specifying the
    initial state that an instance starts execution in.
  \item Event handlers begin with \inlinemedik{on} followed by the event name,
    an optional list of variables bound to data in the event, followed by the
    keyword \inlinemedik{do} a block of the handler's code.
\end{itemize}
%A machine definition starts with the keyword \inlinemedik{machine},
%followed by its name (\autoref{lstline:machine-declaration}). On
%\autoref{lstline:machine-receives}, following the
%\inlinemedik{receives} keyword, is a comma-separated list of identifiers
%signifying the events that the machine can receive from other machines.
%One machine in a program can be
%prefixed with the \inlinemedik{init} keyword. This machine is referred to as the
%initial machine.
%On \autoref{lstline:global-declarations}, following the keyword
%\inlinemedik{vars}, another comma-separeted list of identifiers signifies
%the instance-variables. During execution, each instance maintains a mapping from
%these variables to values. Each machine defines a set of states, such as
%the one in Lines \ref{lstline:state-block-start}-\ref{lstline:state-block-end}.
%A state has a name, an optional entry block
%(\autoref{lstline:entry-block-start}-\autoref{lstline:entry-block-end}),
%and a set of event handlers (Lines
%\ref{lstline:handle-block-start}-\ref{lstline:handler-block-end}). The entry block
%begins with the keyword \inlinemedik{entry}, and may contain a list of variables
%that are bound to values when the state is entered during execution.
%One state in the machine may be prefixed with \inlinemedik{init}, specifying the
%initial state. When execution begins, an implicit instance
%of the initial machine is created, and the \inlinemedik{entry} block of
%its initial state is executed. When an instance of a machine is dynamically
%created during runtime, the \inlinemedik{entry} block of its initial state is executed.
%Event handlers within a state begin with \inlinemedik{on} followed by the event name
%and an optional list of variables. When the event handler is executed, data from
%the received event's payload is bound to aforementioned variables which
%can be used in the code block that follows the \inlinemedik{do} keyword.
%A machine definition consists of:

%\begin{figure}[h]
%  \centering
%\begin{lstlisting}[style=mediksty,language=medik,basicstyle=\ttfamily\scriptsize,numbers=right]
%[init] machine <IDENTIFIER>
%  receives <IDENTIFIER_LIST> {
%  vars <IDENTIFIER_LIST>;
%
%  [init] state <IDENTIFIER> {
%    entry [(IDENTIFIER_LIST)] {
%      <STMT> // entry block
%    }
%    on <IDENTIFIER> [(IDENTIFIER_LIST)] do {
%      <STMT> // event handler code
%    }
%  }
%}
%\end{lstlisting}
%\caption{Machine Definition in \MediK{}}\label{fig:machine-def}
%\end{figure}

Each entry and event handler block contains statements defined
by syntax shown in \autoref{lst:medik-stmt-syntax}. The statements
are written over expressions given by the syntax in
\autoref{lst:medik-exp-syntax}.

Recall from \autoref{sec:semantics-in-k}, in $\K$, productions
are defined using the keyword \inlinek{syntax}, where
terminals are enclosed in quotes (\inlinek{""}), and non-terminals
begin with an upppercase character.

\MediK{} uses statement over expressions resemble counterparts in many commonly
used programming languages. For instance, Lines
\ref{lstline:value-start}-\ref{lstline:value-end} enable one to write
expressions using program identifiers (denoted by the builtin $\K{}$ production
\inlinek{Id}), and values such as booleans, or rationals, or \say{\inlinek{this}}, which enables
an instance to refer to itself. \autoref{lstline:dot-syntax} defines the usual dot operator (\inlinek{.}),
which can be used to access members of an instance.
Lines \ref{lstline:exp-syntax-begin}-\ref{lstline:exp-syntax-end} declare
common expressions such as \inlinek{+, -. >, >=} over rationals and \inlinek{&&, ||, !} over booleans.
We use the production \inlinek{StandaloneExp} to define certain expressions that
are typically not used in conjunction with other expressions, such as
\inlinek{new (...)} on \autoref{lstline:new-syntax} that resembles constructs in
OOP languages used to create object instances. Note however, that \MediK{} also
has several constructs not commonly found in other languages, such as
\inlinek{createFromInterface(...)} (\autoref{lstline:create-from-interface-syntax}) and \inlinek{obtainFrom(...)}
(\autoref{lstline:obtain-from-syntax}). We shall describe their need and purpose
in upcoming sections.

\begin{lstlisting}[
  float=ht,
  frame=single,
  style=ksty,
  basicstyle=\ttfamily\footnotesize,
  language=k,
  numbers=left,
  numberstyle=\tiny,
  xleftmargin=3ex,
  caption={\MediK{} Expressions Syntax},
  label={lst:medik-exp-syntax}
]
syntax StandaloneExp ::= "new" Id "(" Exps ")"                       [strict(2)] @\label{lstline:new-syntax}@
                       | "createFromInterface" "(" Id "," String ")" [strict(2)] @\label{create-from-interface-syntax}@
                       | Id "(" Exps ")"                             [strict(2)]
syntax Exp ::= Id                                                                @\label{lstline:value-start}@
             | Val
             | Rat
             | FloatLiteral
             | "this"                                                            @\label{lstline:value-end}@
             | UndefExp
             | "obtainFrom" "(" Exp "," Exp ")"            [seqstrict]           @\label{lstline:obtain-from-syntax}@
             | "(" Exp ")"                                 [bracket]
             > Exp "." Exp                                 [strict(1), left]     @\label{lstline:dot-syntax}@
             > Exp "+" Exp                                 [seqstrict, left]     @\label{lstline:exp-syntax-begin}@
             | Exp "-" Exp                                 [seqstrict, left]
             | Exp "*" Exp                                 [seqstrict, left]
             | Exp "/" Exp                                 [seqstrict, left]
             | Exp ">" Exp                                 [seqstrict, left]
             | Exp "<" Exp                                 [seqstrict, left]
             | Exp ">=" Exp                                [seqstrict, left]
             | Exp "<=" Exp                                [seqstrict, left]
             | "!" Exp                                     [seqstrict, left]
             | Exp "&&" Exp                                [strict(1), left]
             | Exp "||" Exp                                [strict(1), left]
             > Exp "==" Exp                                [seqstrict, left]     @\label{lstline:exp-syntax-end}@
             | "interval" "(" Exp "," Exp ")"
             > Exp "in" Exp
             | "parseInt" "(" Exp ")"                      [strict]
             | StandaloneExp
\end{lstlisting}

As reflected in \autoref{lst:medik-stmt-syntax} \MediK{} support many statements
commonly found in conventional programming languages, such as
variable assignment (\autoref{lstline:assignment-stmt}), \inlinemedik{if-else}
(Lines \ref{lstline:if-stmt} and \ref{lstline:if-else-stmt}) and \inlinemedik{while}
(\autoref{lstline:while-stmt}). Others, such as \inlinemedik{broadcast} (Lines
\ref{lstline:broadcast-stmt-macro} and \ref{lstline:broadcast-stmt-syntax}),
\inlinemedik{goto} and \inlinek{interface} declarations have nuanced meanings,
and will be explained in upcoming sections.

\begin{lstlisting}[
  float=h
  ,style=ksty
  ,language=k
  ,basicstyle=\ttfamily\footnotesize,
  ,numbers=left
  ,numberstyle=\tiny
  ,xleftmargin=3ex
  ,caption={\MediK{} Statement Syntax}
  ,label={lst:medik-stmt-syntax}
]
  syntax Stmt ::= StandaloneExp ";"                               [strict]
                | "sleep" "(" Exp ")" ";"                         [strict(1)]
                | "send" Exp "," ExtId ";"                        [macro]
                | "send" Exp "," ExtId "," "(" Exps ")" ";"       [seqstrict(1, 3)]
                | "broadcast" Id ";"                              [macro]     @\label{lstline:broadcast-stmt-macro}@
                | "broadcast" Id "," "(" Exps ")" ";"             [strict(2)] @\label{lstline:broadcast-stmt-syntax}@
                | "goto" Id ";"                                   [macro]
                | "goto" Id "(" Exps ")" ";"                      [strict(2)]
                | "print" "(" Exp ")" ";"                         [strict]
                > "return" ";"                                    [macro]
                | "return" Exp ";"                                [strict(1)]
                | "var" Id "=" Exp ";"                            [macro]
                > Exp "=" Exp ";"                                 [strict(2)] @\label{lstline:assignment-stmt}@
                > "var" Id ";"
                | "vars" Ids ";"                                  [macro-rec]
                | Block
                > "if" "(" Exp ")" Block                          [strict(1)] @\label{lstline:if-stmt}@
                | "if" "(" Exp ")" Block "else" Block             [strict(1)] @\label{lstline:if-else-stmt}@
                | "while" "(" Exp ")" Block                                   @\label{lstline:while-stmt}@
                | "entry" Block                                   [macro]
                | "entry" "(" Ids ")" Block
                | "on" ExtId "do" Block                           [macro]
                | "on" ExtId "(" Ids ")" "do" Block
                | "fun" Id "(" Ids ")" Block
                | NonDetStmt
                | Exp "in" "{" CaseDecl "}"
                | StateDecl
                > "machine" Id Block                              [macro]
                | "machine" Id "receives" Ids Block
                | "interface" Id Block                            [macro]    @\label{lstline:interface-declaration}@
                | "interface" Id "receives" Ids Block                        @\label{lstline:interface-declaration}@
                | "init" "machine" Id Block                       [macro]
                | "init" "machine" Id "receives" Ids Block
                | "yield" ";"
                > Stmt Stmt                                       [right]
\end{lstlisting}

In lines 7-15, we define syntax for \MediK{} statements. Some of these,
such as variable assignment (line 7), \inlinek{if-else} (line 8)
and \inlinek{new Id(..);} (line 9) are commonly found in other languages, and
have expected meanings. The remaining statements (lines 10-15) have nuanced
meanings in context of state machines. We shall go over these while discussing
\MediK{}'s semantics in section \ref{sec:semantics}.

%\begin{figure}[b]
%  \centering
%\begin{lstlisting}[style=ksty,language=k,basicstyle=\ttfamily\scriptsize,numbers=right]
%configuration
%  <instance multiplicity="*" type="Map"> ...
%    <k> createMachineDefs($PGM)
%     ~> createInitInstances </k>
%    <genv> .Map </genv>
%    <env> .Map </env>
%    <inBuffer> .List </inBuffer>
%    <activeState> . </activeState>
%  </instance>
%  <machine multiplicity="*" type="Map"> ...
%    <machineName> . </machineName>
%    <states>
%      <state multiplicity="*" type="Map">
%        <stateName> . </stateName>
%        <entryBlock> . </entryBlock>
%        <eventHandlers> ... </eventHandlers>
%      </state>
%    </states>
%  </machine>
%\end{lstlisting}
%\caption{\MediK{}'s $\K$ Configuration}\label{fig:k-config}
%\end{figure}

\subsection{Semantics}\label{sec:semantics}
% Once the syntax has been defined, $\K$ can
% automatically generate a parser for programs in the language.
% The semantics, described via $\K$ rewrite rules, describe
% how the program evolves
% during execution.
Semantics of a language defined in $\K$ has two components:
\begin{enumerate*}[label=(\arabic*)]
  \item description of program state via $\K$-configurations, and
  \item $\K$ rules that dictate state evolution.
\end{enumerate*}
Next we describe these components in detail.

\subsubsection{$\K{}$-Configuration}
$\K$ represents program execution state using $\K$-configurations.
A $\K$-configuration is an unordered list of (potentially nested) \emph{cells},
specified using an XML-like notation.
When declaring rules (as rewrites) over this state,
any subset of the cells present in the configuration can be mentioned.
This allows specifying only necessary parts of the state for a given rule,
letting $\K{}$ assume that the rest of the configuration remains unchanged.
The following configuration defines the initial state for any \MediK{} program:

\begin{lstlisting}[
  style=ksty
  ,language=k
  ,basicstyle=\ttfamily\scriptsize
  ,numbers=left
  ,numberstyle=\tiny
  ,framexleftmargin=1.5em
  ,xleftmargin=2em
  ]
configuration
  <instance multiplicity="*" type="Map"> ...
    <k> createMachineDefs($PGM)
     ~> createInitInstances </k>
    <genv> .Map </genv>
    <env> .Map </env>
    <inBuffer> .List </inBuffer>
    <activeState> . </activeState>
  </instance>
  <machine multiplicity="*" type="Map"> ...
    <machineName> . </machineName>
    <states>
      <state multiplicity="*" type="Map">
        <stateName> . </stateName>
        <entryBlock> . </entryBlock>
        <eventHandlers> ... </eventHandlers>
      </state>
    </states>
  </machine>
\end{lstlisting}

The keyword \inlinek{configuration} (line 1) defines a $\K{}$-configuration,
followed by xml-like notation for the $\K$-cells. For example \inlinek{<foo> ... </foo>}
corresponds to a $\K$-cell with the name \inlinek{foo}.
The \inlinek{<instance>} cell (lines 2-9) contains state of each \MediK{}
machine instance during execution. Each instance manages its
instance variables using a map in the \inlinek{<genv>} cell (line 5),
a buffer of incoming events in the \inlinek{<inBuffer>} cell (line 7)
and the currently executing code in the \inlinek{<k>} cell (lines 3-4).\footnote{For brevity, we present a simplified version of the configuration. See \cite{medikUrl} for the entire configuration.}
When a \MediK{} program is executed, $\K$ replaces \inlinek{$PGM} (line 3) with the Abstract Syntax Tree (AST)
of the program, obtained by parsing the program using the syntax from
section \ref{sec:syntax}.
The \inlinek{createMachineDefs} constructs
is defined (using rewrite rules) to traverse the program AST
and populate the configuration with information related to each machine.
The \inlinek{createInitInstances} creates an instance for the machine
with the \inlinek{init} keyword, leading to execution of the initial
machine's entry block. Note that \scantokens{\lstinline{~>}} symbol (line 3)
is interpreted by $\K$ as \say{followed-by}, i.e., execution of \inlinek{createMachineDefs}
is followed by execution of \inlinek{createInitInstances}.
The attribute \inlinek{multiplicity="*"}
on lines 2 and 10 signifies that multiple copies of the
corresponding cells, in this case \inlinek{<machine>} and \inlinek{<instance>} cells,
can exist in the configuration during execution. This allows, during execution,
for multiple machine definitions, each with multiple
instances, to exist.
The \inlinek{<machine>} cell (lines 10-19) holds information relevant to
a machine definition, such as the name in the \inlinek{<machineName>}
cell (line 11) and states in the \inlinek{<states>} cell (lines 12-18).
The \inlinek{<state>} (lines 13-17) holds information
relevant to a state, such as the entry block in cell
\inlinek{<entryBlock>} (line 15) and event handlers in cell
\inlinek{<eventHandlers>} (line 16).
%using the \inlinek{createMachineDefs}
%construct, and new instances for those definitions, using the \inlinek{new} keyword.


\subsubsection{$\K$-Rules}
$\K$-rules operate over the configuration and
define the evolution of program state during execution. A $\K$-rule
begins with the keyword \inlinek{rule}, and is a statement of the form $\phi
\Rightarrow \psi$, where $\phi$ and $\psi$ are
patterns over configuration terms and $\K$-variables. We
say $\phi$ is the $\LHS$ and $\psi$ is the $\RHS$ of the rule.
Let substitution $\theta$ be a map from $\K$-variables to terms. Say, for given
pattern $\phi$ and substitution $\theta$, $\phi\theta$ be the term
obtained by replacing each variable $v$ in $\phi$ with $\theta(v)$.
During execution, if the current configuration $C$, i.e. program execution state,
matches $\phi$ with substitution $\theta$, then it is rewritten to
$\psi\theta$. We say pattern $\phi$ matches configuration $C$ iff
there exists a substition $\theta$ s.t. $C = \phi\theta$.
For example, consider the following rule for updating the value of a local program variable.
\begin{lstlisting}[
  style=ksty
  ,language=k
  ,basicstyle=\ttfamily\scriptsize
  ,numbers=left
  ,numberstyle=\tiny
  ,framexleftmargin=1.5em
  ,xleftmargin=2em
  ]
rule <k> I:Id = V:Val => V ... </k>
     <env> (I |-> Loc) ... </env>
     <store> Store => Store[Loc <- V] </store>
\end{lstlisting}
Here, \inlinek{I}, \inlinek{V}, \inlinek{Loc}, and \inlinek{Store}
are $\K$-variables. Note the distinction between program variables and $\K$-variables:
while program variables are simply identifiers, $\K$-variables have logical meaning.
The \inlinek{...} is used to denote parts of the configuration
not relevant to the rule.
Typically, the top of the \inlinek{k} cell contains the statement currently
being executed. Suppose we're executing the statment \inlinek{i = 2;}. In this
case, the current configuration will have a \inlinek{k} cell of the form
\inlinek{<k> i = 2 ... </k>}, an environment cell \inlinek{env} where variable
\inlinek{i} maps to some pointer $p$, and a store cell
\inlinek{store} containing a map $M$ with some value pointed-to by $p$.
The $\LHS$ matches with substitution
$\theta = \left(I \mapsto i, V \mapsto 2, Loc \mapsto p, Store \mapsto M\right)$,
resulting in the top of the \inlinek{k} cell to be rewritten to the value $2$,
and pointer $p$ updated to point to $2$ in $M$.
Note if there exist multiple rules that can match the current configuration,
then one rule is non-deterministically chosen and applied. An execution is a
sequence of rule applications that continues until no rule matches the configuration.

The following sections we present several \MediK{} constructs relevant to
defining \BPGs{} using their $\K$-rules. We first present the
rule for sending and receiving messages.
\begin{lstlisting}[
  style=ksty
  ,language=k
  ,basicstyle=\ttfamily\scriptsize
  ,numbers=left
  ,numberstyle=\tiny
  ,framexleftmargin=1.5em
  ,xleftmargin=2em
  ]
rule
<instance>
 <k> send instance(RecvId) , EventName:Id , ( Args )
     =>  done ... </k> ...
</instance>
<instance>
 <id> RecvId </id> ....
 <inBuffer> ... (.List
  => ListItem(
      eventArgsPair(EventName | Args | Epoch + 1
      )))
 </inBuffer> ...
</instance>
<epoch> Epoch </epoch>
\end{lstlisting}
When the top of the \inlinek{k} cell has
\inlinek{send},  the rule above
\begin{enumerate*}[label=(\roman*)]
  \item obtains the \inlinek{id} of the receiver instance,
    the event name and the event arguments by matching
    the variables \inlinek{RecvId}, \inlinek{EventName} and \inlinek{Args}
    against the current configuration (line 3),
  \item rewrites the top of the \inlinek{k} cell (line 4) to \inlinek{done},
    marking the completion of execution for the construct,
  \item adds the event and associated arguments to the
    buffer of incoming events (lines 8-12) of the instance
    with \inlinek{id} \inlinek{RecvId} (line 7).
  \item The \epoch{} decides when the machine can run, and is discussed in
    Section \ref{sec:scheduling-semantics}.
\end{enumerate*}

To handle interaction with heterogeneous external sources, \MediK{}
models them as interfaces. An interface is a \FSM{} that has its
transition system defined externally. For example, certain measurements
such as the heart rate are often obtained from sensors.
The following code shows the process of obtaining external measurements in
\MediK{}.
\begin{lstlisting}[
  style=mediksty
  ,language=medik
  ,basicstyle=\ttfamily\scriptsize
  ,numbers=left
  ,numberstyle=\tiny
  ,framexleftmargin=1.5em
  ,xleftmargin=2em
  ]
interface HeartRateSensor { }

machine TreatmentMachine { ...
  var hrSensor = createFromInterface(HeartRateSensor,
                              "heartRateSensor");
  var heartRate = obtainFrom(hrSensor, "heartRate");
}
\end{lstlisting}
Since we don't have the transition system for
the heart rate sensor, we declare it as an interface (line 1).
Next, instead of using \inlinemedik{new} to create an instance, we use
a builtin \MediK{} construct \inlinemedik{createFromInterface}, which takes as
arguments
\begin{enumerate*}[label=(\alph*)]
  \item the inteface name (lines 4),
  \item a unique identifier string used to identify the
    instance outside the \MediK{} process.
\end{enumerate*}
All other \MediK{} machines can interact with external sensor using
variable \inlinemedik{hrSensor}. There is no need to make any distinction
between external, and \MediK{}-based machines.
To deal with external interactions, input and output
pipes are provided to the \MediK{} process at launch.
When the \inlinemedik{send} construct is used on an external machine,
\MediK{} will write
a JSON \cite{jsonUrl} message with the event data, the
identifier from line 5, and a unique transaction id
to the \emph{write-end} of the output pipe. At the
\emph{read-end}, we need to write external code (in any programming language)
to handle the JSON message. In the example above, this involves reading from
the external heart rate sensor. To send data to \MediK{}, a JSON message in
a pre-specified format needs to be written to the \emph{write-end} of the
input pipe.

Next, we desribe the rule for supporting tables in \MediK{}.
Once a measurement, such as the heart rate has been obtained
from a sensor, we need to use a table, such as \tablename{} \ref{table:vital-signs}
to check if the measurement is within a normal range. In \MediK, we can write a
function that does the required check, as shown in \figurename{}
\ref{fig:hr-check-fun}.
\begin{figure}[h]
  \centering
\begin{lstlisting}[
  style=mediksty
  ,language=medik
  ,basicstyle=\ttfamily\scriptsize
  ,numbers=left
  ,numberstyle=\tiny
  ,framexleftmargin=1.5em
  ,xleftmargin=2em
  ]
fun isHeartRateNormal() {
  days(age) in {
    interval(days(0)  , months(1)): return hr > 205;
    interval(months(1), months(3)): return hr > 205;
    // omitting other cases
    default                       : return hr > 100;
  }
}
\end{lstlisting}
  \caption{Checking abnormality using tables}\label{fig:hr-check-fun}
\end{figure}
In the code, if the \inlinemedik{age} lies in
any of the intervals (closed on the left, open on the right) on lines 3-5, the
corresponding statement to the right of the colon (\inlinemedik{:}) is run. Otherwise
line 6 is run.
In \MediK{}, the following rules are responsible for assigning semantics to
the \inlinemedik{in-interval} construct:
\begin{lstlisting}[
  style=ksty
  ,language=k
  ,basicstyle=\ttfamily\scriptsize
  ,numbers=left
  ,numberstyle=\tiny
  ,framexleftmargin=1.5em
  ,xleftmargin=2em
  ]
rule E in interval(L, U) => (E >= L) && (E < U)
  [macro]
rule E in { interval(L, U): S:Stmt Cs:CaseDecl }
  => if (E in interval(L, U)) {S} else {E in { Cs }}
  [macro-rec]
\end{lstlisting}

Note the rules above are marked with the attributes \inlinek{macro}
(line 2) or \inlinek{macro-rec} (line 5). This specifies that
these constructs are not part of the language's semantics, but merely
syntactic sugar. On line 1, we specify that
\inlinek{E in interval(L, R)} desugars to checking the expression \inlinek{e}
is between the lower and upper bound \inlinek{L} and \inlinek{U} respectively.
Similary we desugar each case statement to an \inlinek{if-else} statement.
In lines 3-5, we say that if the expression \inlinek{E} is in
\inlinek{interval} with lower and upper bounds \inlinek{L} and \inlinek{U}
respectively, then execute \inlinek{S}, otherwise check \inlinek{E} against the
remaining cases \inlinek{Cs}. Note the postfix \inlinek{-rec} after
\inlinek{macro} specifies that the rule applies recursively, to desugar
the remaining case statements.

\paragraph{\MediK{} Scheduling Semantics}\label{sec:scheduling-semantics}

Since the $\K{}$-generated
interpreter is single-threaded, \MediK{} employs interleaving-semantics
for concurrency, using a single \inlinemedik{executor} thread shared
between machine instances. A machine instance that is either at the start of
an entry block, or has an event in the input buffer that it can handle is
said to be \emph{enabled}, i.e. one that can run once the \inlinemedik{executor}
becomes available. But, a naive strategy that non-deterministically
chooses one \emph{enabled} machine instance may lead to unfairness.
Specifically, there may be situations where a machine instance is
\emph{enabled} but is never chosen for execution.
Therefore, to ensure fairness, we use a scheduling strategy based on a monotonically
increasing global counter called the \epoch{}.
% Instead of just tracking
% \emph{enabled} machines, we also track the \epoch{} value at which
% said machines became enabled.
We show this execution strategy in \figurename{}
\ref{fig:scheduling-semantics}

\begin{figure}[ht]
  \centering
    \begin{algorithmic}[1]

    \State $\epoch{} \gets 0$
      \State $\scheduled{} \gets \left\{ \Instance_{\Machine{}_{0},0}^{0} \right\}$
    \While{$\scheduled{} \neq \emptyset$}
       \State \begin{varwidth}[t]{\linewidth}
           $\Instance_{\Machine_{i},j}^{\tau} \gets \textit{choose}\left(\scheduled{}\right) \text{ s.t. }$
           \par
              \hskip\algorithmicindent $\qquad \tau \leq \epoch{} \wedge \enabled(\Instance_{\Machine_{i},j}^{\tau})$ \par
            \end{varwidth}
      \State $\scheduled{} \gets \scheduled \setminus \Instance_{\Machine_{i},j}^{\tau}$
      \State $\textit{execute}(\Instance_{\Machine_{i},j}^{\tau}, \scheduled{})$
      \If{$\not\exists i',j',\tau'$ s.t.
      $(\Instance_{\Machine_{i'},j'}^{\tau'} \in \scheduled{})$
      \\ \qquad \qquad $\wedge (\tau' \leq \epoch{}) \wedge (\enabled(\Instance_{\Machine_{i'},j'}^{\tau'}))$}
      \newline
       \State $\epoch{} \gets \epoch{} + 1$
      \EndIf
    \EndWhile
  \end{algorithmic}
  \caption{\MediK{} Scheduling Semantics}\label{fig:scheduling-semantics}
\end{figure}

Recall from Section \ref{sec:syntax} that a \MediK{} program consists of a set
of machines, of which one, prefixed with the keyword \inlinemedik{init}, is the \emph{initial} machine.
Each machine has one state prefixed with \inlinemedik{init}, referred to as the
\emph{initial} state.
Let $P = \{ \Machine_{0}, \Machine_{1}, \dots, \Machine_{n-1} \}$ be a program
with $n$ machines, where $\Machine_{0}$ is prefixed with \inlinemedik{init}.
\MediK{} allows instances of a machine to be created dynamically at runtime.
For machine $\Machine_i \in P$, let $\Instance_{\Machine_i,j-1}$ be its
$j$-th instance.

Execution begins in \epoch{} zero with the implicit (first) instance of the
initial machine, denoted by $\Instance_{\Machine_{0},0}^{0}$.
We use $\Instance_{\Machine_{i},j-1}^{\tau}$ to
say that the $j$-th instance of machine $\Machine_{i}$
is scheduled for execution in \epoch{} $\tau$.
Recall that a state definition may have an entry block, containing code
that is executed when the state is entered, or event handlers containing
code that is executed when an event is dequeued from the input buffer. When execution
begins, the entry block of the \emph{initial state} of the \emph{implicit instance}
of the \emph{initial machine} $\Instance_{\Machine_0,0}$ becomes scheduled (line 2) at \epoch{} 0.
On line 4, an instance $\Instance_{\Machine_{i},j}^{\tau}$ is
non-deterministically chosen from all machines that are both scheduled to run when
$\tau \leq \epoch{}$ and \textit{enabled}. We use
$\textit{execute}(\Instance_{\Machine_{i},j}^{\tau}, \scheduled{})$ on line 5
to denote this execution process. Execution of the entry or event handler block is atomic,
i.e., a context-switch can only occur at the end of the block. Note that when
a new instance of a machine is created using the keyword \inlinemedik{new}, the
\emph{entry} block of the \emph{initial} state of the \emph{target} machine
is executed synchronously before control returns to the source machine, and the
instance is added to the multiset of \textit{scheduled} machines.
A context switch only occurs in three cases: \inlinemedik{goto}, \inlinemedik{sleep}, and \inlinemedik{obtainFrom},
which we describe later.

During execution, if an instance $\Instance_{\Machine_i,j}$ sends an event
to another instance $\Instance_{\Machine_{i'},j'}$, then
the event is scheduled to be handled by $\Instance_{\Machine_{i'},j'}$ in or
after the next epoch, i.e., $\scheduled{} \gets \scheduled{} \cup
\{\Instance_{\Machine_{i'}, j'}^{\epoch{}+1}\}$. Similary, if a \inlinemedik{goto}
statement is encoutered, the entry block of the target state is scheduled
for execution at \epoch{} + 1. If no other machine is both \textit{scheduled} to run
in the current \epoch{}, and \textit{enabled}, then the \epoch{} advances by one (line 8).

%\MediK{} employs a simple epoch-based scheduling approach.
%Execution begins in epoch 0, with
%
%
%\begin{lstlisting}[style=mediksty,language=medik,basicstyle=\ttfamily\scriptsize,numbers=left]
%rule
%<k>   enterState(SName | Args | Scheduled )
% =>   StateDecls
%   ~> assign(BlockVars | Args)
%   ~> EntryBlock
%   ~> releaseExecutor
%   ~> handleEvents ...
%</k>
%<activeState> _ => SName </activeState>
%<env> _ => .Map </env>
%<class> MName </class>
%<machine>
% <machineName> MName </machineName>
%  <state>
%   <stateName> SName </stateName>
%   <stateDeclarations>
%    StateDecls
%   </stateDeclarations>
%   <entryBlock> EntryBlock </entryBlock>
%   <args> BlockVars </args> ...
%  </state> ...
% </machine>
%<executorAvailable>
%  true => false
%</executorAvailable>
%<epoch> Current </epoch>
% requires Scheduled <=Int Current
%\end{lstlisting}

%Initially, the machine prefixed with the \inlinemedik{init} is
%executed, at epoch 0. We say a machine is enabled, if:
%\begin{enumerate*}[label=(\roman)]
%  \item it is at the start of an entry block of a state,
%  \item the entry block
%\end{enumerate*}


\paragraph{Timer Semantics}
Next, we discuss how \MediK{} handles temporal aspects of \BPGs{}. For instance, consider the
Fluid resuscitation guideline \BPG{} from Section \ref{fig:fluid-therapy}.
After administering fluids, the \BPG{} recommends waiting for 15 minutes
before evaluating their effectiveness. This waiting behavior in
\MediK{} is implemented using a \inlinemedik{sleep(duration)} statement.
Formalizing the execution semantics of such a statement in $\K{}$ presents
a challenge as $\K{}$ does not provide builtin support for timers. Therefore,
in \MediK{}, \inlinemedik{sleep(duration)} is described by the following rule:
\begin{lstlisting}[
  style=ksty
  ,language=k
  ,basicstyle=\ttfamily\scriptsize
  ,numbers=left
  ,numberstyle=\tiny
  ,framexleftmargin=1.5em
  ,xleftmargin=2em
  ]
rule <k> sleep(Duration:Int) ;
      =>    jsonWrite( { "action"   : "sleep"
                       , "duration" : Duration
                       , "tid"      : TId }
                       , ... )
         ~> releaseExecutor
         ~> waitForSleepResponse(TId) ...
     </k>
     <tidCount> TId => TId +Int 1 </tidCount>
\end{lstlisting}
\inlinemedik{sleep} results in a JSON message being sent to a remote
endpoint (lines 1-5) specified when the \MediK{} process is launched. This mimics
sending an event to an external \emph{timer} machine, with the desired duration
as the payload. At the remote endpoint,
code must be provided (in any programming language) to parse the message, and
respond with a JSON message indicating the expiration of the timer
once the desired duration has passed. A unique transaction-id (lines 4, 7, 9),
which the code at the endpoint is expected to provide in the response,
uniquely identifies the machine instance being responded to.
\inlinemedik{sleep} causes a context-switch to occur on line 6, releasing the
executor lock to process other \emph{scheduled} machines.

When a message signifying the expiration of the timer is sent to
the \MediK{} process, along with the transaction id of source instance,
the corresponding event signalling the completion of the sleep statement
is placed at the \emph{beginning} of the source machine instance's input buffer,
and the instance is scheduled to resume execution in the next epoch.
The following rule handles the external response:
\begin{lstlisting}[
  style=ksty
  ,language=k
  ,basicstyle=\ttfamily\scriptsize
  ,numbers=left
  ,numberstyle=\tiny
  ,framexleftmargin=1.5em
  ,xleftmargin=2em
  ]
rule
<k> waitForSleepResponse(TId) => . ... </k>
<inBuffer>
  (ListItem(event($SleepDone | TId | Tau ))
 => .List)  ...
</inBuffer>
<executorAvailable>
  true => false
</executorAvailable>
<epoch> Epoch </epoch>
  requires Tau <=Int Epoch
\end{lstlisting}
The \inlinemedik{waitForSleepResponse(TId)}
blocks execution until the external response indicating
the expiration of the sleep timer is received in the input buffer (line 4).
Once the response is received, the machine instance resumes execution when
\begin{enumerate*}[label=(\alph*)]
  \item the execution lock becomes available
  (indicated by \inlinemedik{true} on line 8), and,
  \item the epoch the instance was scheduled in (line 4) is less than or equal to the current
  epoch (lines 10-11).
\end{enumerate*}

An \inlinemedik{obtainFrom} statement also results in a context switch.
Just as in the case of sleep, a json message is sent to the remote
endpoint, while the machine instance release the execution lock, and waits for a
response. Once data for the requested field is available, it's communicated as an
event to the \MediK{} process, and the machine resumes execution.
%
%\begin{lstlisting}[style=ksty,language=k,basicstyle=\ttfamily\scriptsize,numbers=right]
%rule
%<k> obtainFrom(instance(Id:Int), Field:String)
%    =>   jsonWrite( { "name"      : "Obtain"
%                    , "args"      : [ Field , .JSONs ] } ...)
%      ~> releaseExecutor
%      ~> waitForObtainResponse(TId) ...
%   </k>
%   <tidCount> TId => TId +Int 1 </tidCount>
%\end{lstlisting}


\section{Evaluation}\label{sec:evaluation}

\subsection{Sepsis Management \CDSS}

To evaluate our approach, we collaborated with the Children's Hospital
of Illinois at OSF St. Francis Medical Center to develop a \MediK{}-based
\CDSS{}
for screening and management of Pediatric Sepsis \footnote{the entire \CDSS{}
for sepsis management is available at \cite{psepsis-url}.}.

Recall from \figurename{} \ref{fig:sepsis-screening} the guideline
for sepsis screening.
In \figurename{} \ref{fig:medik-sepsis-screening}, we show
\MediK{} code corresponding to the sepsis screening guideline.
When modeled in \MediK{}, a flowchart in the guideline is represented using
a \MediK{} machine. \emph{Nodes} in the flowchart are represented as
\emph{states} in a \MediK{} machine, while flowchart \emph{edges} as \emph{state-transitions}.
Note that we use \emph{node} to refer to constructs
in the flowchart, and \emph{state} to refer to counterparts in \MediK{}.
Also, while it's desirable to represent each flowchart \emph{node} as a
state machine \emph{state}, the task in the flowchart \emph{node} may warrant using
multiple state-machine \emph{states}. For example,
in \figurename{} \ref{fig:sepsis-screening}, the step \say{Obtain Patient Age,
Weight, and High Risk Conditions} is translated to states \inlinemedik{ObtainAge} (lines
7-13), \inlinemedik{ObtainWeight} (line 14), and
\inlinemedik{ObtainHighRiskConditions} (line 15) in \figurename{} \ref{fig:medik-sepsis-screening}.
Within each of these states, the code
permits communication with heterogeneous external agents for obtaining
required parameters. For instance, on line 9, an
\inlinemedik{Instruct} event is sent to an external \inlinemedik{tablet} machine
with the payload \inlinemedik{"get age"}. The recipient process
runs on a tablet held by the Healthcare Provider, and handles
the event by prompting the provider to enter the patient's age.
A \inlinemedik{ConfirmAgeEntered} event, emitted once the age
is obtained, enables the screening machine to proceed to the next
step (lines 11-13). Once all appropriate measurements have been obtained,
they are checked for abnormality (lines 18-26) using tables shown in \figurename{}
\ref{fig:hr-check-fun} to arrive upon a diagnosis.

\begin{figure}
  \centering
\begin{lstlisting}[style=mediksty
,language=medik
,basicstyle=\ttfamily\scriptsize
,numbers=left
,numberstyle=\tiny
,framexleftmargin=1.5em
,xleftmargin=2em
]
machine SepsisScreening receives .. {
  init state Start {
    on StartScreening do {
      goto ObtainAge;
    }
  }
  state ObtainAge {
    entry {
      send tablet, Instruct, ("get age");
    } on ConfirmAgeEntered do {
      goto ObtainWeight;
    }
  }
  state ObtainWeight { ... }
  state ObtainHighRiskConditions { ... }
  state CalculateScore {
      var hrAbnormal = !isInNormalRange("HR", ...);
      var bucket1    = hrAbnormal ||  ...
      var bucket3    = mentalStatusAbnormal || ...

      var sepsisSuspected
        = bucket1 && bucket2 && bucket3;

      send tablet, SepsisDiagnosis
        , (sepsisSuspected);
  }
}
\end{lstlisting}
  \caption{Sepsis Screening in \MediK{}}\label{fig:medik-sepsis-screening}
\end{figure}

%Apart from structurally resembling the workflow, the code
%permits communication with heterogenous external agents, and
%\emph{obtains} patient parameters such as age and weight, and
%uses them to evaluate the patient for sepsis. For instance on line 9,
%we send an event to the external machine that is the tablet held by the \HCP{},
%to obtain the patient's age. Once the age is entered, an event
%\inlinemedik{ConfirmAgeEntered} is sent to the machine, prompting the machine to
%move on to collecting the patient's weight.
% When evaluating whether the patient
%meets the criteria for sepsis, we use functions, such as the one on line 18,
%that obtain the value from a sensor, and check whether
%it lies in acceptable ranges using tables
%indexed by the patient's age and weight, similar to the paper-based \BPG{}.
%Treatment machines \emph{receive} events from
%the screening machine, and are triggered if the diagnosis is \textit{sepsis
%positive}, suggesting commencement of treatment.

\begin{figure}[b!]
  \centering
\begin{lstlisting}[style=mediksty
,language=medik
,basicstyle=\ttfamily\scriptsize
,numbers=left
,numberstyle=\tiny
,framexleftmargin=1.5em
,xleftmargin=2em
]
machine FluidTherapy
  receives StartFluidTherapy, ... {


  init state Start {
    on StartFluidTherapy do {
      goto ObtainRisks;
    }
  }

  state ObtainRisks {
    // Obtain fluid overload related risks
  }

  state SuggestFluidDosage {
    // Suggest a dosage based on risks
  }

  state WaitForAdministerFluidConfirmation {
    // Handler for Normal Saline Administration
    on ConfirmNormalSalineAdministered do {
      sleep(900);
      goto EvaluateResponsiveness;
    }
  }

  state EvaluateResponsiveness {
    entry {
      send tablet
        , Instruct
        , ("get responsiveness to fluids");
    }

    on FluidResponsivenessEntered(responsiveness) do {
      isResponsiveToFluids = responsiveness;
      goto ObtainFluidOverloadSigns;
    }
  }
  state ObtainFluidOverloadSigns {
    // Obtain signs of fluid overload
  }

  state AskNextStep {
    entry {
      var recommendation;
      if (this.fluidOverload) {
        recommendation = "handle fluid overload";
      } else {
        // obtain total saline dose
        if ((totalSalineDose >
              measurementBounds.salineDosageUpperBound) {
          if (isResponseiveToFluids) {
            recommendation = "maintainence fluids"
          } else {
            recommendation = "consider inotropic support";
            broadcast ConsiderInotropicSupport;
          }
        } else {
          recommendation = "repeat fluid bolus";
        }
      }
      // Send recommendation to tablet
      // Wait for HCP response
    }
  }
}
\end{lstlisting}
  \caption{Fluid Resuscitation in \MediK{}}\label{fig:medik-fluid-therapy}
\end{figure}


Recall from Section \ref{sec:motivating-example} that once a sepsis diagnosis has been arrived upon,
one of the guideline suggested actions include administering fluids as shown in
\figurename{} \ref{fig:fluid-therapy}. In \figurename{} \ref{fig:medik-fluid-therapy},
we show the corresponding \MediK{} code for administering fluids.
The process starts when an external \inlinemedik{StartFluidTherapy} event,
corresponding to a button press by the \HCP{} is received (line 6).
The next steps include
\begin{enumerate*}[label=(\alph*)]
  \item obtaining any \emph{risks} associated with administering
    fluids (lines 11-13),
  \item suggesting an appropriate \emph{dose} to administer based
    on the risks, if any (lines 15-17), and,
  \item waiting for the \HCP{} to confirm that the suggested dose was
    administered (line 21).
\end{enumerate*}
Once the dose is administered, the machine waits for the
for 15 minutes as specified by the guideline (line 22), before prompting
the \HCP{} to evaluate the patient's responsiveness to the administered fluid
dose (lines 27-38), and check for any signs of fluid overload (lines 39-41).
If the patient exhibits any signs of fluid overload, then a recommendation
to handle the overload is made (line 47). Otherwise,
the total dose of administered fluid is obtained from an external source (line 50).
If the total dose is above the maximum allowed dose, then
a recommendation based on the patient's responsiveness to administered
fluids is made to either
\begin{enumerate*}[label=(\alph*)]
  \item reduce the fluid flow to maintenance levels (line
53), or,
  \item switch to inotropic support to address circulatory issues is made (lines 52-58).
\end{enumerate*}
If the total dose of administered
fluids is less than the maximum allowed limit, then a recommendation to
administer one more fluid bolus made (lines 58-60).

Note that both the \inlinemedik{SepsisScreening} and
\inlinemedik{FluidTherapy} machines structurally resemble their
paper based counterparts in \figurename{} \ref{fig:sepsis-screening} and
\figurename{} \ref{fig:fluid-therapy} respectively, making
it easier for Healthcare Providers to comprehend and validate
the code.



\subsection{Formal Analysis using \MediK{}}

During execution of a \MediK{} program, a machine may be considered
\emph{stuck} if an event at the head of its input buffer does not
have an associated handler, rendering said machine non-responsive. For this
reason, languages for modeling large concurrent systems, such as P
\cite{DesaiPLDI13} raise an exception for unhandled events. To mitigate
such exceptions, we can enforce every machine to define event handlers
for all possible events in all states, and use static analysis to detect possible violations.
But, for \MediK{} programs, we found that for complex \CDSSs{},
such as the one for screening and management of sepsis:
\begin{enumerate*}[label=(\alph*)]
  \item it's tedious and error prone to define handlers for every event in every
    state, and,
  \item it reduces the comprehensibility of the program, as many spurious
    event handlers that may never fire during execution have to be specified.
\end{enumerate*}

Thus, for \MediK{}, we employ a weaker notion of responsiveness. We verify that
every event that a state may possibly receive during execution must have a handler defined for
it. This presents a challenge for reactive systems, or systems involve interactions
with the external world, such as \MediK{}-based \CDSSs, as exploring
the system's state space requires modeling the external components.
In \MediK{}, we address this by specifying external components
as \emph{ghost} machines - a technique also used by other state machine
formalisms such as P \cite{DesaiPLDI13}.
For program analysis, \emph{ghost} machines substitute external agents,
permitting exploration of the state space.
During execution, \emph{ghosts} are discarded and replaced by actual external agents.
Due to this, ghosts machines may have statements to express non-determinism
in processes. Consider, for instance, on a positive sepsis diagnosis, a
\HCP{} may chose to either administer fluids first, followed by antibiotics, or
vice-versa. \MediK{} supports such non-determinism using \inlinemedik{either-or}
statements as follows:
\begin{lstlisting}[style=mediksty
,language=medik
,basicstyle=\ttfamily\scriptsize
,numberstyle=\tiny
,framexleftmargin=1.5em
,xleftmargin=2em
]
either {
  broadcast StartFluidTherapy;
  broadcast StartAntibioticTherapy;
} or {
  broadcast StartAntibioticTherapy;
  broadcast StartFluidTherapy;
}
\end{lstlisting}

When writing ghosts, values of measurements need to be abstract, to
encompass all possible values that may be encountered during execution.
For instance, when modeling entering a parameter such as the Heart Rate,
we need to use an abstract value, representing all possible concrete values.
To this end, we allow using an abstract value \inlinemedik{#nondet} in ghost machines,
with the following abstract semantics:
\begin{lstlisting}[
  style=ksty
  ,language=k
  ,basicstyle=\ttfamily\scriptsize
  ,numbers=left
  ,numberstyle=\tiny
  ,framexleftmargin=1.5em
  ,showspaces=false
  ,xleftmargin=2em
  ]
  rule #nondet + _:Val    => #nondet
  rule _:Val   <= #nondet => #nondet
  rule #nondet && _       => #nondet
  rule if (#nondet) Block => Block
  rule if (#nondet) _     => .
\end{lstlisting}

The use of abstract encodings leads to a reduction of the state space.
Recall from Section \ref{sec:motivating-example} that we needed to:
\begin{enumerate*}[label=(\arabic*)]
  \item utilize patient's basic information such as age and weight to calculate normal ranges
    for clinical measurements such as blood pressure and heart rate, and,
  \item calculate abnormality in clinical measurements using aforementioned
    ranges.
\end{enumerate*}
For example, determining whether the patient's heart rate is abnormal
is performed using the \inlinemedik{in-interval} construct as show in \figurename{} \ref{fig:hr-check-fun}.
Recall that the \inlinemedik{in-interval} construct is merely syntatic
sugar for nested \inlinemedik{if-else} statements. When using ghost machines for
model checking, since the actual measurement is an abstract value, we know the
final result of this abnormality checking operation is an abstract boolean value. Thus,
instead of exploring each branch of \inlinemedik{if-else} statements
corresponding to \inlinemedik{in-interval} constructs in \figurename{}
\ref{fig:hr-check-fun}, we replace the entire checking process with an
final abstract boolean value.
This reduces the state space but still allows us to explore all treatment options for both the normal and abnormal cases.

\subsection{Model Checking the Sepsis \CDSS{}}
To verify responsiveness of the Sepsis \CDSS{}, we implemented ghost machines
for the external components using support for non-determinism and abstract
values. We then added the following rule to the semantics, that takes
a machine in an active state with an unhandled event at the
head of the input buffer to a terminal \inlinemedik{stuck} state.

\begin{lstlisting}[
  style=ksty
  ,language=k
  ,basicstyle=\ttfamily\scriptsize
  ,numbers=left
  ,numberstyle=\tiny
  ,framexleftmargin=1.5em
  ,showspaces=false
  ,xleftmargin=2em
  ]
  rule
  <k> handleEvents ~> _ => stuck </k>
   <activeState> ActiveState </activeState>
   <class> MachineName </class>
   <inBuffer>
    ListItem(event(InputEvent | _ | _ )) ...
   </inBuffer> ...
  </k>
\end{lstlisting}
We utilize the semantics-generated bounded model checker
to search the state space to a depth of 300,000
for a \inlinemedik{stuck} pattern, i.e., a machine that's no longer responsive.
This depth we used was adequate for a complete run of the
both the fluid and antibiotics machines simultaneously. The search command
was executed on a machine with 64 GB of memory, and took roughly 90 minutes, and
reported no such state was possible. To the best of our knowledge, this makes
ours the first system for screening and management of sepsis with some formal
safety guarantees.

\section{Discussion}\label{sec:discussion}

\chapter{Future Work}

This chapter outlines major tasks that remain towards enhancing the capabilities
of our system.

\section{Soundness of Model Checking Abstraction}\label{sec:abstraction-soundness}

In \autoref{sec:formal-analysis}, we introduced an abstraction that enabled us
to model check the system. Instead of interpreting $\MediK{}$ programs
concretely, we introduced rules that operate over abstract values
that hold semantic meaning. However, we did not establish that the analysis is
indeed sound. In this section, we briefly outline the steps to prove the
soundness of our abstraction-based analysis, and outline limitations that
prevented us from establishing it.

Recall that in \autoref{sec:theoretical-foundations}

\chapter{Conclusion}\label{chapter:conclusion}

Chapter \ref{chapter:introduction} explains the persistent challenge
that preventable medical errors (\PMEs{}) present in medicine.
The seriousness of the problem can be gauged by the fact that
President's Council on Science and Technology (\PCAST{})
declared the prevalence of \PMEs{} to be an
\say{urgent national public health issue}.

Among the solutions
outlined in the PCAST{} report include:
\begin{enumerate*}[label=(\roman*)]
  \item Utilizing \emph{evidence-based} medicine.
  \item Investing in technology that assists healthcare
  practitioners (\HCPs{}) in decision making.
\end{enumerate*}
Clinical best practice guidelines (\BPGs{}) exemplify evidence-based medicine.
Produced by hospital and medical associations, these guidelines
are evidence-based statements that codify recommended treatment for
various clinical scenarios, and growing evidence supports their use
in medicine to reduce \PMEs{}. But their effectiveness in practice is often
limited by challenges in their uptake and adoption. As outlined in
\autoref{chapter:hurdles-cdss-adoption}, integrating \BPGs{} into \HCP{} workflows
and following them in practice is often difficult, resulting in treatment
that deviates from prescribed best-practices, and as a result, leads to
worse outcomes. Guidelines-based clinical decision support systems
that assist \HCPs{} with situation-specific advice to enhance guideline
adherence have hence been shown to improve patient outcomes.

Despite many apparent benefits, the prevalence of \CDSSs{} in practice
remains low, due to factors outlined in \autoref{chapter:hurdles-cdss-adoption}.
In this work, we utilize a semantics-first approach to building safe clinical
decision support systems.

By semantics-first, we mean that:
\begin{itemize}
  \item The semantics of medical knowledge are accurately captured in the
  system.
  \item The language for expressing medical knowledge has a well-defined, formal
  semantics from which tools for execution and formal analysis are derived.
\end{itemize}

\CDSSs{} are typically developed by software developers lacking expertise in
medicine. Thus, it is possible for medical knowledge in a \CDSS{} to be incorrectly
encoded during development. As \CDSSs{} operate in safety critical settings, such
bugs can lead to adverse outcomes for patients. To address this,
we developed a domain specific language for describing medical knowledge
called \MediK{}. Developed as an academic endeavor, \MediK{} has been designed to be
comprehensible to \HCPs{}. This allows \HCPs{} to examine and validate the accuracy
medical knowledge in a \MediK-based \CDSS{}, enabling greater transparency in
the source of the \CDSS{}'s recommendations.

As outlined in \autoref{chapter:related-work}, the need for \DSLs{} for
expressing \BPGs{} to have formal semantics and support for comprehensive
support for formal analysis has already been recognized. \MediK{} builds
on the state-of-art in modeling concurrent systems and medical knowledge
to enable complex medical guidelines with multiple concurrent workflows
and interdependencies to be conveniently expressed. To demonstrate this,
we collaborate with OSF St. Francis Hospital in Peoria, Illinois--a
major pediatric hospital to develop a \CDSS{} for their pediatric
sepsis management guidelines. During the development process, bugs in
medical logic were uncovered by collaborating physicians by examining
\MediK{} code, and learnings from our endeavor were incorporated into the
design of the language.

We believe the work outlined in this thesis presents the first steps towards
addressing challenges inhibiting broader \CDSS{} uptake. But a comprehensive
solution involves addressing several more challenges, some of which are
outlined \autoref{chapter:future-work}. Some of these challenges demand
collaboration from multiple stake-holders, from both computer-science and
medicine. Thus, they remain unaddressed for now. But we hope addressing
them, as part of building on top of work presented here, can lower \PMEs{},
leading to better quality healthcare at reduced costs.


\bibliographystyle{IEEE_ECE}
\bibliography{references}  % Put references in BibTeX format in thesisrefs.bib.

\end{document}
