\chapter{\MediK{}: Towards Safe Guidelines-based \CDSSs{}}

In \autoref{chapter:k-based-guidelines}, we discussed the process
of encoding best practice guidelines (\BPGs{}) as \K{} definitions
to describe a systematic way of building \CDSSs{}. Specifically,
we described the process of encoding medical knowledge in guidelines
via notionally via workflows that can be expressed abstractly as
concurrently executing finite state machines that communicate
via passing messages. Next, we described the process of
expressing state machines as \K{} definition, where \K{} features
such as configuration abstraction and local rewriting
enable \emph{conciseness}. We then combined the \K{} definition
with a simple javascript-based frontend to develop a \CDSS{}
for assisting healthcare practitioners (\HCPs{}) follow the
Advanced Life Support (\ALS{}) guidelines for managing
cardiac arrest published by the American Heart Association.

In \autoref{chapter:hurdles-cdss-adoption}, we described
that wider \CDSS{} adoption is incumbent on a having a
systematic way of developing guidelines with validated
content. The approach described in \autoref{sec:semantics-first}
attempts to enable such a way. It dictates that:
\begin{enumerate*}[label=(\roman*)]
  \item the semantics of the language be formally defined, from
  which tools are derived from a correct-by-construction fashion, and,
  \item ensuring that semantics of medical knowledge is accurately
  described.
\end{enumerate*}
As is typically the case, the \KACLS{} system from
\autoref{sec:kacls-cdss} through collaboration between
\HCPs{} and computer scientists. While the \K{}-based
representation of the \ALS{} guideline was concise,
it was not easily comprehensible to \HCPs{}, or to other
software engineers without prior experience of using \K{}.


