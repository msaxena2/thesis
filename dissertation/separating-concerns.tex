\chapter{Separating Concerns: Modular and Safe Clinical Decision Support}\label{chapter:separating-concerns}

In \autoref{sec:cdss-components}, we briefly mentioned components that
conceptually make up a \CDSS{}. This chapter describes the motivations
behind decomposing the system as such. First, we describe where
\CDSSs{} may loosely fit within the larger context of control
theory and motivate the need for tailored architectures for \CDSSs{}.
Then we briefly identify some unique challenges of encoding medical knowledge.
Finally, we talk about encoding medical knowledge directly through
rewriting, and associated issues.

\section{Control Theory and \CDSSs{}}
Control theory deals with the design and analysis of \emph{closed-loop}
systems, i.e., systems where the inputs are affected at-least in part by
by outputs. Typically, such systems are characterized by:
\begin{itemize}
  \item The \emph{plant} represents the part of the system \emph{to be
  controlled}.
  \item The \emph{control} or \emph{compensator} is the part of the system
  that provides \emph{satisfactory characteristics} or \emph{regulation}
  \cite{SimrockTR08}.
\end{itemize}

When viewed from the lens of a closed-loop system, the patient behaves
as the plant, and the goal is optimizing patient outcomes
\cite{SongSMC23}\footnote{Joint work with Song et al.}.

