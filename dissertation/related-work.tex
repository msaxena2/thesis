\chapter{Related Work}

In chapter \ref{chapter:hurdles-cdss-adoption}, we provided a brief
overview of existing approaches and their limitations. This chapter
provides a comprehensive discussion of related approaches. Recall
from section \ref{sec:modular-architectures} that implementing a guidelines-based
clinical decision support system requires collaboration between
experts in medicine and software engineers for knowledge formalization, i.e.,
the process of encoding medical knowledge in textual
\BPGs{} in some programming language. Using a conventional programming
language for knowledge formalization can lead to an inaccurate
encoding medical knowledge, as experts in medicine, being unaccustomed
to computer programming, cannot validate the accuracy of the encoding.
To address this, \DSLs{} designed specifically for expressing
\BPGs{} in a computer-interpretable format are utilized. \BPGs{} expressed
in such languages can serve as both textual guideline documents and knowledge
bases in \CDSSs{}. But, medical knowledge in \BPGs{} has also been expressed
and formalized using other non domain-specific approaches. The discussion in
this chapter has also been split along the same lines. In section
\ref{sec:dsl-based-approaches}, we discuss approaches involving \DSLs{} for
computer interpretable guidelines. In \ref{sec:general-approaches}, we
discuss other approaches that aren't specific to \BPGs{}.

\section{\DSL{}-based Approaches}\label{sec:dsl-based-approaches}

\subsection{Arden Syntax}\label{sec:arden-syntax}

The Arden Syntax is among the earliest and most widely-used
standards for expressing medical logic, with the first
draft of the standard appearing in 1989.
It was also among the earliest attempts to create a domain
specific language specifically for use in \CDSSs{} \cite{SamwaldJBI12}.

In Arden Syntax, code is organized into self-contained medical
logic modules (\MLMs{}) that have a well-defined structure to
separate higher-level medical logic from low-level implementation
details such as variable declarations. The language continues to
evolve to accommodate diverse uses, and is supported by multiple execution
engines \cite{AnandMed04,KaradimasAMIA02}.

The maturity of the Arden Syntax platform is reflected in its
use to implement a large and diverse set of \CDSSs{}. These
include:
\begin{itemize}
  \item Systems for monitoring and infections surveillance \cite{BlackyACI12,SteinbrecherDC02}.
  \item Treatment planning and decision support \cite{EngeleHealth11,BoeglAMIA05}.
\end{itemize}

Being one of the earliest \DSLs{} specifically designed for \CDSSs{}, the
Arden Syntax has found widespread adoption. But, the language also has several
limitations. Notably:
\begin{itemize}
  \item Support on simple, independent guidelines instead of complex treatment
  workflows \cite{ClerqAIM03}.
  \item Lack of formal semantics and clarity in the language specifications \cite{SamwaldJBI12}.
\end{itemize}

\subsection{\GEODECM{}}\label{sec:geodecm}

Guided Entry of Data Elements for Clinical Management (\GEODECM{}) was
an early \DSL{} developed by the Decision Systems Group at Harvard University
Medical School. The \GEODECM{} \DSL{} utilized a
state machine-based architecture to represent medical knowledge.
Clinical problems in \GEODECM{} were broken down into clinical management
states, where entry, exit and transitions between states were determined
by data collected during execution \cite{StoufletJAMIA96}. Each
clinical management state was represented as state machine nodes,
and edges between nodes represented transitions between the various clinical
mangement states \cite{MachadoJAMIA98}.

\subsection{EON}\label{sec:eon}

The EON langauge, developed at Stanford University, also
utilizes a state machine-based architecture for representing medical guidelines.
Treatment protocols are first recursively composed into smaller granular
elementary protocols that cannot be decomposed any further.
The protocols are then represented via directed multigraphs, where
nodes represent both patient and treatment state, and transitions represent
actions or changes in patient conditions \cite{TuAMIA96}.

EON{} has many similarities to \GEODECM{}. Medical knowledge in both \DSLs{}
is expressed using a finite-state machine notation, where nodes are
patient or treatment states and edges represent actions or changes in state.
However, EON, unlike \GEODECM{}:
\begin{itemize}
  \item Has an informal, yet comprehensive operational semantics.
  \item Support for sequencing, looping and synchronization constructs to support
    guidelines with multiple concurrent actions \cite{TuAMIA96}.
\end{itemize}

While Arden Syntax emphasizes expressing medical logic using independent, modular
medical logical modules, EON state machines are more tightly coupled, enabling
representation of complex protocols and guidelines \cite{TuAMIA96}. EON has
been used to implement complex \CDSSs{} for management of conditions such as
AIDS \cite{MusenJAMIA96} and Breast Cancer \cite{TuAMIA96}.

\subsection{\GLIF{}}\label{sec:glif}

The Guideline Interchange Format (\GLIF{}) was introduced in 1998 as a
result of a collaboration between researchers from Columbia University,
Harvard University and Stanford \cite{ClerqAIM03}. Good medical
guidelines improve healthcare quality, but require substantial work
to develop and maintain. At the time, guidelines were published through
unstructured text documents that were not easily shareable. The motivation
behind the \GLIF{} project was to enable \BPG{} between institutions sharing
through the development of:
\begin{itemize}
  \item A standardized electronic format for rapid dissemination.
  \item A repository of guidelines to prevent duplicated effort.
  \item Tools that \HCPs{} could utilize to retrieve and
    execute relevant guidelines.
  \item Analysis tools that enabled authors to develop
    and publish high-quality, unambiguous guidelines.
\end{itemize}

The \GLIF{} team comprised of research groups from various universities
where other languages, including Arden Syntax (section \ref{sec:arden-syntax}),
\GEODECM{} (section \ref{sec:geodecm}) and EON (section \ref{sec:eon}),
were developed. Thus, perspectives and learnings from existing efforts guided
the development of \GLIF{}, including borrowing certain useful constructs
directly from EON \cite{MachadoJAMIA98}

In \GLIF{}, guidelines are expressed using the \GLIF{} class containing
relevant attributes corresponding to treatment information. One such
attribute is an unordered list of all guideline steps. The guideline
specification itself is a directed graph defined over collection of
said steps. A step can be one of the following:
\begin{itemize}
    \item Action steps: Specify clinical-care actions during treatment.
    \item Conditional steps: Determine control flow based on logical statements.
    \item Branch steps: Enable flow to multiple guideline steps concurrently.
    \item Synchronization steps: Converge concurrent flow back to a single
      guideline step \cite{MachadoJAMIA98}.
\end{itemize}
\GLIF{} has been used to implement several \CDSSs{}, including
systems for depression screening and management \cite{ChoiJMI07} and
hyperkalemia patient screening \cite{WangBook04}.

\subsection{PROforma}\label{sec:proforma}

PROforma was initially developed at the Cancer Research UK Advanced Computation
Laboratory, and has been utilized to develop several \CDSSs{} \cite{ClerqAIM03}.
In PROforma, guidelines are modeled as tasks and data items. Tasks are
hierarchically organized into plan, and may be further divided into:
\begin{itemize}
  \item Actions: Procedures that must be performed in an external environment.
  \item Enquiries: Guideline points at which data must be obtained.
  \item Decisions: Points at which a choice determines further control flow
    \cite{SuttonAMIA03}.
\end{itemize}

PROforma guidelines can be visualized as directed graphs, with nodes representing
actions and edges constraints on control flow. Pictorially,
actions are represented as squares, enquiries as diamonds and
decisions as circles. This enables PROforma guidelines' visual representations
to resemble flowcharts and medical algorithms that \HCPs{} are already familiar
with.

A PROforma task has associated properties that determine how it's interpreted.
These include:
\begin{itemize}
  \item Captions and descriptions that improve guideline comprehensibility.
  \item Preconditions that specify conditions for a task's execution to begin.
  \item Scheduling constraints that enable synchronization between concurrent
    tasks.
\end{itemize}
Additionally, plans have termination and abort conditions that specify
successful end of the current plan to facilitate execution of successor
plans.

Using PROforma requires a suite of tools called Tallis \cite{TallisUrl}, written in Java,
that include a composer for creating and viewing guidelines, a tester for
debugging them, and an execution engine for running them \cite{SuttonAMIA03}.

\section{General Approaches}\label{sec:general-approaches}




