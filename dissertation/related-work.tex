\chapter{Related Work}

In chapter \ref{chapter:hurdles-cdss-adoption}, we provided a brief
overview of existing approaches and their limitations. This chapter
provides a comprehensive discussion of related approaches. Recall
from section \ref{sec:modular-architectures} that implementing a guidelines-based
clinical decision support system requires collaboration between
experts in medicine and software engineers for knowledge formalization, i.e.,
the process of encoding medical knowledge in textual
\BPGs{} in some programming language. Using a conventional programming
language for knowledge formalization can lead to an inaccurate
encoding medical knowledge, as experts in medicine, being unaccustomed
to computer programming, cannot validate the accuracy of the encoding.
To address this, \DSLs{} designed specifically for expressing
\BPGs{} in a computer-interpretable format are utilized. \BPGs{} expressed
in such languages can serve as both textual guideline documents and knowledge
bases in \CDSSs{}. But, medical knowledge in \BPGs{} has also been expressed
and formalized using other non domain-specific approaches. The discussion in
this chapter has also been split along the same lines. In section
\ref{sec:dsl-based-approaches}, we discuss approaches involving \DSLs{} for
computer interpretable guidelines. In \ref{sec:general-approaches}, we
discuss other approaches that aren't specific to \BPGs{}.

\section{\DSL{}-based Approaches}\label{sec:dsl-based-approaches}

\subsection{Arden Syntax}

The Arden Syntax is among the earliest and most widely-used
standards for expressing medical logic, with the first
draft of the standard appearing in 1989.
It was also among the earliest attempts to create a domain
specific langauge specifically for use in \CDSSs{} \cite{SamwaldJBI12}.

In Arden Syntax, code is organized into self-contained medical
logic modules (\MLMs{}) that have a well-defined structure to
separate higher-level medical logic from low-level implementation
details such as variable declarations. The language continues to
evolve to accomodate diverse uses, and is supported by multiple execution
engines \cite{AnandMed04}.

\section{General Approaches}\label{sec:general-approaches}






