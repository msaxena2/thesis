\chapter{Related Work}

In chapter \ref{chapter:hurdles-cdss-adoption}, we provided a brief
overview of existing approaches and their limitations. This chapter
provides a comprehensive discussion of related approaches. Recall
from section \ref{sec:modular-architectures} that implementing a guidelines-based
clinical decision support system requires collaboration between
experts in medicine and software engineers for knowledge formalization, i.e.,
the process of encoding medical knowledge in textual
\BPGs{} in some programming language. Using a conventional programming
language for knowledge formalization can lead to an inaccurate
encoding medical knowledge, as experts in medicine, being unaccustomed
to computer programming, cannot validate the accuracy of the encoding.
To address this, \DSLs{} designed specifically for expressing
\BPGs{} in a computer-interpretable format are utilized. \BPGs{} expressed
in such languages can serve as both textual guideline documents and knowledge
bases in \CDSSs{}. But, medical knowledge in \BPGs{} has also been expressed
and formalized using other non domain-specific approaches. The discussion in
this chapter has also been split along the same lines. In section
\ref{sec:dsl-based-approaches}, we discuss approaches involving \DSLs{} for
computer interpretable guidelines. In \ref{sec:general-approaches}, we
discuss other approaches that aren't specific to \BPGs{}.

\section{\DSL{}-based Approaches}\label{sec:dsl-based-approaches}

\subsection{Arden Syntax}\label{sec:arden-syntax}

The Arden Syntax is among the earliest and most widely-used
standards for expressing medical logic, with the first
draft of the standard appearing in 1989.
It was also among the earliest attempts to create a domain
specific language specifically for use in \CDSSs{} \cite{SamwaldJBI12}.

In Arden Syntax, code is organized into self-contained medical
logic modules (\MLMs{}) that have a well-defined structure to
separate higher-level medical logic from low-level implementation
details such as variable declarations. The language continues to
evolve to accommodate diverse uses, and is supported by multiple execution
engines \cite{AnandMed04,KaradimasAMIA02}.

The maturity of the Arden Syntax platform is reflected in its
use to implement a large and diverse set of \CDSSs{}. These
include:
\begin{itemize}
  \item Systems for monitoring and infections surveillance \cite{BlackyACI12,SteinbrecherDC02}.
  \item Treatment planning and decision support \cite{EngeleHealth11,BoeglAMIA05}.
\end{itemize}

Being one of the earliest \DSLs{} specifically designed for \CDSSs{}, the
Arden Syntax has found widespread adoption. But, the language also has several
limitations. Notably:
\begin{itemize}
  \item Support on simple, independent guidelines instead of complex treatment
  workflows \cite{ClerqAIM03}.
  \item Lack of formal semantics and clarity in the language specifications \cite{SamwaldJBI12}.
\end{itemize}

\subsection{Dilemma}\label{sec:dilemma}

The Dilemma project was among the first attempts at
standardizing \BPGs{} to originate out of Europe. In Dilemma, a
protocol describes the set of activities that must be carried out
in order to reach on objective based on clinical indications \cite{HerbertMPB95}.
Protocols are recursively broken into smaller elementary protocols. Each
protocol has one or more associated indication that determine when
it becomes applicable, and a set of objectives that define the expected
outcome.

Applying a protocol to a patient state is defined as a procedure, where
leaf-level procedures are specifically referred to as activities. Activities
are carried out by agents (usually \HCPs{} such as doctors, nurses, etc.),
and typically involve:
\begin{itemize}
  \item Maintaining or changing patient state through treatment steps.
  \item Obtaining information or confirming procedures.
  \item Communication or interaction with other agents.
\end{itemize}

Dilemma was utilized to implement guidelines from several European
countries, such as primary care guidelines from the Netherlands and the UK and cancer
treatment guidelines from France \cite{HerbertMPB95}.

\subsection{\GEODECM{}}\label{sec:geodecm}

Guided Entry of Data Elements for Clinical Management (\GEODECM{}) was
an early \DSL{} developed by the Decision Systems Group at Harvard University
Medical School. The \GEODECM{} \DSL{} utilized a
state machine-based architecture to represent medical knowledge.
Clinical problems in \GEODECM{} were broken down into clinical management
states, where entry, exit and transitions between states were determined
by data collected during execution \cite{StoufletJAMIA96}. Each
clinical management state was represented as state machine nodes,
and edges between nodes represented transitions between the various clinical
mangement states \cite{MachadoJAMIA98}.

\subsection{EON}\label{sec:eon}

The EON langauge, developed at Stanford University, also
utilizes a state machine-based architecture for representing medical guidelines.
Treatment protocols are first recursively composed into smaller granular
elementary protocols that cannot be decomposed any further.
The protocols are then represented via directed multigraphs, where
nodes represent both patient and treatment state, and transitions represent
actions or changes in patient conditions \cite{TuAMIA96}.

EON{} has many similarities to \GEODECM{}. Medical knowledge in both \DSLs{}
is expressed using a finite-state machine notation, where nodes are
patient or treatment states and edges represent actions or changes in state.
However, EON, unlike \GEODECM{}:
\begin{itemize}
  \item Has an informal, yet comprehensive operational semantics.
  \item Support for sequencing, looping and synchronization constructs to support
    guidelines with multiple concurrent actions \cite{TuAMIA96}.
\end{itemize}

While Arden Syntax emphasizes expressing medical logic using independent, modular
medical logical modules, EON state machines are more tightly coupled, enabling
representation of complex protocols and guidelines \cite{TuAMIA96}. EON has
been used to implement complex \CDSSs{} for management of conditions such as
AIDS \cite{MusenJAMIA96} and Breast Cancer \cite{TuAMIA96}.

\subsection{\GLIF{}}\label{sec:glif}

The Guideline Interchange Format (\GLIF{}) was introduced in 1998 as a
result of a collaboration between researchers from Columbia University,
McGill University, Harvard University and Stanford \cite{ClerqAIM03}. Good medical
guidelines improve healthcare quality, but require substantial work
to develop and maintain. At the time, guidelines were published through
unstructured text documents that were not easily shareable. The motivation
behind the \GLIF{} project was to enable \BPG{} between institutions sharing
through the development of:
\begin{itemize}
  \item A standardized electronic format for rapid dissemination.
  \item A repository of guidelines to prevent duplicated effort.
  \item Tools that \HCPs{} could utilize to retrieve and
    execute relevant guidelines.
  \item Analysis tools that enabled authors to develop
    and publish high-quality, unambiguous guidelines.
\end{itemize}

The \GLIF{} team comprised of research groups from various universities
where other languages, including Arden Syntax (section \ref{sec:arden-syntax}),
\GEODECM{} (section \ref{sec:geodecm}) and EON (section \ref{sec:eon}),
were developed. Thus, perspectives and learnings from existing efforts guided
the development of \GLIF{}, including borrowing certain useful constructs
directly from EON \cite{MachadoJAMIA98}

In \GLIF{}, guidelines are expressed using the \GLIF{} class containing
relevant attributes corresponding to treatment information. One such
attribute is an unordered list of all guideline steps. The guideline
specification itself is a directed graph defined over collection of
said steps. A step can be one of the following:
\begin{itemize}
    \item Action steps: Specify clinical-care actions during treatment.
    \item Conditional steps: Determine control flow based on logical statements.
    \item Branch steps: Enable flow to multiple guideline steps concurrently.
    \item Synchronization steps: Converge concurrent flow back to a single
      guideline step \cite{MachadoJAMIA98}.
\end{itemize}

\GLIF{} was used to implement several \CDSSs{}, such as a web-based
application for support in clinical consultations \cite{BoxwalaAMIA99} and
the Partners Computerized Algorithm Processor and Editor (\PCAPE{}) system
for creating and deploying \BPGs{} \cite{ZielstorffAMIA98}.

The experience and lessons from implementing aforementioned \CDSSs{} in
\GLIF{} also revealed certain limitations of the language, such as:
\begin{itemize}
  \item Integration of heterogeneous clinical systems was challenging.
  \item Guidelines were cumbersome to develop and readability was suboptimal.
\end{itemize}

To address these shortcomings, an updated version of the language called
\GLIF3{} was introduced  \cite{PelegAMIA00}. \GLIF3{} utilized different
abstraction levels to present relevant information based on the user's role.
Guidelines are modeled using three hierarchical \say{abstraction levels}:
\begin{enumerate}[label=\roman*.]
  \item The conceptual level, where the guideline is represented as
    a flowchart for consumption by \HCPs{}.
  \item The computable level, where information regarding
    execution exists to enable use in a \CDSS{}.
  \item The implementable level, where additional information
    regarding integration with institution-specific \EHRs{} resides.
\end{enumerate}
Like \GLIF{}, \GLIF3{} has also  been used to implement several \CDSSs{}, including
systems for depression screening and management \cite{ChoiJMI07} and
hyperkalemia patient screening \cite{WangBook04}. But, despite its use
in several systems, the semantics of the \GLIF{} framework remain only informally
defined.


\subsection{PROforma}\label{sec:proforma}

PROforma was initially developed at the Cancer Research UK Advanced Computation
Laboratory, and has been utilized to develop several \CDSSs{} \cite{ClerqAIM03}.
In PROforma, guidelines are modeled as tasks and data items. Tasks are
hierarchically organized into plan, and may be further divided into:
\begin{itemize}
  \item Actions: Procedures that must be performed in an external environment.
  \item Enquiries: Guideline points at which data must be obtained.
  \item Decisions: Points at which a choice determines further control flow
    \cite{SuttonAMIA03}.
\end{itemize}

PROforma guidelines can be visualized as directed graphs, with nodes representing
actions and edges constraints on control flow. Pictorially,
actions are represented as squares, enquiries as diamonds and
decisions as circles. This enables PROforma guidelines' visual representations
to resemble flowcharts and medical algorithms that \HCPs{} are already familiar
with.

A PROforma task has associated properties that determine how it's interpreted.
These include:
\begin{itemize}
  \item Captions and descriptions that improve guideline comprehensibility.
  \item Preconditions that specify conditions for a task's execution to begin.
  \item Scheduling constraints that enable synchronization between concurrent
    tasks.
\end{itemize}
Additionally, plans have termination and abort conditions that specify
successful end of the current plan to facilitate execution of successor
plans.

Using PROforma requires a suite of tools called Tallis \cite{TallisUrl}, written in Java,
that include a composer for creating and viewing guidelines, a tester for
debugging them, and an execution engine for running them \cite{SuttonAMIA03}.

PROforma has been used to implement several \CDSSs{} that have demonstrated
improvements in care quality in clinical settings. Among the first applications
of PROforma was the CAPSULE system that assisted general practitioners prescribe
medicines \cite{WaltonBMJ97}. Similarly, the LISA system was designed to \HCPs{} with
dosages for pediatric patients with acute lymphoblastic leukemia
\cite{BuryBJH05}. Other applications include support with image interpretation
and diagnosis \cite{TaylorMIA99}, and support for breast cancer diagnosis and
treatment \cite{FoxECAI06}.

PROforma is also among the first \DSLs{} to identify the for a formally defined
syntax and semantics, and expend significant effort in doing so.
In \cite{SuttonAMIA03}, the authors enunciate
that PROforma \say{must have} open source syntax and semantics definitions
to ensure that medical logic in PROforma is unambiguous, and to ensure
tools \say{correctly} read and process PROforma.

\subsection{Asbru}\label{sec:asbru}

Asbru, developed by collaborators from Stanford, Vienna
University of Technology, and Ben Gurion University, focuses on
time-oriented clinical guidelines \cite{ClerqAIM03}.
The language attempted to address limitations from existing approaches
such Dilemma (section \ref{sec:dilemma}) and EON (section \ref{sec:eon})
such as:
\begin{itemize}
  \item Lack of human-readability.
  \item Inability to integrate well with electronic patient records.
  \item Inability to model flexibility in guideline-prescribed steps.
\end{itemize}

To address these shortcomings, especially readability and modeling
flexibility, Asbru utilizes a skeletal plans-based approach.
Skeletal plans are a sequence of generalized steps that are instantiated
in a specific problem context to solve a given problem \cite{FriedlandJAR85}.
Thus, skeletal plans provide a broad framework for solving a class of problems,
instead of fine-grained details required to solve a specific problem.
This flexibility enables Asbru guidelines to have intended objectives that
can be achieved in different ways. For instance, to manage diabetes,
an \HCP{} can either administer additional insulin, or instruct
the patient to consume less carbohydrates during a meal. Both actions
have the intended outcome of moderating blood sugar levels.
Specifically, an Asbru skeletal plan based guideline has:
\begin{itemize}
  \item Actions to be performed by the \HCP{}.
  \item An intended plan made of action described as pattern of actions.
  \item The intended outcome described through patterns over patient states \cite{ShaharAIM98}.
\end{itemize}

When executed, the \HCP{} applies the guideline by following actions,
which are observed, recorded and abstrated over time into an absracted plan.
both patient state, and the inferred intention is also recorded
\cite{ShaharAIM98}. Asbru's execution model also provides the ability
to critique execution along these axes:
\begin{itemize}
  \item Guideline-prescribed actions vs \HCP{} followed actions.
  \item Guideline-intended plan vs \HCP{} followed plan.
  \item Guideline-intended state vs the \HCP{}'s intended state.
  \item Guideline-intended state vs actual patient state.
  \item \HCPs{}'s intended state vs actual patient state.
\end{itemize}
These axes provide 32 different scenarios of guideline execution
\cite{ShaharAIM98}.

Asbru provides comprehensive support for temporal aspects of guidelines through
the use of time reference annotations.
These represent uncertainty in starting, ending and duration
of time intervals. An annotation can be an absolute reference point,
a reference point with uncertainty (defined by a region), or a function of
some previous plan. Deviation from the reference point represent uncertainty
in starting. Through the use of these annotations, intervals have an
earliest start shift, latest start shift, earliest finishing shift, latest
finishing shift and minimum duration and maximum durations \cite{ShaharAIM98}.

A guideline in an Asbru library consists of a set
of plans and corresponding time annotations. During execution,
the interpreter attempts to decompose the plan into subplans from the library.
If a plan is non-decomposable, it is considered to be atomic, usually corresponding
to actions to be performed by an external agent such as obtaining information,
administering treatment, etc.

Library plans have four states: considered, possible, rejected and ready that
determine whether a plan is applicable. During execution, a ready plan is
instantiated. Apart from the name, arguments and time annotation, a plan also has:
\begin{itemize}
  \item Preferences: Factors that influence the selection of a plan to
    achieve a goal. Examples include start conditions for the plan,
    heuristics (such as exact-fit) to specify a selection criteria, etc.
  \item Intentions: High level goals of the plan.
  \item Effects: Functional relations between arguments and measurable
    clinical parameters.
  \item Body: Other plans (that also have the same structure) that may be
    executed in parallel, in sequence, or some combination thereof
    \cite{ShaharAIM98}.
\end{itemize}

The Asbru language has been used to implement several guidelines in
Endocrinology, Pulmonology, Gynaecology, such as diabetes mellitus, jaundice in
infants, and mechanical ventilation in premature babies
\cite{YoungAMIA05}. The language has a formal syntax in \BNF{}, and the earlier versions of the language
also have a formal structural operational semantics \cite{BalserPIDP02}.

During the development of Asbru, it was realized that
\CDSSs{} are limited by a lack of standardized \EHR{} formats, and in many
settings, a complete lack of \EHR{} support. To address these issues,
a new Asbru-based system that enables implementing \CDSSs{} in
settings with complete, partial, or no \EHR{}, called Spock, was developed.
Spock utilizes Hybrid-Asbru, a semi-formal variant of Asbru developed
for the Digital Electronic Guideline Library (\DEGEL{}) \cite{YoungAMIA05,YoungAIM05}.
Unlike Asbru, Hybrid-Asbru lacks a complete executable semantics and an
interpreter with any correctness guarantees.

\subsection{Guideline Acquisition, Representation, and Execution (\GLARE{})}\label{sec:glare}

The \GLARE{} system attempts to decouple the representation of knowledge
in \BPGs{} to their execution \cite{TerenzianiAIM01}. The \GLARE{} representation
language represents guidelines using actions, that may be either atomic or
composite. Atomic actions involve elementary steps that cannot be broken
down further, and can be one of the following:
\begin{itemize}
  \item Work actions: Steps or procedures that must be
    performed at a particular point during guideline execution.
  \item Query actions: Obtaining information from the outside world,
    for instance, from doctors, or databases.
  \item Decision actions: Selecting alternate actions in the guideline
    based on guideline-specified criteria.
  \item Conclusion actions: Finalizing the outcome of
    executing the guideline .
\end{itemize}
Atomic actions provide the elementary primitives for modeling
medical knowledge, but, to specify a guideline, additional mechanisms for
expressing relations over said actions is needed. Relations in
\GLARE{} can either be structure-related or control-related.
Structural relations enable building trees from actions, where
nodes represent composite actions, and leaves atomic ones.

Guideline execution requires specifying temporal ordering on
actions. This is supported through control-related relations, which
can be one of:
\begin{itemize}
  \item Sequence relations that enforce a strict ordering on actions, where,
    if any one of the actions in the sequence fails, then the entire sequence
    is considered to have failed.
  \item Concurrent relations that allow actions to run in any order, including
    in parallel.
  \item Decision relations that allow execution of alternative actions based
    on the conclusion of a decision action.
  \item Iteration relations that allow looping behavior, where
    an action may be repeated for a given duration, or, until a particular
    condition is met.
\end{itemize}

To enhance usability, actions in \GLARE{} must further follow a particular
structure. Work actions are further refined into clinical procedures and
pharmacological prescriptions. Additionally, work actions may additionally
have properties such as:
\begin{itemize}
  \item Preconditions that specify the context under which the action may apply.
  \item Cycles that specify temporal behavior \cite{TerenzianiAIM01}.
\end{itemize}


\section{General Approaches}\label{sec:general-approaches}

\subsection{Maude}

