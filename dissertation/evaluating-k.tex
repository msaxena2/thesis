\chapter{Evaluating \K{}}\label{chapter:evaluating-k}

In \autoref{chapter:semantics-first-cdss}, we introduced the
semantics-first approach to building systems, and discussed how $\K$,
a framework for defining semantics of programming languages and
type systems, enables it. Specifically, \autoref{sec:why-use-semantics-first}
discusses the rationale for using the semantics-first approach over
the traditional state-of-art for programming language design that
leads to tools based on ad-hoc language semantics and duplicated effort.
In \autoref{sec:semantics-first-pitfalls}, we also touched upon
some additional considerations for using $\K$ in the context of
developing a language for developing a computer-interpretable-guideline DSL.
We outlined the following concerns:
\begin{enumerate}[label=(Q\arabic*)]
 \item Can the $\K$ generated tools be performant enough to serve as
 the \text{sole} tools?
 \item Can the $\K$ semantics replace an informal language description/manual?
 \item Can the $\K$-based semantics first approach be used to develop
 a semantics from scratch, without even an informal description of the language,
  and, comprehensive tests to go long with the same? Can the semantics
  serve as a comprehensive document to implement other tools for the language?
\end{enumerate}
In the following sections attempt to answer these questions.
Specifically, in \autoref{sec:kevm}, we describe work on modeling the
formal semantics of the Ethereum Virtual Machine (\EVM{}) in $\K$, where
the generated interpreter's performance compares favorably against a
custom C++ implementation, providing evidence that (Q1) can be addressed.
Moreover, given the executable nature of the semantics, the $\K$ semantics
elaborate on ambiguities in the Yellow Paper---the official paper-based
specification of the \EVM{}. Not only can the $\K$-semantics replace an informal
language description, their detailed nature can also serve as a solid foundation
to base other tools on, addressing (Q2).

In \autoref{sec:ethereum-abi-dsl}, we discuss work on implementing a
\DSL{} for calling functions in Ethereum smart contracts using the
Application Binary Interface. Unlike previous $\K$ languages, this
\DSL{} neither has an informal semantics, nor existing test suites.
But the language and tests were developed concurrently with the $\K$
definition, providing evidence that (Q3) can addressed.

\section{\KEVM: Semantics of the Ethereum Virtual Machine}\label{sec:kevm}

The Ethereum ecosystem is a blockchain-based platform where accounts
have snippets of computer code called \emph{smart contracts}, that
enable programmatic governance of transactions on the Ethereum network.
All \emph{smart contracts} on the blockchain
must be specified in an assembly-like stack-based language called
the Ethereum Virtual Machine (\EVM{}). The semantics of the
Ethereum Virtual Machine are described semi-formally in a specification
called the Ethereum Yellow Paper \cite{WoodReport14}.

Smart contracts execute when a transaction calls the account, and, among other features,
these contracts can tally user votes, communicate with other contracts,
store or represent tokens and digital assets, and send or receive money in
cryptocurrencies, without requiring trust in any third party to faithfully
execute the contract \cite{SzaboReport94,PetersBook16}.
The growing popularity of smart contracts has led to increased scrutiny of their security.
Bugs in such contracts can be financially devastating to the involved parties.
An example of such a catastrophe is the DAO attack \cite{delCastilloReport16},
 where 150 million USD worth of Ether was stolen,
 prompting an unprecedented hard fork of the Ethereum blockchain
 \cite{DaianReport16}.

To address these issues in a principled manner, in \cite{HildenbrandtCSF18}
\footnote{Joint work with E. Hildenbrandt et al.},
we formalized the semantics of the \EVM{} in \K{}, with the intention to
utilize the semantics-generated tools for both execution and analysis of \KEVM{}
programs. Specifically, we intended to demonstrate that:

\begin{itemize}
  \item The $\K$ definition can be used not only to derive semantics-based
  tools, but also as an \emph{unambiguous alternative} to the
  informal Yellow Paper semantics specification to implement new tools. In other
  words, the semantics themselves are \emph{readable} to enable one to
  comprehend all subtleties and nuances of the language.
  \item The generated tools, including the interpreter, are \emph{performant}
  enough for use in place of custom \EVM{}-specific tools.
  \item  The $\K$ definition is \emph{complete}, as in, the interpreter
  generated from it passes all tests in the reference test suite.
  \item The definition is useful, i.e., the generated tools can be used for
  formal analysis in some meaningful way.
\end{itemize}

For a more complete reading, please see
\cite{HildenbrandtCSF18}. Here, we only present arguments that
help answer the questions we posed at the beginning of this chapter.
The $\K{}$ semantics of \EVM{} presented in \cite{HildenbrandtCSF18}
has 2644 lines of non-blank and non-literate code and passes
all reference tests published by the Ethereum Foundation for testing
EVM implementations. For reference,
the EVM specific C++ implementation has 4588 lines of code.
This measurement was taken on commit-hash ee0c6776c of
\url{https://github.com/ethereum/cpp-ethereum}
by counting non-blank lines of all \inlinek{*.h} and \inlinek{*.cpp} files
in subdirectory \inlinek{libevm}.
We argue that these numbers are not atypical for implementing an
interpreter for a small real-world programming language,
not to mention the extra tools that \K{} provides for analysis and security along the way.

In \autoref{table:kevm-perf}, we present a performance comparison between KEVM,
the Lem semantics~\cite{HiraiWSTC17},
and the C++ reference implementation distributed by the Ethereum foundation.
At the time of publication of \cite{HildenbrandtCSF18}, the Lem semantics:
the only other executable formal specification of the EVM that we are aware of
at the time of comparison.

\begin{table}[h]
    \centering
      \begin{tabular}{ l l l l }
          \textbf{Test Set} (no. tests) & \textbf{Lem EVM} & \textbf{KEVM} & \textbf{cpp-ethereum} \\
          Lem (40665)                   & 288:39           & 34:23         & 3:06                  \\
          VMStress (18)                 & -                & 72:31         & 2:25                  \\
          VMNormal (40665)              & -                & 27:10         & 2:17                  \\
          VMAll (40683)                 & -                & 99:41         & 4:42                  \\
          GSNormal(22705)               & -                & 35:00         & 1:30                  \\
          GSQuad (250)                  & -                & 855:24        & 0:21                  \\
          GSAll (22955)                 & -                & 889:00        & 1:51                  \\
      \end{tabular}
  \caption{Lem EVM vs KEVM vs cpp-ethereum} \label{table:kevm-perf}
\end{table}


All execution times are given as the full sequential CPU time (in MM:SS format)
  on an Intel i5-3330 processor (3GHz on 4 hardware threads) and 24 GB of RAM.
The row Lem indicates a run of all the tests that the Lem semantics
can run (a subset of the VMTests).
The row VMStress indicates a run of all 18 stress tests in the test-suite,
    to compare the performance of KEVM with the C++ implementation.
The row VMNormal is a run of all the non-stress
tests in the test-suite (\textit{not} the same set of tests as Lem).
VMAll is the addition of the second and third rows and is included for completeness.
The last three rows indicate a runs of the GeneralStateTests;
GSNormal are the non-stress tests, GSQuad are the stress tests,
         and GSAll is the addition of the two.
Under the GeneralStateTests, our tools performs well except
in the case of QuadraticStateTests (250 out of 22955).

As shown in the comparison,
   the automatically extracted interpreter for KEVM outperforms
   the currently available formal executable EVM semantics.
\KEVM{} compares favorably to the C++ implementation,
     performing under 30 times slower on the stress tests,
     roughly 20 times slower on all tests, and
     only 11 times slower on the Lem and VMNormal tests. Note
     advancements to \K's concrete execution capabilities since
     the publication of \cite{HildenbrandtCSF18} may
     make the \KEVM{} interpreter compare even more favorably against the
     C++ implementation.

Next, we look at a more holistic comparison of \KEVM{} against related approaches,
some of which are also semantics-based. We compare these approaches using the
following metrics:
\begin{itemize}
    \item \textbf{Spec.:} Suitable as a formal specification?
    \item \textbf{Exec.:} Executable on concrete tests?
    \item \textbf{Tests:} Passes the Ethereum test-suites?
    \item \textbf{Prover:} Serves as theorem prover for EVM?
    \item \textbf{Bugs:} Heuristic-based tools for finding bugs?
    \item \textbf{Gas:} Analyzes gas complexity of EVM programs?
\end{itemize}

Table~\ref{table:comparison} shows an overview of the results of our comparison.
We briefly describe each effort and compare it to the relevant KEVM artifact.
The projects fit two categories: semantic specifications and smart contract analysis tools.


\subsection{Semantic Specifications}

\paragraph{Ethereum Yellow Paper:}
The Yellow Paper is the official document describing the execution of the EVM
\cite{WoodReport14},
    as well as other data, algorithms,
    and parameters required to build consensus-compatible
    EVM clients and Ethereum implementations.
It cannot be tested against the conformance test-suite;
instead, it serves as a guide for implementations to follow.
Much of the machine definition is supplied as several mutually
recursive functions from machine-state to machine-state.
The Yellow Paper is occasionally unclear or incomplete about
the exact operational behavior of the EVM;
in these cases, it is often easier to simply consult one of the executable implementations.

\paragraph{Cpp-ethereum:}
Cpp-ethereum is the
C++ implementation that at the time served as a de-facto semantics of the EVM
\cite{CppEthereumUrl}.
The Yellow Paper and the C++ implementation were
developed by the same group early in the project,
          so the Yellow Paper conforms mostly to the C++ implementation.
In addition, the conformance test-suite is generated from the C++ implementation.
This means that if the Yellow Paper and the C++ implementation disagree,
     the C++ implementation is favored.

\paragraph{Lem semantics:}
A Lem (\cite{MulliganSIGPLAN14}) implementation of \EVM{} provides
an executable semantics of EVM for doing formal verification of smart contracts \cite{HiraiWSTC17}.
Lem compiles to various interactive theorem provers,
    including Coq, Isabelle/HOL, and HOL4.
The Lem semantics does not capture inter-contract
execution precisely as it models function calls as non-deterministic events with an external (speculated) relation dictating the ``allowed non-determinism''.
This semantics is executable and passes all tests in the VMTests
test-suite except for those dealing with more complicated inter-contract execution,
  providing high levels of confidence in its correctness.

\paragraph{GMS small-step specification:}
A small-step specification of the EVM inspired by the
EtherLite semantics of \cite{LuuReport16}.
The specification is non-executable but provides a precise guide for
implementers of the EVM \cite{GrishchenkoTR18}.

\begin{table}[th]
\centering
\begin{tabular}{c | c | c | c | c | c | c}
    Tool          & Spec.  & Exec.  & Tests  & Prover & Bugs   & Gas    \\ \hline
    Yellow Paper  & \greencheck & \redcross   & \redcross   & \redcross   & \redcross   & \redcross   \\
    cpp-ethereum  & \redcross   & \greencheck & \greencheck & \redcross   & \redcross   & \redcross   \\
    Lem spec      & \greencheck & \greencheck & \greencheck & \greencheck & \redcross   & \redcross   \\
    Oyente        & \redcross   & \greencheck & \redcross   & \redcross   & \greencheck & \greencheck \\
    hevm          & \redcross   & \greencheck & \redcross   & \redcross   & \redcross   & \redcross   \\
    Manticore     & \redcross   & \greencheck & \redcross   & \redcross   & \greencheck & \greencheck \\
    REMIX         & \redcross   & \greencheck & \redcross   & \redcross   & \greencheck & \greencheck \\
    \Fstar        & \redcross   & \greencheck & \redcross   & \greencheck & \greencheck & \redcross   \\
    \KEVM{}       & \greencheck & \greencheck & \greencheck & \greencheck & \redcross   & \greencheck \\
\end{tabular}
\caption{Feature comparison of EVM semantics and other software quality tool efforts.} \label{table:comparison}
\end{table}


\subsection{Smart Contract Analysis Tools}

\paragraph{Oyente:}
Oyente is an EVM symbolic execution engine written in Python supporting most of
the EVM, with heuristics-based drivers of the engine for bug-finding \cite{OyenteUrl}.

\paragraph{Manticore:}
Manticore is symbolic execution engine for virtual machines, including models for x86, x86\_64, ARMv7, and EVM
\cite{ManticoreUrl}.
This tool exports a Python API for specifying programs, driving symbolic execution, and checking assertions.

\paragraph{REMIX}
REMIX is a JavaScript implementation of the EVM with a browser-based IDE for building and debugging smart contracts
\cite{RemixUrl}.
Some static analysis is built into the tool, allowing it to catch pre-specified classes of bugs in smart contracts.

\paragraph{Hevm:}
Hevm \cite{HevmUrl} is a Haskell implementation of EVM including on an interactive debugger mode, which allows stepping through contract execution one opcode at a time.

\paragraph{\Fstar{} formalization of EVM:}
An implementation of the EVM in the \Fstar{} language \cite{FstarUrl}
 which passes roughly half of the VMTests at the time of writing.
The same paper discusses an on-paper small-step specification of the EVM as well
\cite{GrishchenkoETAPS18}.

\paragraph{Discussion:}
At the start of this chapter, we asked whether tools generated from
the semantics can be performant enough to replace their language-specific
counterparts. We argue that the semantics-first treatment of
\KEVM{} addresses said question. Specifically, the
semantics-derived interpreter compares favorably against a language-specific
C++ implementation, and other tools are able to perform as well as
other language-specific tools on metrics from \autoref{table:comparison}.

We based our $\K$ semantics on the Yellow Paper and
found several inconsistencies that were confirmed by its developers \cite{HildenbrandtCSF18}.
This stems from the fact that unlike the $\K$ semantics,
the Yellow Paper is an informal document in natural language.
Often, the Yellow Paper was found to be unclear, underspecified
or inconsistent with the behavior of actual implementations.
As the $\K$ semantics are executable, all details must be
considered for our interpreter to pass all tests in the reference
test-suite. We utilized the developer documentation for our semantics
to generate an alternative to the Yellow Paper itself, called the Jello Paper
\cite{EvmJellopaperUrl}. The Jello Paper has been well-received by the
community, with discussions to designate it as the \emph{canonical document}
instead of the Yellow Paper. For more details, we refer the reader to
section VII of \cite{HildenbrandtCSF18}. Thus, it demonstrates the $\K$ semantics
can serve as a standalone language manual instead of an informal paper-based
one.

\section{Ethereum ABI \DSL{}}\label{sec:ethereum-abi-dsl}

EVM lacks high-level constructs including functions and explicit data types.
Any invocation of an EVM program always begins with the VM’s program counter set to 0.
Higher level languages such as Solidity and Viper, however, have functions and types, allowing users to directly call functions in contracts without dealing with low-level EVM code.
This lack of high level constructs makes writing specifications for programs
at the EVM level tedious and error prone.

To address this, in \cite{HildenbrandtCSF18}, we introduced a DSL modeled on the
Ethereum Application Binary Support Interface ABI.
The Ethereum ABI ~\cite{EthereumAbiUrl} is a mechanism for simulating a function call at the EVM level.
Contracts are written in higher level languages and call functions in other contracts.
The ABI facilitates this by specifying the encoding and decoding rules for a transaction's
calldata: an input field available to each transaction. The ABI
is an additional protocol that higher level
languages can choose to comply with to ensure compatibility with other smart
contracts written in other languages. Thus, compilers for popular high-level
smart contract languages often compile to ABI-compliant EVM code. As it's not
an official part of the Ethereum protocol, it's not a part of the Yellow Paper.

Since most smart contracts needing formal verification are written in a high-level
language like Solidity \cite{SolidityUrl} that produce ABI-compliant code, we
created a Domain Specific Language (\DSL{}) for interacting with ABI-compliant compiled
code. By using our DSL, in \cite{ParkFSE18} \footnote{Joint work with Park et al.},
we demonstrated that verification of complex smart contracts at the level of the
EVM bytecode, was feasible. Operating at the EVM-level had many benefits,
namely:
\begin{itemize}
  \item Our methodology worked for any high-level language, as long as it
    compiled down to ABI-compliant code.
  \item As our \DSL{} is built atop the EVM semantics, updates to the semantics
    are automatically accommodated.
\end{itemize}


\subsection{Verifying Contracts Using Our \DSL{}}\label{sec:verifying-contracts-dsl}

\paragraph{Specifying the ERC20 Standard in \K{}:}
The ERC20 standard \cite{ERC20Url} is a standard for implement tokens within
smart contracts, essentially enabling trade of third-party tokens using the
Ethereum infrastructure.
As an informal document, it specifies the correctness properties that
token contracts must satisfy.
Unfortunately, however, it leaves several corner cases unspecified,
making it less than ideal to use in the formal verification of token contracts. % implementations.

As part of verifying ERC20 token implementations, we implemented the semantics
of ERC20 in \K{}. Our definition
clarifies what data
(e.g., balances and allowances) are handled by the various ERC20 functions
and the precise meaning of those functions on such data.
It also clarifies the meaning of all the corner cases that the ERC20 standard omits to discuss,
such as transfers to itself or transfers that result in arithmetic overflows,
following the most natural implementations that aim at minimizing gas consumption.

To evaluate the effectiveness of both the EVM semantics, and the
associated ABI \DSL{}, we verified three popular smart contracts:
the Vyper ERC20 token\footnote{\url{https://github.com/ethereum/vyper/blob/master/examples/tokens/ERC20_solidity_compatible/ERC20.vy}},
the HackerGold (HKG) ERC20 token\footnote{\url{https://github.com/ether-camp/virtual-accelerator/blob/master/contracts/StandardToken.sol}},
and OpenZeppelin's ERC20 token\footnote{\url{https://github.com/OpenZeppelin/openzeppelin-solidity/blob/master/contracts/token/ERC20/StandardToken.sol}}.
Of these, the Vyper ERC20 token was written in Vyper, and the others are written in Solidity.
We compiled the source code down to the EVM bytecode using each language compiler,
and executed our verifier to verify that the compiled EVM bytecode satisfies the
ERC20 standard.
During this verification process, and found divergent behaviors across these contracts that do not conform to the standard.

\subsection{Discussion}
Note that for a more thorough introduction to our \DSL{} and corresponding
verification efforts, we refer the reader to \cite{HildenbrandtCSF18,ParkFSE18}.
Here, we only mention it in the context of answering
the questions we posed at the start of this chapter. All previous
attempts at defining semantics in \K{}, such as C, Java and EVM,
had relied on existing informal or semi-formal semantics documentation.
Moreover, existing test suites could be relied upon to \emph{quantifiably test}
the completeness of the semantics. The \DSL{} described in this section,
however, represents a break from traditional \K{} development as:
\begin{enumerate*}[label=(\alph*)]
  \item the \DSL{} had to be implemented from scratch without an existing
     language manual, and,
   \item tests had to be written as the language was developed.
\end{enumerate*}
Moreover, as discussed briefly in \ref{sec:verifying-contracts-dsl},
our \DSL{} can be used to verify \emph{real-world smart contracts} at the
level of the EVM-bytecode.

At the beginning of this chapter (see \autoref{chapter:evaluating-k},
we attempted to evaluate whether $\K$ would be a good-choice for
a \emph{solely} semantics-first approach to clinical decision support using the following
questions:
\begin{enumerate}[label=(Q\arabic*)]
 \item Can the $\K$ generated tools be performant enough to serve as
 the \text{sole} tools?
 \item Can the $\K$ semantics replace an informal language description/manual?
 \item Can the $\K$-based semantics first approach be used to develop
 a semantics from scratch, without even an informal description of the language,
  and, comprehensive tests to go along with the same? Can the semantics
  serve as a comprehensive document to implement other tools for the language?
\end{enumerate}
By \emph{solely}, we meant that we intend the semantics-based tools to be
the \emph{de-facto} tools for our language. In other words, the
semantics-derived tools should supplant any language specific ones.

In \autoref{sec:kevm}, we presented a formalization of the
Ethereum Virtual Machine (\EVM{}) in \K{}, called \KEVM{}.
\KEVM{} was developed using an informal semantics of
the EVM described in a document known as the Yellow Paper.
We demonstrated that the $\K{}$ semantics-derived
interpreter not only passed all tests in EVM's conformance test suite,
but also compared favorably in terms of performance to the de-facto
C++ language-specific implementation, while offering stronger correctness
guarantees, addressing (Q1). Moreover, we found inconsistencies
in the original Yellow Paper, to which our semantics has since become a popular
alternative, addressing (Q2).

Next, we implemented a \DSL{} based on the Ethereum ABI to make verification of
smart contracts easier. Unlike earlier $\K$-based efforts, the language was
designed and implemented entirely using $\K$, without the help of any
existing documentation or tests. Although it is much smaller in scope and
less complex than other $\K$ languages (such as C or Java), our DSL could be
utilized to verify real-world smart contracts against desired non-trivial specs
(such as ERC20). Thus, our work demonstrated that \K{} can be used to implement
languages from scratch, with \emph{performant} and \emph{useful}
semantics-derived tools that can function as the \emph{sole} tools for the
language, addressing (Q3).

