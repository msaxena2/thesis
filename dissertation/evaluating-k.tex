\chapter{Evaluating \K{}}\label{chapter:evaluating-k}

In \autoref{chapter:semantics-first-cdss}, we introduced the
semantics-first approach to building systems, and how $\K$,
a framework for defining semantics of programming languages and
type systems, enables it. Specifically, \autoref{sec:why-use-semantics-first}
discusses the rationale for using the semantics-first approach over
the traditional state-of-art for programming language design that
leads to tools based on ad-hoc language semantics and duplicated effort.
In \autoref{sec:semantics-first-pitfalls}, we also touched upon
some additional considerations for using $\K$ in the context of
developing a language for developing a computer interpretretable giudeline DSL.
We outlined the following concerns:
\begin{enumerate}[label=(Q\arabic*)]
 \item Can the $\K$ generated tools be performant enough to serve as
 the \text{sole} tools?
 \item Can the $\K$-based semantics first approach be used to develop
 a semantics from scratch, wihout even an informal description of the language,
  and, comprehesive tests to go long with the same? Can the semantics
  serve as a comprhenesive docuemnt to implement other tools for the language?
\end{enumerate}

In the following sections attempt to answer aforementioned questions.

\section{\KEVM: Semantics of the Ethereum Virtual Machine}\label{sec:kevm}

The Ethereum ecosystem is a blockchain-based platform where accounts
have snippets of computer code called \emph{smart contracts}, that
enable programmatic governance of transactions on the Ethereum network.
All \emph{smart contracts} on the blockchain
must be specified in is an stack-based assembly-like language called
the Ethereum Virtual Machine (\EVM{}). The semantics of the
Ethereum Virtual Machine are described semi-formally in a specification
called the Ethereum YellowPaper \cite{WoodReport14}.

Smart contracts execute when a transaction calls the account, and, among other features,
these contracts can tally user votes, communicate with other contracts,
store or represent tokens and digital assets, and send or receive money in
cryptocurrencies, without requiring trust in any third party to faithfully
execute the contract \cite{SzaboReport94,PetersBook16}.
The growing popularity of smart contracts has led to increased scrutiny of their security.
Bugs in such contracts can be financially devastating to the involved parties.
An example of such a catastrophe is the DAO attack \cite{delCastilloReport16},
 where 150 million USD worth of Ether was stolen,
 prompting an unprecedented hard fork of the Ethereum blockchain
 \cite{DaianReport16}.

To address these issues in a principled manner, in \cite{HildenbrandtCSF18}
\footnote{Joint work with E. Hildenbrandt et al.},
we formalized the semantics of the \EVM{} in \K{}, with the intention to
utilize the semantics-generated tools for both execution and analysis of \KEVM{}
program. Specifically, we intended to demonstrate that:

\begin{itemize}
  \item The $\K$ definition can be used not only to derive semantics-based
  tools, but also as an \emph{unambiguous alternative} to the
  informal YellowPaper semantics specification to implement new tools. In other
  words, the semantics themselves are \emph{readable} to enable one to
  comprehend all subtleties and nuances of the language.
  \item The generated tools, including the interpreter, are \emph{performant}
  enough for use in place of custom \EVM{}-specific tools.
  \item  The $\K$ definition is \emph{complete}, as in, the interpreter
  generated from it passes all tests in the reference test suite.
  \item The definition is useful, i.e., the generated tools can be used for
  formal analysis in some meaningful way.
\end{itemize}

For a more complete reading, please see
\cite{HildenbrandtCSF18}. Here, we only present arguments that
help answer the questions we posed at the beginning of this chapter.
The $\K{}$ semantics of \EVM{} presented in \cite{HildenbrandtCSF18}
has 2644 lines of non-blank and non-literate code, and passes
all reference tests published by the Ethereum Foundation for testing
EVM implementations. For reference,
the EVM specific C++ implementation has 4588 lines of code.
This measurement was taken on commit-hash ee0c6776c of
\url{https://github.com/ethereum/cpp-ethereum}
by counting non-blank lines of all \inlinek{*.h} and \inlinek{*.cpp} files
in subdirectory \inlinek{libevm}.
We argue that these numbers are not atypical for implementing an
interpreter for a small real-world programming language,
not to mention the extra tools that \K{} provides for analysis and security along the way.

In \autoref{table:kevm-perf}, we present a performance comparison between KEVM,
the Lem semantics~\cite{HiraiWSTC17},
and the C++ reference implementation distributed by the Ethereum foundation.
At the time of publication of \cite{HildenbrandtCSF18}, the Lem semantics:
the only other executable formal specification of the EVM that we are aware of
at the time of comparison.

\begin{table}[h]
    \centering
      \begin{tabular}{ l l l l }
          \textbf{Test Set} (no. tests) & \textbf{Lem EVM} & \textbf{KEVM} & \textbf{cpp-ethereum} \\
          Lem (40665)                   & 288:39           & 34:23         & 3:06                  \\
          VMStress (18)                 & -                & 72:31         & 2:25                  \\
          VMNormal (40665)              & -                & 27:10         & 2:17                  \\
          VMAll (40683)                 & -                & 99:41         & 4:42                  \\
          GSNormal(22705)               & -                & 35:00         & 1:30                  \\
          GSQuad (250)                  & -                & 855:24        & 0:21                  \\
          GSAll (22955)                 & -                & 889:00        & 1:51                  \\
      \end{tabular}
  \caption{Lem EVM vs KEVM vs cpp-ethereum} \label{table:kevm-perf}
\end{table}


All execution times are given as the full sequential CPU time (in MM:SS format)
  on an Intel i5-3330 processor (3GHz on 4 hardware threads) and 24 GB of RAM.
The row Lem indicates a run of all the tests that the Lem semantics
can run (a subset of the VMTests).
The row VMStress indicates a run of all 18 stress tests in the test-suite,
    to compare the performance of KEVM with the C++ implementation.
The row VMNormal is a run of all the non-stress
tests in the test-suite (\textit{not} the same set of tests as Lem).
VMAll is the addition of the second and third rows and is included for completeness.
The last three rows indicate a runs of the GeneralStateTests;
GSNormal are the non-stress tests, GSQuad are the stress tests,
         and GSAll is the addition of the two.
Under the GeneralStateTests, our tools performs well except
in the case of QuadraticStateTests (250 out of 22955).

As shown in the comparison,
   the automatically extracted interpreter for KEVM outperforms
   the currently available formal executable EVM semantics.
\KEVM{} compares favorably to the C++ implementation,
     performing under 30 times slower on the stress tests,
     roughly 20 times slower on all tests, and
     only 11 times slower on the Lem and VMNormal tests. Note
     advancements to \K's concrete execution capabilities since
     the publication of \cite{HildenbrandtCSF18} may
     make the \KEVM{} interpreter compare even more favorably against the
     C++ implementation.

Next, we look at a more holistic comparison of \KEVM{} against related approaches,
some of which are also semantics-based. We compare these approaches using the
following metrics:
\begin{itemize}
    \item \textbf{Spec.:} Suitable as a formal specification?
    \item \textbf{Exec.:} Executable on concrete tests?
    \item \textbf{Tests:} Passes the Ethereum test-suites?
    \item \textbf{Prover:} Serves as theorem prover for EVM?
    \item \textbf{Bugs:} Heuristic-based tools for finding bugs?
    \item \textbf{Gas:} Analyzes gas complexity of EVM programs?
\end{itemize}

Table~\ref{table:comparison} shows an overview of the results of our comparison.
We briefly describe each effort and compare it to the relevant KEVM artifact.
The projects fit two categories: semantic specifications and smart contract analysis tools.


\subsection{Semantic Specifications}

\paragraph{Ethereum YellowPaper:}
The YellowPaper is the official document describing the execution of the EVM
\cite{WoodReport14},
    as well as other data, algorithms,
    and parameters required to build consensus-compatible
    EVM clients and Ethereum implementations.
It cannot be tested against the conformance test-suite;
instead it serves as a guide for implementations to follow.
Much of the machine definition is supplied as several mutually
recursive functions from machine-state to machine-state.
The Yellow Paper is occasionally unclear or incomplete about
the exact operational behavior of the EVM;
in these cases it is often easier to simply consult one of the executable implementations.

\paragraph{Cpp-ethereum:}
Cpp-ethereum is the
C++ implementation that at the time served as a de-facto semantics of the EVM
\cite{CppEthereumUrl}.
The Yellow Paper and the C++ implementation were
developed by the same group early in the project,
          so the Yellow Paper conforms mostly to the C++ implementation.
In addition, the conformance test-suite is generated from the C++ implementation.
This means that if the Yellow Paper and the C++ implementation disagree,
     the C++ implementation is favored.

\paragraph{Lem semantics:}
A Lem (\cite{MulliganSIGPLAN14}) implementation of \EVM{} provides
an executable semantics of EVM for doing formal verification of smart contracts \cite{HiraiWSTC17}.
Lem compiles to various interactive theorem provers,
    including Coq, Isabelle/HOL, and HOL4.
The Lem semantics does not capture inter-contract
execution precisely as it models function calls as non-deterministic events with an external (speculated) relation dictating the ``allowed non-determinism''.
This semantics is executable and passes all of the VMTests
test-suite except for those dealing with more complicated inter-contract execution,
  providing high levels of confidence in its correctness.

\paragraph{GMS small-step specification:}
A small-step specification of the EVM inspired by the
EtherLite semantics of \cite{LuuReport16}.
The specification is non-executable, but provides a precise guide for
implementers of the EVM \cite{GrishchenkoTR18}.

\begin{table}[th]
\centering
\begin{tabular}{c | c | c | c | c | c | c}
    Tool          & Spec.  & Exec.  & Tests  & Prover & Bugs   & Gas    \\ \hline
    Yellow Paper  & \greencheck & \redcross   & \redcross   & \redcross   & \redcross   & \redcross   \\
    cpp-ethereum  & \redcross   & \greencheck & \greencheck & \redcross   & \redcross   & \redcross   \\
    Lem spec      & \greencheck & \greencheck & \greencheck & \greencheck & \redcross   & \redcross   \\
    Oyente        & \redcross   & \greencheck & \redcross   & \redcross   & \greencheck & \greencheck \\
    hevm          & \redcross   & \greencheck & \redcross   & \redcross   & \redcross   & \redcross   \\
    Manticore     & \redcross   & \greencheck & \redcross   & \redcross   & \greencheck & \greencheck \\
    REMIX         & \redcross   & \greencheck & \redcross   & \redcross   & \greencheck & \greencheck \\
    Dr. Y's       & \redcross   & \greencheck & \redcross   & \greencheck & \redcross   & \redcross   \\
    \Fstar        & \redcross   & \greencheck & \redcross   & \greencheck & \greencheck & \redcross   \\
    \KEVM{}       & \greencheck & \greencheck & \greencheck & \greencheck & \redcross   & \greencheck \\
\end{tabular}
\caption{Feature comparison of EVM semantics and other software quality tool efforts.} \label{table:comparison}
\end{table}


\subsection{Smart Contract Analysis Tools}

\paragraph{Oyente:}
Oyente is an EVM symbolic execution engine written in Python supporting most of
the EVM, with heuristics-based drivers of the engine for bugfinding \cite{OyenteUrl}.

\paragraph{Hevm:}
Hevm \cite{HevmUrl} is a Haskell implementation of EVM including on an interactive debugger mode, which allows stepping through contract execution one opcode at a time.

\paragraph{\Fstar{} formalization of EVM:}
An implementation of the EVM in the \Fstar{} language \cite{FstarUrl}
 which passes roughly half of the VMTests at the time of writing.
The same paper discusses an on-paper small-step specification of the EVM as well
\cite{GrishchenkoETAPS18}.
