\chapter{Evaluating \K{}}\label{chapter:evaluating-k}

In \autoref{chapter:semantics-first-cdss}, we introduced the
semantics-first approach to building systems, and how $\K$,
a framework for defining semantics of programming languages and
type systems, enables it. Specifically, \autoref{sec:why-use-semantics-first}
discusses the rationale for using the semantics-first approach over
the traditional state-of-art for programming language design that
leads to tools based on ad-hoc language semantics and duplicated effort.
In \autoref{sec:semantics-first-pitfalls}, we also touched upon
some additional considerations for using $\K$ in the context of
developing a language for developing a computer interpretretable giudeline DSL.
We outlined the following concerns:
\begin{itemize}
 \item Can the $\K$ generated tools be performant enough to serve as
 the \text{sole} tools?
 \item Can the $\K$-based semantics first approach be used to develop
 a semantics from scratch, wihout even an informal description of the language,
  and, comprehesive tests to go long with the same? Can the semantics
  serve as a comprhenesive docuemnt to implement other tools for the language?
\end{itemize}




