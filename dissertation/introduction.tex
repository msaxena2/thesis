\chapter{Introduction}

Preventable medical errors (\PMEs{}) characterized by misdiagnosis or mistreatment present
a signficant challenge in healthcare, both in the United States, and abroad \cite{RodziewiczStatsPearls18}.
According to a seminal report on the subject by the Institute of Medicine, in 1997,
between 44,000 and 98,000 deaths were estimated to have been caused by \PMEs{} in
the United States alone \cite{DonaldsonBook00}.
A more recent study analyzed data from the eight-year
period between 2000 and 2008, and estimated that in 2013, the number of deaths
caused by \PMEs{} was more than 250,000, making \PMEs{} the third-leading
cause of death in the United States \cite{MakaryBMJ16}.

The prevalence of \PMEs{} in healthcare now finds increasing recognition.
In September 2023, the President's Council of Advisors on Science and
Technology (PCAST) published a report stating
patient safety to be an \say{urgent national public health issue}, and calling
for a \say{transformational effort} to address the problem \cite{PCAST23}.
Approximately a quarter of medicare
patients, the report mentioned, experienced adverse events during their hospitalization, with
many catastrophic outcomes. Worryingly, 40\% of these adverse events were
due to \PMEs{}. Thus, while recognizing the commitment that
healthcare practitioners and their organizations towards ensuring quality
care, the report deemed rates of medical errors and injuries as \say{alarmingly high}.

The adverse effects of \PMEs{} extend beyond patient outcomes.
\PMEs{} have a negative effect on healthcare providers (\HCPs). According to the authors of
\cite{RodziewiczStatsPearls18}, \PMEs{} caused physcological effects such
as anger and guilt in healthcare providers (\HCPs{}), adversely impacting their mental
health. Moreover, there are costs associated with such errors.
In 2008, the financial burden of \PMEs{} to the United States was
estimated to be \$19.5 billion, of which \$17 billion could be
directly attributed to additional healthcare expenditure such as
ancillary and prescription drug services, and inpatient and outpatient care.
Additionally, \$1.4 billion was attributed to increased mortality and loss
of productivity from missed work due to short-term disability claims \cite{AndelJHCF12}.

In the United States, several initiatives to improve patient safety by
reducing preventable errors have been undertaken at the federal government
level. The aforementioned September 2023 PCAST report itself included several suggestions to
address patient safety, such as:
\begin{enumerate}[label=(\roman*)]
  \item ensuring patients receive evidence-based practice for preventing
    harm and assessing risk, where as many measures as possible are
    generated in real-time from electronic health data.
  \item incentivizing hospitals to utilize evidence-based practices
    and associated novel computing tools.
  \item promoting research and development in areas such as
    computer science and health technology.
    including tools in prototype form, or limited deployment that
    hold promise to guide decision making to improve quality and safety.
  \item empowering hospitals with safety-critical methods, transparency,
    and a instilling a culture of safety.
  \item harnessing advances in technology, such as in Artificial Intelligence,
    while employing best practices and engineering principles to mitigate safety
    and reliability of such systems.
\end{enumerate}
The report highlights that technology is expected to play a major role in
addressing the problem. The emphasis on technology to reduce human errors
isn't unprecedented. The report recommends the establishment of a
multidisciplinary National Patient Safety Team, along the lines of
the Commerical Aviation Safety Team (CAST) \cite{CASTUrl}, comprising of
experts from clinical, safety and computer sciences, to address challenges
in patient safety. In the decade following its inception in 1997,
CAST was responsible for reducing the fatality risk for commercial aviation
in the United States by 83\%, and aims to reduce the risk further by 50\% by
2025. Several solutions were instrumental in achieving such results, including:
\begin{itemize}
  \item fly-by-wire control that optimizes pilot inputs for safety, reliability,
    and passenger comfort \cite{FBWSkybraryUrl}.
  \item flight management and autopilot systems that automate certain navigation
    and flying tasks to reduce pilot workloads and make bandwidth available for monitoring and
    managing emergencies \cite{CockpitAutomationSkybraryURL}.
  \item diagnostic and assurance systems that support enhanced pilots' and
    maintenance staff's understanding of aircraft systems, faults, and
    mitigation strategies \cite{CockpitAutomationSkybraryURL}.
  \item traffic management systems that enhance situational awareness of
    pilots, air traffic controllers, and flight planners \cite{ATMSkybraryUrl,TCASUrl}.
\end{itemize}
CAST was instrumental in enabling the development of, and enacting policy to
encourage adoption of aforementioned solutions \cite{CASTSafetySkybrary}.
