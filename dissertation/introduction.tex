\chapter{Introduction}


Preventable medical errors (\PMEs{}) characterized by misdiagnosis or mistreatment present
a signficant challenge in healthcare, both in the United States, and abroad \cite{RodziewiczStatsPearls18}.
According to a seminal report on the subject by the Institute of Medicine, in 1997,
between 44,000 and 98,000 deaths were estimated to have been caused by \PMEs{} in
the United States alone \cite{DonaldsonBook00}.
A more recent study analyzed data from the eight-year
period between 2000 and 2008, and estimated that in 2013, the number of deaths
caused by \PMEs{} was more than 250,000, making \PMEs{} the third-leading
cause of death in the United States \cite{MakaryBMJ16}.

The prevalence of \PMEs{} in healthcare now finds increasing recognition.
In September 2023, the President's Council of Advisors on Science and
Technology (PCAST) published a report stating
patient safety to be an \say{urgent national public health issue}, and calling
for a \say{transformational effort} to address the problem \cite{PCAST23}.
Approximately a quarter of medicare
patients, the report mentioned, experienced adverse events during their hospitalization, with
many catastrophic outcomes. Worryingly, 40\% of these adverse events were
due to \PMEs{}. Thus, while recognizing the commitment that
healthcare practitioners and their organizations towards ensuring quality
care, the report deemed rates of medical errors and injuries as \say{alarmingly high}.

The adverse effects of \PMEs{} extend beyond patient outcomes.
In 2008, the financial burden of \PMEs{} to the United States was
estimated to be \$19.5 billion, of which \$17 billion could be
directly attributed to additional healthcare expenditure such as
ancillary and prescription drug services, and inpatient and outpatient care.
Additionally, \$1.4 billion was attributed to increased mortality and loss
of productivity from missed work due to short-term disability claims \cite{AndelJHCF12}.

\PMEs{} have a negative effect on healthcare providers (\HCPs) as well. According to the authors of
\cite{RodziewiczStatsPearls18}, \PMEs{} caused physcological effects such
as anger and guilt in healthcare providers (\HCPs{}), adversely impacting their mental
health.
