\chapter{Future Work}

This chapter outlines major tasks that remain towards enhancing the capabilities
of our system.

\section{Soundness of Model Checking Abstraction}\label{sec:abstraction-soundness}

In \autoref{sec:formal-analysis}, we introduced an abstraction that enabled us
to model check the system. Instead of interpreting $\MediK{}$ programs
concretely, we introduced rules that operate over abstract values
that hold semantic meaning. However, we did not establish that the analysis is
indeed sound. In this section, we briefly outline the steps to prove the
soundness of our abstraction-based analysis, and outline limitations that
prevented us from establishing it.

Recall that in \autoref{sec:theoretical-foundations}, we presented the
matching logic: the theoretical foundations of the $\K$ framework. We explained
that a $\K$ definition of a language $L$, compiles to a matching logic theory
$\Gamma^L$. A symbol $\snext \in \Sigma$ is then defined as a part of the
signature to capture dynamic behavior of rewrite rules.
Intuitively, given a configuration
pattern $\gamma$, the pattern $\snext \gamma$ matches all possible
configurations that $\gamma$ can rewrite to in one-step. Thus
the one-step rewrite relation can be defined as:
$$
  \varphi \To^1 \psi \equiv \varphi \to \snext \psi
$$

Next, the reflexive, transitive closure of the one-step rewrite relation
can be defined using fixed-points as:
$$
  \eventually \varphi \equiv \mu X \ldot \varphi \vee \snext X
$$
and the rewrite $\varphi \To \psi$ as:
$$
  \varphi \To \psi \equiv \varphi \to \eventually \psi
$$
In other words, if a configuration matches
$\varphi$ then it matches $\varphi$ in zero or more steps.
Note that the existence of a fixed point of $\eventually\varphi$
is guaranteed by the Knaster-Tarski theorem as long as the interpretation
$\mathcal{F}^{\rho}_{X,\varphi} = \interpret{\varphi}{\rho\left[A / X\right]}$
is monotone \cite{ChenLICS19}.

Typically, establishing soundness of abstraction-based static analysis
requires coming up partial order over abstract states,
$\left(\Abs, \sqsubseteq\right)$, and, semantic rules defined over $\Abs$.
For $a_1, a_2 \in \Abs$, $a_1 \sqsubseteq a_2$ typically means $a_1$ is
\emph{more precise} than $a_2$. Recall that in \autoref{sec:formal-analysis},
that we defined a value \inlinek{#nondet}, denoting the very simple abstract
domain $\top$ or \emph{undetermined},
and rules that operate over it. Let $\ToAbs$ be the transition
system defined by rules over the abstract state. We first need to
establish that for any $a_1, a_2 \in \Abs$, if $a_1 \ToAbs a_2$ then $a_1
\sqsubseteq a_2$. In other words, we need to show that the transition relation
$\ToAbs$ is monotonic w.r.t. the ordering $\sqsubseteq$.

Next, we need to relate the concrete transition system $\To$ to $\ToAbs$.
Typically, this is done by defining a homomorphism $\beta: \Concrete \to \Abstract$
from the set of concrete values $\Concrete$ to the set of abstract values
$\Abstract$. We can then utilize $\beta$ to define a complete lattice
configuration terms. Recall from \autoref{sec:configuration} that
$\MediK{}$'s configuration comprises of an \inlinek{<store>}-cell that
stores map $\rho$ of pointer-value bindings.
Given configuration terms $\overline{\tau}_1, \overline{\tau}_2 \in
\Abs$ with \inlinek{<store>} cells containing maps $\rho_{\overline{\tau}_1}$ and
$\rho_{\overline{\tau}_2}$ respectively.
We say $\overline{\tau}_1 \sqsubseteq \overline{\tau}_2$ iff
they only differ in their
\inlinek{<store>} cell (all other cells are similar), and, for every
pointer in $p \in$ \inlinekmath{keys($\rho_{\overline{\tau}_1} \cup \rho_{\overline{\tau}_2}$)}
we have $\rho_{\overline{\tau}_1}[p] \sqsubseteq \rho_{\overline{\tau}_2}[p]$. Note we
say, given a map $\rho$ of pointer-value bindings,
$\rho[p] = \bot$ for any $p \not\in$ \inlinekmath{keys($\rho$)}.

To complete the soundness argument, we need to need to make the complete the connection
between concrete and abstract executions. Given homomorphism $\beta: \Concrete \to \Abstract$
from concrete values to abstract values, and a concrete configuration term
$\tau$. We derive an abstract configuration term $\overline{\beta}\left(\tau\right)$ from $\tau$ s.t.
every \inlinekmath{$($pointer $\ \mapsto\ $ value$)$} pair in $\tau$'s
\inlinek{<store>}-cell is mapped to \inlinekmath{$($pointer $\ \mapsto\ \beta($value$))$}
in $\overline{\beta}\left(\tau\right)$'s \inlinek{<store>}-cell.
We then need to show that for
concrete terms $\tau, \tau'$ and abstract term $\overline{\tau}$, if
\begin{enumerate*}[label=(\alph*)]
  \item $\tau \To \tau'$, and,
  \item $\overline{\beta}\left(\tau\right) \sqsubseteq \overline{\tau}$
\end{enumerate*}
then there exists an abstract state $\overline{\tau}'$ such that
$\overline{\tau} \ToAbs \overline{\tau}'$ and $\overline{\beta}\left(\tau'\right) \sqsubseteq \overline{\tau}'$.
This, along with the proof of monotonicity of $\ToAbs$ w.r.t. the ordering
$\sqsubseteq$ enables us to show that any trace in the concrete semantics is safely approximated
by a corresponding trace in the abstract semantics, as shown in
\cite{SchmidtLISP98}. However, while we briefly sketch out a proof outline
in this work, a complete proof for a language as complex as \MediK{} is a
significant undertaking, which we leave to future work.

Note that instead of replacing a concrete value with a corresponding abstract value,
we could have simply used a logical variable and symbolic execution to explore
possible interleavings. This would have allowed us to
\begin{enumerate*}
  \item use satisfiability modulo theory (\SMT{}) solvers to eliminate
  non-reachable branches, and,
  \item eliminate the need to develop an maintain an abstract semantics, as
  existing rules would work for both concrete and symbolic execution.
\end{enumerate*}

$\K$ has multiple backends, including a fast execution LLVM backend
that can only handle \emph{ground} patterns, i.e. patterns with logical
variables, and a haskell backend that can handle ML patterns in their
full generality \cite{KFrameworkBackendsUrl}. Symbolic execution requires the use of $\K$'s haskell
backend, which presented signficant performance related challenges when
handling $\MediK{}$ programs as large as the pediatric sepsis management
guidelines presented in our work. Our abstraction-based model checking was in part motivated by
said performance issues. We leave moving to symbolic execution based
program analysis as part of future work.

\section{Semantics Based Visual Representation Generation}\label{subsec:visual-representation-generation}

Extremely vital to the success of \MediK{}, the ability
to generate visual representation of \MediK{} programs to aid
comprehensibility to medical domain experts, enabling them to verify
accuracy of encoded medical knowledge. Even though \MediK{} code
structurally resembles paper-based guidelines, said
representations aid comprehensibility for medical
domain experts as the representations resemble their paper-based counterparts,
imparting a sense of familiarity.

Control Flow Graphs (\CFGs{}) form the basis of not only such visual
representations, but other important analyses. For example,
flow-sensitive analyzers can be used to verify that a medical procedure
is performed on both treatment branches.
\CFGs{} find use in applications from compilers to model checkers.
But, despite their importance, they are not well understood.
To illustrate the lack of a \emph{canonical} definition of a \CFG{},
in \cite{KoppelICFP22}, the authors
used three different \CFG{} generators
on the same program, and obtaining three different \CFGs{}, each
tailored to the objectives of the generator.

Thus, generation of \CFGs{} that present \emph{all relevant} information,
with \emph{minimal} intervention during generation, in a
\emph{correct-by-construction} fashion, is indeed challenging.
In \cite{KoppelICFP22} that authors argue that operational semantics
are not suitable to be used directly for generation of \CFGs{} by demonstrating
that many control flow edges correspond to different halves of the same rule,
while some correspond to none. To address this, the authors utilize an algorithm
to construct an Abstract Machines (AM) style semantics from the SOS.

Consider for example, the following SOS rules for assignment:
$$
\infer[AssnCong]
{x := e, \mu\; \leadsto\; x := e^{\prime}, \mu^{\prime}}{e, \mu\; \leadsto\; e^{\prime}\mu^{\prime}}
\quad \quad
\infer[AssnEval]
{x := v,\mu\; \leadsto\; \left(\text{skip}, \mu\left[x \rightarrow v\right]\right)}{}
$$

The authors then automatically generate an AM, written
$\left\langle \left(t, \mu\right) \;\mid\; K\right\rangle$ where $K$ is the
\emph{context} or \emph{continuation} composed of frames of the form $[x\ :=\
(\Box_{t}, \Box_{\mu})]$ indicating holes for an evaluated value and
environment. Thus, the AM rules are:
$$
\left\langle \left( x := e, \mu \right)\;\mid\; k \right\rangle \rightarrow
\left\langle \left( e, \mu \right)\;\mid\; k \circ \left[x := \left(\Box_t,\Box_{\mu}\right) \right]\right\rangle
$$
$$
\left\langle \left( v, \mu \right)\;\mid\; k \right\rangle \rightarrow
\left\langle \left(\text{skip}, \mu\left[x \rightarrow v \right] \right) \;\mid\; k \right\rangle
$$

Next, as it's possible for the AM to have an infinitely-many states, an
abstraction is required to collapse the states and make the transition system
finite. To this end, the authors utilize a \emph{value irrelevant} abstraction
that replaces all values with a single one $\star$ that representing any of
them. With this, the abstract states can be treated as nodes in the CFG.
Thus, program execution with this abstraction will produce a CFG.
For example, consider evaluation of statement $x := y$, under an
infinite number of environments, can lead to an infinite number of
states of the AM.
But, converting all values to $\star$ results in
one-possible execution: $\left\langle \left( x := y, \left[ y \mapsto \star
    \right] \right) \;\mid\; k \right\rangle$
$\rightarrow  \left\langle \left(y, \left[ y \mapsto \star \right] \right) \;\mid\; k \circ \left[ x := \left(\Box_t,\Box_u\right) \right] \right\rangle$
$\rightarrow  \left\langle \left(\star, \left[ y \mapsto \star \right] \right) \;\mid\; k \circ \left[ x := \left(\Box_t,\Box_u\right) \right] \right\rangle$
$\rightarrow  \left\langle \left(\text{skip}, \left[x \mapsto \star, y \mapsto \star \right] \right) \;\mid\; k \right\rangle$

Note that expressions built over $\star$-values can still pose a
challenge. For instance, a statement of the form \inlineimp{while e do s} when
evaluated under an environment $\left[\star \mapsto \star\right]$ requires
handling $\left\langle \left(e, \left[ \star \mapsto \star \right] \right)\;\mid\; k
\right\rangle$, which can be matched by any expression rule. To handle this,
$\left\langle \left(e, \left[ \star \mapsto \star \right] \right)\;\mid\; k \right\rangle$
is over-approximated to $\left \langle \left(\star, \left[ \star \mapsto \star \right] \right)\;\mid\; k \right\rangle$
as evaluating any such $e$ would eventually result in $\star$.


%However, this still requires manual specification of abstract
%semantics to collapse the inifinite transition system into a finite one.
%To this end, the
%authors utilize a \emph{value irrelevance} abstration, where all values
%are replaced by a single abstract value representing all of them: \startext.
%This makes the system finite without direct changes to the semantics.
%This approach enables them to canonically define any \CFG{} as a projection
%of the transition system an abstracted abstract state machine, where their
%treatment of Abstract Machine matches the one in \cite{VanhornArxiv10}.

In this work, we plan to utilize ideas from \cite{KoppelICFP22}
in conjunction with the $\K{}$-Summarizer toolchain. The $\K{}$-
Summarizer is an effort to improve performance of semantics-based tools
by modifying the $\K$-definition itself. One of the primary objectives
of the summarizer is to enable semantics-based compilation.
Semantics-based compilation can be considered a generalization of
established techniques for established techniques for program optimization
such as partial evaluation of programs \cite{Jones93Book}.

For a given language $L$, let $\llbracket\_\rrbracket$ be the
interpreter for L. Say, for given $p \in L$--$\text{Program}$, and inputs
$\text{in}_1, \text{in}_2$,
$\text{output} = \llbracket p \rrbracket\left[\text{in}_1,\text{in}_2\right]$
if $p$ terminates.
We can now express partial evaluation of program p, using another
program $\mix$, called so since performs a mix of both execution and code
generation. Now, we get a two stage execution process:
$$ p_{\text{in}_1} = \llbracket \mix \rrbracket [p, \text{in}_1]$$
$$ \text{output} = \llbracket p_{\text{in}_1} \rrbracket \text{in}_2$$
%  $$\llbracket \llbracket \mix \rrbracket \left[p, \text{in}_1\right] \rrbracket
%  \left[\text{in}_2\right]
%  = \llbracket p \rrbracket \left[\text{in}_1, \text{in}_2\right]$$
Now, for given program $p$, if $\text{in}_1$ is statically known (at stage 1),
and $\text{in}_2$ is dynamically known (at stage 2), then,
$\llbracket p_{\text{in}_1} \rrbracket$ can be computed \emph{once}, and
re-used for any new $\text{in}_2$.
Semantics-based compilation can be seen as a generalization of this idea.
Say $L$ is a $\K$ definition. We can generate a new definition
$L^{p_{\text{in}_1}}$ that directly implements $\llbracket p_{\text{in}_1}
\rrbracket$. At runtime, only the dynamically determined input $\text{in}_2$
to complete execution.

Conceptually, the summarization process results in a \CFG{}, where
each basic block is summarized as a $\K$ rule. But, in \MediK{}'s
case, the inherent concurrency presents additional challenges
that we intend to tackle.

While \SBC{} is one of the primary goals of the $\K{}$-Summarizer
project, it has also been used to optimize $\K$ definitions
by \emph{merging} multiple rules that take smaller steps into
one rule that \emph{summarizes} all of them. This is similar
to the translation from \SOS{} to abstract machine presented in
\cite{KoppelICFP22}. In our work, we plan to generate such
\emph{summaries} soundly, and utilize them to generate visualizations
of \MediK{} programs. Note, that both $\K$ Summarizer and the work
in \cite{KoppelICFP22} have been tried in \emph{deterministic} settings.
We plan to address additional challenges from \emph{non-determinism} in our
work.

\section{Applicability to other \CDSSs{}}\label{subsec:applicability}

Throughout this work, we have used the sepsis management \BPG{} as
a case study for clinical decision support systems.
While the \BPG{} is a real-world, complex
\CDSSs{} for management of sepsis in pediatric cases, it remains
to be seen if \MediK{} can act as a convenient medium for building \CDSSs{} in general.
To answer this, we plan to implement at least one more real-world \CDSS{}
using \MediK{}. To this end, we plan to express the American Heart Institute's
(AHA) \BPG{} for management of cardiac arrest \cite{AHAUrl}. Our choice
is motivated by the following reasons:
\begin{enumerate*}[label=(\roman*)]
  \item an effective \CDSS{} for cardiac arrest can significantly improve
    outcomes,
  \item enables us to evaluate the \MediK{} code against directly encoding
    the guideline in $\K$, as we had used the same \BPG{} for a $\K$-based \CDSSs{}
    (see section \ref{subsec:BPG-in-K}).
\end{enumerate*}


%Say $\varphi, \psi$ are $\K{}$ $\K$ configuration
%terms, i.e., terms  of the form \inlinekmath{<k> .. </k> <store> ... </store> ...}.






%
%
%containing axioms of the form $\varphi \to \snext\psi$ that
%encode rewrite rules $\varphi \To \psi$, where $\eventually\psi \equiv \mu X \ldot
%\psi \vee \snext X$. The symbol $\snext \in \Sigma$, referred to as the
%\say{one-path next}, defines a transition $\ToExec$ system as follows:

