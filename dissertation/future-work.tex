\chapter{Future Work}

This chapter outlines major tasks that remain towards enhancing the capabilities
of our system.

\section{Soundness of Model Checking Abstraction}\label{sec:abstraction-soundness}

In \autoref{sec:formal-analysis}, we introduced an abstraction that enabled us
to model check the system. Instead of interpreting $\MediK{}$ programs
concretely, we introduced rules that operate over abstract values
that hold semantic meaning. However, we did not establish that the analysis is
indeed sound. In this section, we briefly outline the steps to prove the
soundness of our abstraction-based analysis, and outline limitations that
prevented us from establishing it.

Recall that in \autoref{sec:theoretical-foundations}, we presented the
matching logic: the theoretical foundations of the $\K$ framework. We explained
that a $\K$ definition of a language $L$, compiles to a matching logic theory
$\Gamma^L$. A symbol $\snext \in \Sigma$ is then defined as a part of the
signature to capture dynamic behavior of rewrite rules.
Intuitively, given a configuration
pattern $\gamma$, the pattern $\snext \gamma$ matches all possible
configurations that $\gamma$ can rewrite to in one-step. Thus
the one-step rewrite relation can be defined as:
$$
  \varphi \To^1 \psi \equiv \varphi \to \snext \psi
$$

Next, the reflexive, transitive closure of the one-step rewrite relation
can be defined using fixed-points as:
$$
  \eventually \varphi \equiv \mu X \ldot \varphi \vee \snext X
$$
and the rewrite $\varphi \To \psi$ as:
$$
  \varphi \To \psi \equiv \varphi \to \eventually \psi
$$
In other words, if a configuration matches
$\varphi$ then it matches $\varphi$ in zero or more steps.
Note that the existence of a fixed point of $\eventually\varphi$
is guaranteed by the Knaster-Tarski theorem as long as the interpretation
$\mathcal{F}^{\rho}_{X,\varphi} = \interpret{\varphi}{\rho\left[A / X\right]}$
is monotone \cite{ChenLICS19}.

%When $\varphi,\psi$ are \inlinek{configuration} patterns,

Typically, establishing soundness of abstraction-based static analysis
requires coming up partial order over abstract states,
$\left(\Abs, \sqsubseteq\right)$, and, semantic rules defined over $\Abs$.
For $a_1, a_2 \in \Abs$, $a_1 \sqsubseteq a_2$ typically means $a_1$ is
\emph{more precise} than $a_2$. Recall that in \autoref{sec:formal-analysis},
that we defined a value \inlinek{#nondet}, denoting the very simple abstract
domain $\top$ or \emph{undetermined},
and rules that operate over it. Let $\ToAbs$ be the transition
system defined by rules over the abstract state. We first need to
establish that for any $a_1, a_2 \in \Abs$, if $a_1 \ToAbs a_2$ then $a_1
\sqsubseteq a_2$. In other words, we need to show that the transition relation
$\ToAbs$ is monotonic w.r.t. the ordering $\sqsubseteq$.




%
%
%containing axioms of the form $\varphi \to \snext\psi$ that
%encode rewrite rules $\varphi \To \psi$, where $\eventually\psi \equiv \mu X \ldot
%\psi \vee \snext X$. The symbol $\snext \in \Sigma$, referred to as the
%\say{one-path next}, defines a transition $\ToExec$ system as follows:

